%!TEX root = ../main.tex
\chapter{Trace and Residue}\label{ch:trace-and-residue}
\section{Finitepotent maps and their trace}
Let $k$ be a fixed ground field and $V$ a vector space over $k$. In this section we will expand the notion of trace of a linear endomorphism to include certain operators even when $V$ is infinite dimensional.
\subsection*{Finitepotent maps}
\begin{definition}\label{def:finitepotent}
	We will say a linear map $f\colon V \to V$ is \textbf{finitepotent} if
	\[
		\dim f^{n}(V) < \infty
	\]
	for sufficiently large $n$.
\end{definition}
We characterize finitepotent maps as follows.
\begin{lemma}\label{lemm:characterization-of-finitepotent-maps}
	A linear map $f\colon V \to V$ is finitepotent if and only if there exists a subspace $W \subseteq V$ such that
	\begin{enumerate}[label = (\roman*)]
		\item $\dim f(W) < \infty$,
		\item $f(W) \subseteq W$,
		\item the induced map $\bar{f}\colon V/W \to V/W$ is nilpotent.
	\end{enumerate}
\end{lemma}
\begin{proof}
	If $f$ is finitepotent choose $W = f^{n}(V)$ for sufficiently large $n$. The first condition follows from definition. Also, $f(W) = f^{n+1}(V) \subseteq f^{n}(V) = W$. In addition, $\bar{f}^{n} = 0$. On the other hand, if such $W$ exists, note that condition (ii) assures that $\bar{f}$ is well defined. Moreover, as $\bar{f}$ is nilpotent, $f^{n}V \subseteq W$ for sufficiently large $n$ and by condition (i) above $\dim f^{n}(V) < \infty$.
\end{proof}
\subsection*{Trace}
If $f$ is a finitepotent map and $W$ is as above, $\tr_{V}(f)\in k$ may be defined as $\tr_{W}(f)$ where $\tr_{W}(f)$ is the ordinary trace of $f$ viewed as a endomorphism of $W$. First, we will check that this definition does not depend on the choice of $W$. Suppose $W_{1}, W_{2} \subseteq V$ suffice the properties on \cref{lemm:characterization-of-finitepotent-maps}, then $W = W_{1} + W_{2}$ suffices them too. Hence, as the induced maps on $W/W_{1}$ and $W/W_{2}$ are nilpotent, they have have zero ordinary trace and since
\begin{align*}
	\tr_{W}(f) &= \tr_{W_{1}}(f) + \tr_{W/W_{1}}(f) \\
	\tr_{W}(f) &= \tr_{W_{2}}(f) + \tr_{W/W_{2}}(f),
\end{align*}
we obtain $\tr_{W_{1}}(f) = \tr_{W_{2}}(f)$, our desired result. 

This definition extends some of the properties of the ordinary trace.
\begin{lemma}\label{lemm:properties-trace}
	\begin{enumerate}[label = (\alph*)]
		\item If $\dim V < \infty$, any endomorphism $f$ is finitepotent and $\tr_{V}(f)$ coincides with the ordinary trace.
		\item If $f$ is nilpotent, then it is finitepotent and $\tr_{V}(f) = 0$.
		\item If $f$ is finitepotent and $U$ is a subspace such that $f U \subseteq U$ then the induced maps on $U$ and $V/U$ are finitepotent and satisfy
		\[
		 	\tr_{V}(f) = \tr_{U}(f) + \tr_{V/U}(f)
		 \] 
	\end{enumerate}
\end{lemma}
\begin{proof}
	Both (a) and (b) are immediate. For (c) if $W$ suffices the properties in \cref{lemm:characterization-of-finitepotent-maps} for $f$ then $W\cap U$ and $(W + U)/U$ suffice them for the induced maps, that is, they're finitepotent. Since $W/(W\cap U) \cong W+U/U$, the diagram
	\[
		\begin{tikzcd}
			W/(W\cap U) \arrow[r, "\cong"] \arrow[d, "f"] & (W+U)/U \arrow[d, "f"] \\
			W/(W\cap U) \arrow[r, "\cong"] & (W+U)/U
		\end{tikzcd}
	\]
	commutes and trace is invariant under conjugation, we get $\tr_{W/(W\cap U)}(f) = \tr_{(W+U)/U}(f)$. Hence
	\[
	 	\tr_{V}(f) = \tr_{W}(f) = \tr_{W\cap U}(f) + \tr_{(W+U)/U}(f) = \tr_{U}(f) + \tr_{V/U}(f)
	\]
\end{proof}
\begin{definition}\label{def:finitepotent subspace}
	A subspace $F$ of $\End_{k}(V)$ is said to be a \textbf{finitepotent subspace} if there exists an $n$ such that for any family of $n$ elements $f_{1}, \ldots, f_{n}\in F$, the space $f_{1}f_{2}\cdots f_{n}V$ is finite dimensional.
\end{definition}
The following is the natural linearity property for $\tr$.
\begin{proposition}\label{prop:linearity-trace}
	If $F$ is a finitepotent subspace then $\tr_{V}\colon F\to k$ is $k$-linear
\end{proposition}
\begin{proof}
	It is enough to prove it in the case that $F$ is finite dimensional. Let $W = F^{n}V$ for $n$ as in the definition of finitepotent subspace, thus $\dim W < \infty$. Hence, for all $f\in F$, $W$ suffices the conditions in \cref{lemm:characterization-of-finitepotent-maps}. It follows that $\tr_{V}(f) = \tr_{W}(f)$ which is linear.
\end{proof}
\textcolor{red}{add note in ``general'' linearity of trace when .bib is ready}
\begin{proposition}\label{prop:trace-inavriant-under-commutator}
	If $f,g \in \End_{k}(V)$ and $fg$ is finitepotent then $gf$ is also finitepotent and 
	\[
		\tr_{V}(fg) = \tr_{V}(gf).
	\]
\end{proposition}
\begin{proof}
	Since $fg$ is finitepotent let $W =(fg)^{n}V$ for sufficiently large $n$ has finite dimension. On the other hand, $(gf)^{n+1}V= g(fg)^{n}f(V) \subseteq g(W)$, therefore, $gf$ is also finitepotent. Let $W' = (gf)^{n}V$, then $g(W') \subseteq W$ and $f(W) \subseteq W'$. Thus, 
	\[
		\dim W' \leq \dim g(W) \leq \dim W\quad\text{and,}\quad\dim W \leq \dim f(W) \leq \dim W',
	\]
	which implies that $W\cong W'$ and that $g$ and $f$ induce mutually inverse isomorphism between $W$ and $W'$. Moreover, the diagram
	\[
		\begin{tikzcd}
			W \arrow[r, "fg"]\arrow[d, "g"] & W \arrow[d, "g"] \\
			W' \arrow[r, "gf"] & W'
		\end{tikzcd}
	\]
	commutes. We conclude $\tr_{W}(fg) = \tr_{W'}(gf)$ and it follows $\tr_{V}(fg) = \tr_{V}(gf)$.
\end{proof}






