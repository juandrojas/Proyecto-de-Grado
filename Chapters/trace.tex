%!TEX root = ../main.tex
\chapter{Trace and Residue}\label{ch:trace-and-residue}
We extend the definition of trace to a certain class of infinite rank endomorphisms in order to define an abstract residue. We follow the original structure of Tate's elegant article \cite{Tate} while translating his statements in the language of Tate's Linear Algebra.
\section{Finite-potent maps and their trace}
Let $k$ be a fixed field and $V$ a vector space over $k$. In this section we will extend the notion of trace of a linear endomorphism to include certain operators even when $V$ is infinite dimensional.
\subsection{Finite-potent maps}
\begin{definition}\label{def:finite-potent}
	We will say a linear map $f\colon V \to V$ is \textbf{finite-potent} if
	\[
		\dim f^{n}(V) < \infty
	\]
	for sufficiently large $n$.
\end{definition}
The following is characterization of finite-potent endomorphisms.
\begin{lemma}\label{lemm:characterization-of-finite-potent-maps}
	A linear map $f\colon V \to V$ is finite-potent if and only if there exists a subspace $W \subseteq V$ such that
	\begin{enumerate}[label = (\roman*)]
		\item $\dim f(W) < \infty$,
		\item $f(W) \subseteq W$,
		\item the induced map $\bar{f}\colon V/W \to V/W$ is nilpotent.
	\end{enumerate}
	A subspace $W$ is a \textbf{trace-subspace} for $f$ if satisfies the previous previous properties.   
\end{lemma}
\begin{proof}
	If $f$ is finite-potent choose $W = f^{n}(V)$ for sufficiently large $n$. The first condition follows from definition. Also, $f(W) = f^{n+1}(V) \subseteq f^{n}(V) = W$. In addition, $\bar{f}^{n} = 0$. On the other hand, if such $W$ exists, note that condition (ii) assures that $\bar{f}$ is well defined. Moreover, as $\bar{f}$ is nilpotent, $f^{n}V \subseteq W$ for sufficiently large $n$ and by condition (i) above $\dim f^{n}(V) < \infty$.
\end{proof}
Observe that a trace-subspace for a finite-potent map $f$ is not unique. In particular, if $W$ is trace-subspace for $f$ then $f^{n}(W)$ is trace-subspace for $f$ for all $n$.
\begin{notation}\label{not:trace}
	If $f$ is a finite-rank endomorphism in a vector space $V$ we will denote its ordinary trace by $\tr_{V}(f)$. Moreover, if $W$ is a subspace of $V$ invariant under $f$, that is, $f(W) \subseteq W$ then $\tr_{W}(f) := \tr_{W}(f\lvert_{W})$. In addition, if $\overline{f}$ is the induced map such that the following diagram commutes
	\[
	\begin{tikzcd}
		V \arrow[r, "f"] \arrow[d, "\pi_{W}"] & V \arrow[d, "\pi_{W}"] \\
		V/W \arrow[r, "\overline{f}"] & V/W
	\end{tikzcd} 
	\]
	then $\tr_{V/W}(f) := \tr_{V/W}(\overline{f})$. The use of this notation is consistent throughout the document. 
\end{notation}
\subsection{Trace}  If $f$ is a finite-potent map and $W$ is a trace-subspace for $f$ the \textbf{trace} $\tr_{V}(f)\in k$ of $f$ may be defined as
\[
	\tr_{V}(f) = \tr_{W}(f)
\]
Observe that $\tr_{W}(f)$ is well-defined because $f\lvert_{W}$ is of finite-rank. 
\begin{proposition}\label{prop:trace-does-not-depend-on-W}
	The definition of $\tr_{V}$ does not depend on the choice of trace-subspace for $f$.
\end{proposition}
\begin{proof}
	Suppose $W_{1}, W_{2} \subseteq V$ are two trace-subspaces for $f$ then $W = W_{1} + W_{2}$ is trace-subspace for $f$ as well. Hence, the induced maps on $W/W_{1}$ and $W/W_{2}$ are nilpotent. Therefore, $\tr_{W/W_{1}}(f) = \tr_{W/W_{2}}(f) = 0$ and using the well-known identify of the ordinary trace
\begin{align*}
	\tr_{W}(f) &= \tr_{W_{1}}(f) + \tr_{W/W_{1}}(f) \\
	\tr_{W}(f) &= \tr_{W_{2}}(f) + \tr_{W/W_{2}}(f),
\end{align*}
we obtain $\tr_{W_{1}}(f) = \tr_{W_{2}}(f)$, our desired result. 
\end{proof}
This definition extends some of the properties of the ordinary trace.
\begin{lemma}\label{lemm:properties-trace}
	\begin{enumerate}[label = (\alph*)]
		\item If $\dim V < \infty$, any endomorphism $f$ is finite-potent and $\tr_{V}(f)$ coincides with the ordinary trace.
		\item If $f$ is nilpotent, then it is finite-potent and $\tr_{V}(f) = 0$.
		\item If $f$ is finite-potent and $U$ is a subspace such that $f(U) \subseteq U$ then the induced maps on $U$ and $V/U$ are finite-potent and satisfy
		\[
		 	\tr_{V}(f) = \tr_{U}(f) + \tr_{V/U}(f)
		 \] 
	\end{enumerate}
\end{lemma}
\begin{proof}
	Both (a) and (b) are immediate. For (c) if $W$ is a trace-subspace for $f$ then $W\cap U$ and $(W + U)/U$ are trace-subspaces for the induced maps respectively. Hence, by \cref{lemm:characterization-of-finite-potent-maps} both induced maps are finite-potent. Since $W/(W\cap U) \cong (W+U)/U$, the diagram
	\[
		\begin{tikzcd}
			W/(W\cap U) \arrow[r, "\cong"] \arrow[d, "f"] & (W+U)/U \arrow[d, "f"] \\
			W/(W\cap U) \arrow[r, "\cong"] & (W+U)/U
		\end{tikzcd}
	\]
	commutes. Moreover, recall that the ordinary trace is invariant under conjugation, that is, $\tr_{W}(\varphi \circ f \circ \varphi^{-1}) = \tr_{W}(f)$ for every automorphism $\varphi$ of $W$. Therefore, it follows that $\tr_{W/(W\cap U)}(f) = \tr_{(W+U)/U}(f)$. We conclude that
	\[
	 	\tr_{V}(f) = \tr_{W}(f) = \tr_{W\cap U}(f) + \tr_{(W+U)/U}(f) = \tr_{U}(f) + \tr_{V/U}(f).
	\]
\end{proof}
\begin{definition}\label{def:finite-potent subspace}
	A subspace $F$ of $\End_{k}(V)$ is said to be a \textbf{finite-potent subspace} if there exists an $n$ such that for any family of $n$ elements $f_{1}, \ldots, f_{n}\in F$, the space $f_{1}f_{2}\cdots f_{n}V$ is finite dimensional.
\end{definition}
Observe that if $F$ is a finite-potent subspace of $\End_{k}(V)$ then every $f\in F$ is finite-potent. 
\begin{proposition}\label{prop:linearity-trace}
	If $F$ is a finite-potent subspace then $\tr_{V}\colon F\to k$ is $k$-linear
\end{proposition}
\begin{proof}
	It is enough to prove it in the case that $F$ is finite dimensional. Let $W = F^{n}V$ for $n$ as in the definition of finite-potent subspace, thus $\dim W < \infty$. Hence, $W$ is a trace-subspace for all $f\in F$. It follows that $\tr_{V}(f) = \tr_{W}(f)$ for all $f$. Since $\tr_{W}\colon \End_{k}(V) \to k$ is $k$-linear, so is $\tr_{V}\colon F \to k$.
\end{proof}
\begin{remark}\label{rem:general-linearity-trace}
	In his paper, Tate asked if general linearity for finite-potent maps followed. His question was answered negatively in \cite{TATE-TRACE-COUNTER-EXAMPLE} where general linearity is reduced to the following: if the sum of two nilpotent endomorphisms is finite-potent, is the sum necessarily traceless?  
\end{remark}
\begin{proposition}\label{prop:trace-inavriant-under-commutator}
	If $f,g \in \End_{k}(V)$ and $fg$ is finite-potent then $gf$ is also finite-potent and 
	\[
		\tr_{V}(fg) = \tr_{V}(gf).
	\]
\end{proposition}
\begin{proof}
	Since $fg$ is finite-potent then $W =(fg)^{n}V$ is finite-dimensional for a sufficiently large $n$. On the other hand, $(gf)^{n+1}V= g(fg)^{n}f(V) \subseteq g(W)$, therefore, $gf$ is also finite-potent. Let $W' = (gf)^{n}V$, then $g(W') \subseteq W$ and $f(W) \subseteq W'$. Thus, 
	\[
		\dim W' \leq \dim g(W) \leq \dim W,
	\]
	and, 
	\[
		\dim W \leq \dim f(W) \leq \dim W',
	\]
	which implies that $W\cong W'$ and that $g$ and $f$ induce mutually inverse isomorphisms between $W$ and $W'$. Moreover, the diagram
	\[
		\begin{tikzcd}
			W \arrow[r, "fg"]\arrow[d, "g"] & W \arrow[d, "g"] \\
			W' \arrow[r, "gf"] & W'
		\end{tikzcd}
	\]
	commutes. We conclude $\tr_{W}(fg) = \tr_{W'}(gf)$ and it follows $\tr_{V}(fg) = \tr_{V}(gf)$.
\end{proof}
\subsection{Trace and Tate Spaces}\label{Tate-and-trace}
Suppose that $V$ is a Tate space and consider $\End_{k}(V)$ the space of continuous endomorphisms of $V$. By \cref{prop:compact-and-discrete-2-sided-ideal} and \cref{prop:discrete-compact-operators-present-the-whole-space} we have 2-sided ideals $\End_{0}(V), \End_{+}(V)$ and $\End_{-}(V)$ of $E$ such that $\End_{+}(V) + \End_{-}(V) = E$ and $\End_{0}(V) = \End_{+}(V) \cap \End_{-}(V)$. Moreover, \cref{rem:discrete-composition-compact} implies that $\End_{0}(V)$ is a finite-potent subspace.
\begin{lemma}\label{lemm:traceless-commutator}
	Suppose $f \in \End_{+}(V)$ and $g \in \End_{-}(V)$ or $f \in \End_{-}(V)$ and $g\in \End_{+}(V)$. Then the commutator $[f,g] = fg - gf$ belongs to $\End_{0}(V)$ and it is traceless.
\end{lemma}
\begin{proof}
	This immediate from the previous discussion and \cref{prop:trace-inavriant-under-commutator}.
\end{proof}

\section{Differential Calculus}
In this section we introduce the theory of derivations and differentials over an arbitrary commutative $k$-algebra $A$. Let $M$ be an $A$-module. We follow \cite{EGA4} Section 20 and \cite{Matsumura} Section 25. 
\begin{definition}\label{def:derivations-and-differentials}
	A \textbf{derivation} from $A$ to $M$ is a map $D\colon A \to M$ satisfying properties
	\begin{enumerate}[label = (\roman*)]
			\item $D(a + b) = D(a) + D(b)$ and,
			\item (\textit{Leibniz Rule}) $D(ab) = aD(b) + bD(a)$ 
	\end{enumerate}
for all $a,b \in A$.

The set of derivations from $A$ to $M$ is an $A$-module in the natural way. We will denote it by $\Der(A,M)$. Moreover if $A$ is a $k$-algebra through a map $\varphi\colon k \to A$ we say that $D$ is a $\mathbf{k}$-\textbf{derivation} if $D$ is a derivation and $D \circ \varphi = 0$. In this case, the set of all $k$-derivations is denoted $\Der_{k}(A,M)$. If $M = A$, we will denote $\Der_{k}(A,A)$ simply by $\Der_{k}(A)$.
\end{definition}
\begin{definition}\label{def:extension-of-algebras}
Let $B$ be a $k$-algebra and $C$ an ideal in $B$ with $C^{2} = 0$; set $A =B/C$. In this way, $C$ can be viewed as an $A$-module. In this situation we say that $B$ is an \textbf{extension} of the $k$-algebra $A$ by the $A$-module $C$. Usually, we simply write the exact sequence
\[\label{eqn:extension}
	0 \to C \to B \xrightarrow{\pi} A \to 0
\]
As usual, we will say that such sequence \textbf{splits} if there exists a retraction; that is, a $k$-algebra homomorphism $\rho\colon A \to B$ such that $\pi \circ \rho = 1_{A}$. In this case we can identify $B = C\oplus A$. Conversely, starting from any $k$-algebra $A$ and any $A$-module $C$, one can always define a structure on $A \oplus C$ such that $A\oplus C$ is an extension of $A$ by $C$. Namely, 
\[
			(a,c)(a',c') = (aa',ac' + a'c)
\]
for $a,a'\in A$ and $c,c' \in C$. Common notations for this algebra are $D_{A}(C)$ or $A * C$.		
\end{definition}
\begin{definition}\label{def:lifting}
Given a commutative diagram of $k$-algebras
\[
	\begin{tikzcd}
		 	B \arrow[r, "f"] & A \\
		 	& C\arrow[lu, "h"] \arrow[u, "g"]
	\end{tikzcd} 
\]
thinking of $f$ as a fixed map; we say that $h$ is a \textbf{lifting} of $g$ to $B$. 
\end{definition}
\begin{lemma}\label{lemm:two-liftings}
	Let $h$ and $h'\colon C \to B$ be two liftings of $g$ to $B$. Let $K = \ker f$ and suppose $K^{2} = 0$. Then, it follows that $h - h'\colon C \to K$ is a $k$-derivation. Conversely, if $D \in \Der_{k}(C, K)$ then $h + D$ is another lifting of $g$ to $B$.
\end{lemma}
\begin{proof}
	First, observe that $(h - h')(C)$ lies in $K$ because both $h$ and $h'$ are liftings of $g$ to $B$. Since $K^{2} = 0$, then $K$ can be considered as $f(B)$-module and by means of $g$ as a $C$-module. Then, $h-h'\colon C \to K$ is a map of $C$-modules. Now, let $c,c' \in C$ then
	\begin{align*}
	(h -h')(cc') &= h(c)h(c') - h'(c)h'(c') \\
	&= h(c)h(c') - h'(c)h'(c') - h(c)h'(c') + h'(c')h(c) \\
	\end{align*}
	since $c \cdot k = h(c)k = h'(c)k$ for all $k \in K$ it follows that
	\begin{align*}
		(h -h')(cc') &= c\cdot h(c') - c'\cdot h'(c') - c\cdot h'(c') + c'\cdot h(c) \\
		&= c\cdot (h - h')(c') + c'\cdot (h - h')(c)
	\end{align*}
	which implies that $h - h'$ is a $k$-derivation. Observe that $h + D$ is a lifting because $D(C)$ lies in $K$.
\end{proof}
\begin{theorem}\label{prop:differentials-representable-functor}
	If $A$ is a $k$-algebra, consider the covariant functor from the category $\mathsf{Mod}_{A}$ to itself given by $M \mapsto \Der_{k}(A,M)$. This functor is representable. 
\end{theorem}
\begin{proof}
	Let $\mu\colon A \otimes_{k} A \to A$ be the $k$-algebra homomorphism given by $f \otimes g \to fg$. Set
	\[
		I = \ker \mu, \quad \Omega_{A/k} = I/I^{2}, \quad \text{and,} \quad B = (A \otimes_{k} A)/I^{2}.
	\]
	Thus, $\mu$ induces $\mu'\colon B \to A$ such that
	\[
		0 \to \Omega_{A/k} \to B \to A \to 0
	\]
	is an extension of $A$ by $\Omega_{A/k}$. We claim that this extension splits. Moreover it has two splittings, by considering retractions
	\[
		j_{1}\colon A \to B \quad\text{and,}\quad j_{2}\colon A \to B,
	\]
	defined by $a \mapsto a\otimes 1$ mod $I^{2}$ and $a \mapsto 1 \otimes a$ mod $I^{2}$. By \cref{lemm:two-liftings} $d := j_{2} - j_{1}$ is a $k$-derivation of $A$ to $\Omega_{A/k}$. Now, we prove that 
	\begin{align}\label{eqn:representable}
		\Der_{k}(A,M) &\cong \Hom_{A}(\Omega_{A/k}, M).
	\end{align}	
	Let $D \in \Der_{k}(A,M)$ and define $\varphi\colon A \otimes_{k} A \to A * M$ by $\varphi(x \otimes y) = (xy, xD(y))$ then $\varphi$ is a $k$-algebra homomorphism since it is compatible with the operation in $A * M$ defined in \cref{def:extension-of-algebras}. In addition, if $\sum x_{i} \otimes y_{i}$ lies in $I$ then
	\[
		\mu\left(\sum x_{i} \otimes y_{i}\right) = \sum x_{i}y_{i} = 0 \implies \varphi\left(\sum x_{i} \otimes y_{i}\right) = (0, \sum x_{i}D(y_{i}))
	\]
	whence $\varphi(I)$ lies in $M$. Moreover, by Leibniz's Rule $\varphi$ factors through $I^{2}$ yielding a map $f\colon \Omega_{A/k}\to M$. For $a \in A$ it follows that 
	\begin{align*}
		f(da) &= f(1 \otimes a - a \otimes 1 \mod I^{2}) = \varphi(1 \otimes a) - \varphi(a \otimes 1) \\
		&= D(a) - aD(1) = D(a). 
	\end{align*} 	
	Therefore, $D = f\circ d$. Now, we prove that such $f$ is unique. First, observe that $\Omega_{A/k}$ has the $A$-module structure induced by multiplication by $a \otimes 1$ (or $1 \otimes a$ since $1 \otimes a - a \otimes 1 \in I$). Therefore, if $\xi = \sum x_{i} \otimes y_{i} \mod I^{2} \in \Omega_{A/k}$ then $a\xi = \sum ax_{i}\otimes y_{i} \mod I^{2}$, and $f(a\xi) = \sum ax_{i}D(y_{i}) = af(\xi)$, so that $f$ is $A$-linear. We have
	\[
		a \otimes a' = (a \otimes 1)(1 \otimes a' - a' \otimes 1) +  aa' \otimes 1
	\]
	so that if $\omega = \sum x_{i}\otimes y_{i} \in I$ then $\omega \mod I^{2} = \sum x_{i} dy_{i}$ since $\sum x_{i}y_{i} = 0$. We conclude that $\{da\mid a\in A\}$ is a set of generators for the $A$-module $\Omega_{A/k}$. This implies uniqueness of $f$. Therefore, (\ref{eqn:representable}) holds.
\end{proof}
\begin{definition}\label{def:module-of-differentials}
	The module $\Omega_{A/k}$ introduced in the proof of the previous theorem is called \textbf{module of differentials} of $A$ over $k$ or \textbf{module of Kähler differentials}, and for $a \in A$ the element $da\in \Omega_{A/k}$ is called the \textbf{differential} of $a$.
\end{definition}
\begin{example}\label{ex:differentials-of-polynomials}
	If $A$ is generated as $k$-algebra by a subset $S \subseteq A$ then $\Omega_{A/k}$ is generated by $\{ds\mid s \in S\}$. Indeed, if $a \in A$ then there exist $a_{i}\in S$ and a polynomial $f(X) \in k[X_{1}, \ldots, X_{n}]$ such that $a = f(a_{1}, \ldots, a_{n})$. Thus,
	\[
		da = \sum_{i=1}^{n} f_{i}(a_{1}, \ldots, a_{n})da_{i} \quad \text{where}\quad f_{i} = \frac{\partial f}{\partial x_{i}}.
	\]
	In particular, if $A = k[X_{1}, \ldots, X_{n}]$ then $\Omega_{A/k} = A dX_{1} + \ldots A dX_{n}$ since $X_{1}, \ldots, X_{n}$ are linearity independent; this follows from the fact that $\partial_{i} X_{j} = \delta_{ij}$.
\end{example}
\begin{lemma}\label{lemm:c-map-differentials}
	Let $K$ be a $k$-commutative algebra. The map $c\colon K \otimes_{k} K \to \Omega_{K/k}$ defined by $c(f \otimes g) = fdg$ satisfies:
	\begin{enumerate}[label = (\roman*)]
		\item $c$ is surjective.
		\item $\ker c$ is generated over $k$ by the elements of the form
		\[
			f \otimes gh - fg \otimes h - fh \otimes g
		\]
	\end{enumerate}
\end{lemma}
\begin{proof}
	The $k$-bilinear map $(f,g) \mapsto fdg$ induces $c$. Since $\{df \mid f\in K\}$ is a generating set for $\Omega_{K/k}$ as a $K$-module it follows that $c$ is surjective. For (b), observe that it is equivalent showing that $\ker(c)$ is generated over $K$ by the elements of the form $1 \otimes gh - g \otimes h - h \otimes g$. Let $A$ be the $K$-module generated by those elements. We wish to prove that
	\[
		A \to K \otimes_{k} K \to \Omega_{K/k} \to 0
	\]
	is exact. By left-exactness of $\Hom$ it is equivalent to prove that for all $K$-modules $M$ the induced sequence
	\[
		0 \to \Hom_{K}(\Omega_{K/k}, M) \to \Hom_{K}(K \otimes_{k} K, M) \to \Hom_{K}(A, M) 
	\]
	is exact. By \cref{prop:differentials-representable-functor} there is a canonical isomorphism $\Hom_{K}(\Omega_{K/k}, M) \cong \Der_{k}(K, M)$. Under this identification, we wish to prove that
	\[
		0 \to \Der_{k}(K, M) \to \Hom_{K}(K \otimes_{k} K, M) \to \Hom_{K}(A, M) 
	\]
	is exact. Observe that the first map is given by $D \mapsto \varphi_{D}$ where $\varphi_{D}(f \otimes g) = f D(g)$. Note that the restriction $\varphi_{D}\colon A \to M$ is trivial. Indeed,
	\[
		\varphi_{D}(1 \otimes gh - g \otimes h - h \otimes g) = D(gh) - gD(h) - hD(g) = 0
	\]
	by the Leibniz rule. Now, let $\psi\in \Hom_{K}(K \otimes_{k} K, M)$ so that $\psi(A) = 0$. Let $D_{\psi}\colon K \to M$ be the $k$-derivation defined by $f \mapsto \psi(1 \otimes f)$. First, we prove that $\psi_{D}$ is a $k$-derivation. Observe that $k$-linearity is obvious. Now, we prove the Leibniz rule for $D_{\psi}$. Consider
	\begin{align*}
		D_{\psi}(fg) = \psi(1 \otimes fg) &= \psi(f \otimes g + g \otimes f) \\ &= f \psi (1 \otimes g) + g \psi (1 \otimes f) \\ &= f D_{\psi} (f) + g D_{\psi} (g),
	\end{align*}
	where the third equality follows from the fact that $\psi$ vanishes in $A$. Finally, it is clear that $\varphi_{D_{\psi}} = \psi$. 
\end{proof}
\textcolor{red}{Further theory to be included if necessary.}
\section{Abstract Residue and its properties}
Throughout this section let $k$ be a field, $K$ a commutative $k$-algebra with $1$, and $V$ a $K$-module so that when viewed as a $k$-vector space it is a Tate space and $K$ acts continuously on $V$. Namely, for all $f \in K$ the map 
\begin{align*}
	f\colon &V \to V \\
	&x \mapsto fx 
\end{align*}
is continuous. In this way, $K$ operates on $V$ through $\End_{k}(V)$ (maintaining notation from \cref{Tate-and-trace}). We will not notationally distinguish $f\in K$ from its induced map in $\End_{k}(V)$.
\begin{lemma}\label{lemm:trace-only-depends-on-K}
	Let $f,g \in K$. Then, there are $f_{+}, g_{+} \in \End_{+}(V)$ so that 
	\[
	f = f_{+} \mod \End_{-}(V) ,\quad g = g_{+} \mod \End_{-}(V)
	\]
	and, the equality
	\[
		\tr([f_{+},g_{+}]) = \tr([f, g_{+}]) = \tr([f_{+}, g])
	\]
	holds.
\end{lemma}
\begin{proof}
The existence of $f_{+}$ and $g_{+}$ is immediate from the fact that $\End_{k}(V) = \End_{+}(V) + \End_{-}(V)$.  Clearly $[f_{+},g_{+}]\in \End_{+}(V)$ and the fact that $K$ is commutative implies that $[f,g] = 0$. Therefore,
\[
	[f_{+},g_{+}] = [f,g] \mod \End_{-}(V).
\]
Hence, $[f_{+},g_{+}]\in E_0$. Similarly, $[f,g_{+}]$ and $[f_{+}, g]$ belong to $\End_{0}(V)$. Whence, one can consider their trace. Furthermore, $f_{+} \in \End_{+}(V)$ and $g_{+} - g \in \End_{-}(V)$ thus $\tr([f_{+}, g_{+} - g]) = 0$ by \cref{lemm:traceless-commutator}; we conclude $\tr([f_{+}, g_{+}]) = \tr([f_{+}, g])$. The other equality follows similarly.
\end{proof}
\begin{notation}\label{not:plus-and-minus}
	\cref{lemm:trace-only-depends-on-K} implies that common values of traces $[f_{+},g_{+}]$, $[f_{+}, g]$ and $[f, g_{+}]$ depend only on $f$ and $g$ and not in the choice of $f_{+}$ and $g_{+}$. Therefore, we will always denote $f_{\pm}$ to be elements in $\End_{\pm}(V)$ such that
	\[
		f = f_{+} \mod \End_{-}(V) ,\text{ and } f = f_{-} \mod \End_{+}(V).
	\] 
	Choices of $f_{\pm}$ are not unique, but for practical reasons we will not worry about those issues. 
\end{notation}
\cref{lemm:trace-only-depends-on-K} implies that the assignment $(f,g) \mapsto \tr([f_{+}, g_{+}])$  is well-defined. Observe that this assignment $k$-bilinear by \cref{prop:linearity-trace}. Thus, there exists a map
\begin{align*}
	r\colon &K\otimes_{k} K \to k \\
	&f \otimes g \mapsto \tr([f_{+}, g_{+}]).
\end{align*}
With these tools at our hands we are ready to prove the existence of residue.
\begin{theorem}\label{thm:existence-of-residue}
	There exists a unique $k$-linear map
	\[
		\res_{V}\colon \Omega_{K/k}\to k
	\]
	such that for each pair of elements $f,g \in K$ we have
	\[
		\res_{V}(fdg) = \tr([f_{+}, g_{+}]).
	\]
\end{theorem}
\begin{proof}
	Let $c\colon K \otimes_{k} K \to \Omega_{K/k}$ be as in \cref{lemm:c-map-differentials}. Then, since $c$ is surjective, $\res_{V}$ if it exists it is uniquely determined by the commutativity of the following diagram
	\[
	\begin{tikzcd}
		K \otimes_{k} K \arrow[r, "r"] \arrow[d, two heads, "c"] & k \\
		\Omega_{K/k} \arrow[ur, dotted, swap, "\res_{V}"] &
	\end{tikzcd}
	\]
	Therefore, such map exists if and only if it vanishes on $\ker c$. To see this, let $f,g$ and $h$ in $K$ and choose $f_{+}, g_{+}$ and $h_{+}$ in $\End_{+}(V)$ coinciding with $f, g$ and $h$ modulo $\End_{-}(V)$ respectively. Then,
	\[
		fg = f_{+}g_{+} + (f_{+}g_{-} + f_{-}g_{+} + f_{-}g_{-}),
	\]
	and $f_{+}g_{-} + f_{-}g_{+} + f_{-}g_{-} \in \End_{-}(V)$. Whence, $(fg)_{+} = f_{+}g_{+}$. Analogously $(gh)_{+} = g_{+}h_{+}$ and $(fh)_{+} = f_{+}h_{+}$. This fact and the identify
	\[
		[f_{+},g_{+}h_{+}] - [f_{+}g_{+}, h_{+}] - [f_{+}h_{+}, g_{+}] = 0
	\]
	imply the desired conclusion.
\end{proof}
\subsection{Properties of residue}
We prove some of the main properties of $\res$.
\begin{proposition}\label{prop:linearity-residue}
	For all $f,g \in K$ it follows that
	\begin{enumerate}[label = (\alph*)]
		\item $\res_{V}(fdg) + \res_{V}(gdf) = 0$, and
		\item $\res_V(df) = 0$.
	\end{enumerate}
\end{proposition}
\begin{proof}
	Since $[f_{+}, g_{+}] + [g_{+}, f_{+}] = 0$ we get (a). For (b) use (a) with $g = 1$. 
\end{proof}
\begin{proposition}\label{prop:linearity-residue-closed-submodule}
	Let $W$ be a closed $K$-submodule of $V$. Then, for $\omega \in \Omega_{K/k}$ the identity
	\[
		\res_{V}(\omega) = \res_{W}(\omega) + \res_{V/W}(\omega)
	\]
	holds.
\end{proposition}
\begin{proof}
	It is enough to prove the claim for $\omega = f dg$. By \cref{lemm:properties-trace} item (c) we only need to check that for all $f \in K$ the induced map $\overline{f}\colon V/W \to V/W$ and $f \circ \iota$, where $\iota$ denotes the inclusion $W \to V$, satisfy
	\begin{align*}
	\overline{f} &= \overline{f_{+}} \mod \End_{-}(V/W), \\
	f \circ \iota &= f_{+} \circ \iota \mod \End_{-}(W), \\ 
	\overline{f_{+}}&\in \End_{+}(V/W), \quad\text{and} \\
	f_{+} \circ \iota &\in \End_{+}(W),
	\end{align*}
	These statements are straightforward to prove and we leave them as an exercise to the reader.
\end{proof}
\begin{proposition}\label{prop:direct-sum-residue}
	If $V$ is the direct sum of two closed submodules $W_{1}$ and $W_{2}$ then 
	\[
		\res_{V}(\omega) = \res_{W_{1}}(\omega) + \res_{W_{2}}(\omega)
	\]
	holds for all $\omega \in \Omega_{K/k}$.
\end{proposition}
\begin{proof}
	Immediate from \cref{prop:linearity-residue-closed-submodule}.
\end{proof}
If our Tate space is trivial so its residue.
\begin{proposition}\label{prop:residue-trivial-tate-space}
	If $V$ is either linearly compact or discrete then $\res_{V}(\Omega_{K/k}) = 0$.
\end{proposition}
\begin{proof}
	If $V$ is linearity compact then $\End_{+}(V) = E$ and $f_{+} = f$ for all $f \in K$. Since $[f,g] = 0$ then 
	\begin{align}\label{eqn:trivial-residue}
		\res_{V}(fdg) = 0.
	\end{align}
	On the other hand, if $V$ is discrete then $E = \End_{-}(V)$. Thus, $f = 0 \mod \End_{-}(V)$ for all $f \in K$. Thus, (\ref{eqn:trivial-residue}) holds.
\end{proof}
\begin{proposition}\label{prop:residue-and-continuity}
	Let $f$ and $g$ belong to $K$. Then, if there exists a c-lattice $L$ in $V$ so that $fL + fgL + fg^{2}L \subseteq L$ it holds $\res_{V}(fdg) = 0$. In particular, when there exists $L$ a c-lattice so that $fL \subseteq L$ and $gL \subseteq L$ then $\res_{V}(fdg) = 0$.
\end{proposition}
\begin{proof}
	Let $\pi$ be a continuous projection from $V$ to $L$. Then $\pi f \in \End_{+}(V)$ and $\pi f = f \mod \End_{-}(V)$. Thus, it follows that
	\[
		\res_{V}(fdg) = \tr([\pi f, g]) 
	\] 
	by \cref{lemm:trace-only-depends-on-K}. Let $h = [\pi f,g]$ and $W = L + gL$. Let $h_{V/W}$ and $h_{W}$ be the induced maps on $V/W$ and $W$ respectively. Then, the relation $fL + fgL + fg^{2}L \subseteq L$ implies that $h_{V/W} = 0$ and $h_{W} = 0$. By \cref{lemm:properties-trace} item (c) we conclude
	\[
		\res_{V}(fdg) = \tr_{V}(h) = \tr_{W}(h) + \tr_{V/W}(h) = 0.
	\]
\end{proof}
In the following two propositions we examine the residue of a power.
\begin{proposition}\label{prop:residue-of-a-power}
	Let $f \in K$, then $\res_{V}(f^{n}df) = 0$ for all $n \geq 0$. Moreover, if $f$ is invertible the same holds for $n \leq -2$.
\end{proposition}
\begin{proof}
	First, if $f_{+} = f \mod \End_{-}(V)$ then $f_{+}^{n} = f^{n} \mod \End_{-}(V)$. Therefore,
	\[
		\res_{V}(f^{n} df) = \tr([f_{+},f_{+}^{n}]) = 0.
	\]
	Second, if $f$ is invertible then
	\[
		fd(f^{-1}) + f^{-1}df = d(ff^{-1}) = d(1) = 0.
	\]
	which implies 
	\[
	f^{-2}df = -d(f^{-1}),
	\]
	and multiplying by $f^{-n}$ both sides, where $n\geq 0$, implies
	\[
	f^{-2-n}df = -(f^{-1})^{n}d(f^{-1}).
	\]
	By the preceding statement, $(f^{-1})^{n}d(f^{-1})$ has zero residue. 
\end{proof}
\begin{proposition}\label{prop:residue-of-invertible-element}
	If $f$ is invertible, so that $fL \subseteq L$ for some c-lattice $L$, then
	\[
		\res_{V}(f^{-1}df) = \dim_{k}(L/fL).
	\]
\end{proposition}
\begin{proof}
	If $\pi$ is a continuous projection of $V$ into $L$ then 
	\[
	\res_{V}(f^{-1}df) = \tr([\pi f^{-1}, f]). 
	\]
	Let $g = [\pi f^{-1}, f]$. Since $fL\subseteq L$ we obtain
	\[
		g_{V/L} = 0, \quad g_{L/fL}= 1\quad\text{and,}\quad g_{fL} = 0, 
	\]
	where $g_{V/L}, g_{L/fL}$ and $g_{fL}$ denote the induced maps in $V/L$, $L/fL$ and, $fL$ respectively. Then, by \cref{lemm:properties-trace} item (c) it follows that
	\[
		\tr_{V}(g) = \tr_{L}(g) + \tr_{V/L}(g) = \tr_{fL}(g) + \tr_{L/fL}(g) + \tr_{V/L}(g).
	\]
	Observe that $\dim L/fL < \infty$ since $fL$ is open and $L$ is linearly compact.
\end{proof}
\subsection{Relationship of residues under extensions}
Finally, we explore the case where $K'$ is a commutative $k$-algebra containing $K$. We will examine $\Omega_{K'/k}$ and $\Omega_{K/k}$ and the relationship between their residues. In this case the injection $K \to K'$ induces a map between $\Omega_{K/k} \to \Omega_{K'/k}$ which may not be injective. 
\begin{proposition}\label{prop:residue-commutes-extension}
	Let $V$ be a Tate space such that multiplication by any $f\in K'$ induces a continuous endomorphism in $\End_{k}(V)$. Therefore, for all $g\in K$ multiplication by $g$ is continuous as well. Hence, we can define
\[
	\res_{V}\colon\Omega_{K/k} \to k, \quad\text{and}\quad \res_{V}'\colon \Omega_{K'/k}\to k.
\]
In this situation, the diagram
\[
\begin{tikzcd}
	\Omega_{K/k} \arrow[r] \arrow[dr, swap, "\res_{V}"] & \Omega_{K'/k} \arrow[d, "\res_{V}'"] \\
	& k  
\end{tikzcd}
\]
commutes.
\end{proposition}
\begin{proof}
	For $f,g \in K$ their residue symbol is independent whether $f\,dg$ is thought as an element in $\Omega_{K'/k}$ or $\Omega_{K/k}$. This observation implies the commutativity of the diagram.
\end{proof}
Now, assume that $K'$ is free $K$-module of finite rank $n$ and consider the tensor product $V' = K' \otimes_{K} V$. Since the tensor product and direct sum commute, it follows that $V' \cong K^{n} \otimes_{K} V \cong (K \otimes_{K} V)^{n}\cong V^{n}$. In coordinates, if $(x_{i})$ is a $K$-base for $K'$ then the map $(v_{1}, \ldots, v_{n}) \mapsto x_{1}\otimes v_{1} + \ldots + x_{n}\otimes v_{n}$ is an isomorphism. With the topology induced by this isomorphism $V'$ is a Tate space.
\begin{proposition}\label{prop:identification-matrices-entries-in-endomorphisms}
	The space $\End(V')$ is isomorphic to the space of $n \times n$ matrices with entries in $\End(V)$ denoted $\Mat_{n}(\End_{0}(V))$. Moreover, if $K$ acts continuously on $V$ so does $K'$ on $V'$.
\end{proposition}
\begin{proof}
	Let $\varphi$ be a continuous $k$-endomorphism of $V'$, then there exists a unique set $\{\varphi_{ij}\}_{i,j=1}^{n}$ contained in $\End(V)$ such that
	\[
	\varphi\left(\sum_{i} x_{i} \otimes v_{i}\right) = \sum_{i,j} x_{i} \otimes \varphi_{ij}(v_{j})
	\]
	for all $v_{1}, \ldots, v_{n} \in V$. Now, let $f' \in K'$, then
	\[
		f'x_{i} = \sum f_{ij}x_{j}
	\]
	where $f_{ij}\in K$. Since $f_{ij}\in\End(V)$ it follows that $f' \in \End(V')$ by the description of our topology in $V'$.
\end{proof}
Let $\End'_{0}(V')$ be the inverse image of $\Mat_{n}(\End_{0}(V))$ under the isomorphism in \cref{prop:identification-matrices-entries-in-endomorphisms}. Note that $\End'_{0}(V')\subseteq \End_{0}(V')$. Therefore, the map
\[
	\tr_{V'}\colon \End'_{0}(V') \to k
\]
is well-defined. 
\begin{proposition}\label{prop:trace-is-trace}
	For $\varphi \in \End'_{0}(V')$ the identity
	\[
		\tr_{V'}(\varphi) = \sum_{i}\tr_{V}(\varphi_{ii})
	\]
	holds.
\end{proposition}
\begin{proof}
	Write $(\varphi_{ij})$ as a sum of a strictly lower triangular,  strictly upper triangular and diagonal matrix. Namely,
	\[
		\varphi = \varphi_{LT} + \varphi_{UT} + \varphi_{D}, 
	\]
	where $\varphi_{LT}$, $\varphi_{UT}$ and $\varphi_{D}$  have a matrix representation of a strictly lower, strictly upper and diagonal matrix respectively. Observe that $\varphi_{LT}$, $\varphi_{UT}$, $\varphi_{D}$ belong to $\End'_{0}(V')$ and $\varphi_{LT}$ and $\varphi_{UT}$ are nilpotent. By \cref{lemm:properties-trace} it follows that
	\[
		\tr_{V'}(\varphi) = \tr_{V'}(\varphi_{D}).
	\]
	On the other hand, by definition
	\[
		\tr_{V'}(\varphi_{D}) = \sum \tr_{V}(\varphi_{ii}).
	\]
\end{proof}


\begin{theorem}\label{thm:residue-of-finite-rank}
	For all $f' \in K'$ and $g \in K$ the equality
	\[
		\res_{V}'(f' dg) = \res_{V}((\tr_{K'/K}(f')dg))
	\]
	holds.
\end{theorem}
\begin{proof}
	Let $L$ be a c-lattice in $V$ then $L' = x_{1} \otimes L + \ldots + x_{n} \otimes L$ is a c-lattice in $V'$. Let $\pi\colon V\to L$ be a linear continuous projection and $\pi'$ be the corresponding element to $(\delta_{ij}\pi)$ under the isomorphism $\End(V') \cong \Mat_{n}(\End(V))$. Therefore, $\pi'\colon V' \to L'$ is a linear continuous projection. On the other hand, let $f'\in K'$ and $g\in K$. Then, $f'$ corresponds to $(f_{ij})\in \Mat_{n}(K)$ and let $g'$ be the corresponding element to $(\delta_{ij} g)$ in $\End(V')$. Hence, the commutator $[\pi'f', g']$ is mapped to $[\pi f_{ij}, g]$ by the map $\End(V') \to \Mat_{n}(\End(V))$. By \cref{prop:trace-is-trace}, it follows that
	\begin{align*}
		\res_{V'}(f' dg) &= \tr_{V'}([\pi'f', g'])  \\
		&= \sum \tr_{V}([\pi f_{ii}, g]) \\
		&= \sum \res_{V}(f_{ii} dg) \\
		&= \res_{V}\left(\left(\sum f_{ii}\right)dg\right) \\
		&= \res_{V}\left(\tr_{K'/K}(f')dg\right).
	\end{align*}
\end{proof}

