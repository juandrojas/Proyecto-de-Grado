%!TEX root = ../main.tex
\chapter{Trace and Residue}\label{ch:trace-and-residue}
\section{Finitepotent maps}
Let $k$ be a fixed ground field and $V$ a vector space over $k$. In this section we will expand the notion of trace of a linear endomorphism to include certain operators even when $V$ is infinite dimensional.
\begin{definition}\label{finitepotent}
	We will say a linear map $\theta\colon V \to V$ is \textit{finitepotent} if
	\[
		\dim \theta^{n}V < \infty
	\]
	for sufficiently large $n$.
\end{definition}
We characterize finitepotent maps as follows
\begin{proposition}\label{characterization-of-finitepotent-maps}
	A linear map $\theta\colon V \to V$ is finitepotent if and only if there exists a subspace $W \subseteq V$ such that
	\begin{enumerate}[label = (\roman*)]
		\item $\dim \theta W < \infty$,
		\item $\theta W \subseteq W$,
		\item the induced map $\bar{\theta}\colon V/W \to V/W$ is nilpotent.
	\end{enumerate}
\end{proposition}
\begin{proof}
	If $\theta$ is finitepotent choose $W = \theta^{n}V$ for sufficiently large $n$. The first condition follows from definition. 
\end{proof}


