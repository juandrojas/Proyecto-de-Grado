%!TEX root = ../main.tex
\chapter{Tate's Linear Algebra}\label{ch:tate-linear-algebra}
\section{Linear topologies}
Fix a ground field $k$. From now on, a vector space will always mean a $k$-vector space.
\begin{definition}\label{def:linear_topology}
A \textbf{linear topology} on a vector space $V$ is a separated (Hausdorff) topology invariant under translations that admits an open local base around zero of vector subspaces. A vector space equipped with a linear topology will be referred as \textbf{linearly topologized}.
\end{definition}
If we endow $k$ with the discrete topology then $V$ will become a topological vector space. From now on, endow $k$ with the discrete topology. \\
Linear topologies behave nicely under basic topological operations.
\begin{proposition}\label{prop:linear_topologies_properties}
Let $V$ be a linearly topologized vector space. Then
	\begin{enumerate}[label = (\alph*)]
		\item Any vector subspace of $V$ is linearly topologized under its subspace topology.
		\item If $W \subseteq V$ is a closed vector subspace then $V/W$ is linearly topologized under its quotient topology.
		\item If $\{V_{\alpha}\}_{\alpha}$ is a collection of linearly topologized vector spaces its product $\prod_{\alpha} V_{\alpha}$ and its direct sum $\bigoplus_{\alpha} V_{\alpha}$ is linearly topologized under its product topology.
		\item If $W$ is a vector subspace of $V$, then its topological closure $\overline{W}$ also is a vector subspace of $V$.
	\end{enumerate}
\end{proposition}
\begin{proof}
	Since intersection of vector subspaces is a vector subspace, (a) follows intersecting the fundamental system of neighborhoods in $V$ by the vector subspace. For (b), let $\pi\colon V \to V/W$ be the quotient map. Since $\pi$ is open and surjective the image of a local base is a local base; moreover, the image of a vector subspace under $\pi$ is a vector subspace. In addition, since  Finally, for (c) let $\{U_{\alpha, \beta}\}_{\beta}$ be a local base of zero in $V_{\alpha}$ of vector subspaces, the products $U_{\alpha_{1}, \beta_{1}} \times \ldots \times U_{\alpha_{n}, \beta_{n}} \times \prod_{\gamma} V_{\gamma}$, where $\gamma$ ranges over $\alpha \neq \alpha_{1}, \ldots, \alpha_{n}$, for any set $\{(\alpha_{1}, \beta_{1}, \ldots, \alpha_{n}, \beta_{n})\}$ form a fundamental system of neighborhoods around zero in $\prod_{\alpha} V_{\alpha}$ of open vector subspaces. Note that since $\bigoplus_{\alpha} V_{\alpha} \subseteq \prod_{\alpha} V_{\alpha}$ is a vector subspace (c) follows from (a). Finally, for (d), suppose $x,y\in \overline{W}$, then, for every open vector subspace $U$, $(x + U)\cap W \neq \emptyset$ and $(y + U)\cap W \neq \emptyset$, therefore for every $\alpha, \beta \in k$ we have $(\alpha x + U)\cap W \neq \emptyset$ and $(\beta y + U)\cap W \neq \emptyset$. Hence, $(\alpha x + \beta y + U)\cap W\neq \emptyset$ for every open vector subspace $U$ and every pair $\alpha, \beta\in k$. It follows (d).
\end{proof}
\begin{remark}\label{rem:limits-and-colimits-in-lintop-category}
	Using an argument similar to the previous proposition one can check that in the category $\mathsf{LinTop}_k$ of linearly topologized vector spaces limits and colimits indexed by small categories exist.
\end{remark}
Finite dimensional vector spaces are meaningless for linear topologies. 
\begin{proposition}\label{prop:finite_dimensional_linear_topologies}
	A finite dimensional linearly topologized vector space $V$ is discrete.
\end{proposition}
\begin{proof}
	Let $U$ be an open vector subspace and $0 \neq x \in U$, since $V$ is separated and linearly topologized there exists an open vector subspace $U_{x}$ such that $x \not\in U_{x}$ then $\dim U_{x} \cap U < \dim U$, since $V$ is finite dimensional this process can be repeated only a finite amount of times; that is $\{0\}$ is open. It follows that $V$ is discrete.
\end{proof}
\subsection*{Commensurability}
We introduce a partial order in the set of vector subspaces of a vector space $V$.
\begin{definition}\label{def:commensurability}
	For vector subspaces $A$ and $B$ of a vector space $V$ we say that $A \prec B$ if the quotient $A/(A\cap B) \cong (A+B)/B$ is finite dimensional (or equivalently $A \subseteq B + W$ for some finite dimensional $W$). In addition, we say that $A$ and $B$ are \textbf{commensurable} (denoted $A \sim B$) if $A \prec B$ and $B \prec A$.
\end{definition}
Observe that $A \sim B$ if and only if $(A+B)/(A\cap B) \cong A/(A\cap B) \oplus B/(A \cap B)$ is finite dimensional. We will constantly refer to a vector space $V$ being finite dimensional as $V \sim 0$.
\begin{proposition}\label{prop:equivalence-relation}
	Let $V$ be a vector spaces and $A,B$ and $L$ be vector subspaces, then:
	\begin{enumerate}[label = (\alph*)]
		\item If $A \sim B$ and $B \sim L$ then
		\[
			(A+B+L)/(A \cap B \cap L) \sim 0
		\]
		\item If $A \prec B$ and $B \prec L$ then $A \prec L$. Moreover, commensurability is an equivalence relation.
	\end{enumerate}
\end{proposition}
\begin{proof}
	Consider the following exact sequences
	\[
		0 \to (A\cap B)/(A \cap B \cap L) \to B/(B \cap L), 
	\]
	and,
	\[
		0 \to (A\cap B)/(A \cap B \cap L) \to (A+B)/(A \cap B \cap L) \to (A+B)/(A \cap B) \to 0
	\]
	induced by inclusions. The first inclusion plus the fact that $B \sim L$ imply that $(A\cap B)/(A \cap B \cap L)$ is finite dimensional. Now, since $A \sim B$ it follows that $(A+B)/(A \cap B)$ is finite dimensional. Hence, the second exact sequence concludes that $(A+B)/(A \cap B \cap L)$. A symmetrical argument shows that $(B+L)/(A \cap B \cap L) \sim 0$. These prove (a). For (b), the inclusion
	\[
		0 \to (A+L)/(A\cap L) \to (A+B+L)/(A \cap B \cap L)
	\]
	plus (a) implies transitivity. 
\end{proof}
Now, we state and prove some useful properties on the relation $\prec$.
\begin{lemma}\label{lemm:properties-order-well-behaved-under-operations}
\begin{enumerate}[label = (\alph*)]
	\item If $A \subseteq B$ then $A \prec B$.
	\item If $A \prec B$ then $f(A) \prec f(B)$ for any $k$-linear map $f$
	\item It holds that
	\[
		\sum_{i=1}^{m} A_{i} \prec \bigcap_{j=1}^{n} B_{j} \iff A_{i} \prec B_{j}\text{ for all } i \text{ and } j.
	\]
\end{enumerate}
\end{lemma}
\begin{proof}
	First, (a) is immediate from the definition of $\prec$. Second, for (b) the map $f$ factors as
	\[
		A/(A\cap B) \to f(A)/(f(A)\cap f(B)) \to 0
	\]
	Finally, for (c), if $\sum_{i=1}^{m} A_{i} \prec \bigcap_{j=1}^{n} B_{j}$ holds then by (a) above, for all $i$ and $j$ we have
	\[
		A_{i} \prec \sum_{i=1}^{m} A_{i} \prec \bigcap_{j=1}^{n} B_{j} \prec B_{j}
	\]
	On the other hand, if $A_{i} \prec B_{j}$ for all $i$ and $j$ then there exists finite dimensional subspaces $W_{ij}$ such that $A_{i} \subseteq B_{j} + W_{ij}$ for all $i$ and $j$. Therefore,
	\[
		\sum_{i=1}^{m} A_{i} \subseteq \bigcap_{j=1}^{n} B_{j} + \sum_{i=1}^{m} \sum_{j=1}^{n} W_{ij}.
	\]

\end{proof}
Next, we consider another useful lemma.
\begin{lemma}\label{lemm:commensurability-addition-and-intersection}
	Let $A,B,A',B'$ be vector subspaces of a vector space $V$ and suppose that $A \sim A'$ and $B \sim B'$. Then $A + B \sim A' + B'$ and $A \cap B \sim A' \cap B'$.
\end{lemma}
\begin{proof}
	The following exact sequence 
	\begin{multline*}
		0 \to (A + A' + B + B')/(A\cap A')\cap(B\cap B') \to \\ (A + A')/(A \cap A') \oplus (B + B')/(B \cap B') \to \\ (A + A' + B + B')/(A\cap A') + (B\cap B') \to 0
	\end{multline*}
	plus $A \sim A'$ and $B \sim B'$ imply that both spaces
	\begin{multline*}
		(A + A' + B + B')/(A\cap A')\cap(B\cap B') \quad\text{and,}\\
		 (A + A' + B + B')/((A\cap A') + (B\cap B'))
	\end{multline*}
	are finite dimensional. Since, $(A + A' + B + B')/(A+A')\cap(B+B')$ is a quotient of the second space and $((A \cap A') + (B \cap B'))/((A\cap A')\cap(B\cap B'))$ is a subspace of the first space we can conclude $A + B \sim A' + B'$ and $A \cap B \sim A' \cap B'$.
\end{proof}
If we consider the set of equivalence classes $\mathscr{L}(V)$ of $\sim$ on a vector space $V$ then $\prec$ is a partial order on it and by \cref{lemm:commensurability-addition-and-intersection} above $\mathscr{L}(V)$ inherits operations $\cap$ and $+$.


\subsection*{Linear compactness}
\begin{definition}\label{def:linear_compactness}
	Let $V$ be a linearly topologized vector space. A closed subset $L \subseteq V$ is \textbf{linearly compact} (respectively \textbf{linearly cocompact}) if for every open vector subspace $U$ we have $L \prec U$ (respectively $V/(L+U) \sim 0$). 
\end{definition}
Linear compactness behaves just as compactness if one uses the correct words.
\begin{proposition}\label{prop:linear_compactness_properties}
	Let $V$ be a linearly compact vector space, then
	\begin{enumerate}[label = (\alph*)]
		\item If $A \subseteq V$ is a vector subspace such that for every open vector subspace $U$ of $V$ it holds $A \prec U$ then $\overline{A}$ is linearly compact.
		\item If $f\colon V \to W$ is a continuous linear homomorphism then $\overline{f(V)}$ is linearly compact.
		\item If $V$ is discrete then $V \sim 0$.
		\item Every closed vector subspace of $V$ is linearly compact.
		\item (Tychonov) If $\{V_{\alpha}\}_{\alpha}$ is a collection of linearly compact vector spaces then its product $\prod_{\alpha} V_{\alpha}$ and its direct sum $\bigoplus_{\alpha}V_{\alpha}$ are linearly compact.
	\end{enumerate}
\end{proposition}
\begin{proof}
	Let $U$ be any open vector subspace of $V$, then $A + U$ is closed, that is $A + U = \overline{A + U} \supseteq \overline{A} + U \supseteq A + U$, thus, $\overline{A} + U = A + U$. Since, $(A + U)/U \sim 0$ then $(\overline{A} + U)/U \sim 0$. We get (a). 

	For (b), since $f$ is a continuous linear map $V \prec f^{-1}(U)$ for all $U$ open vector subspace of $W$, hence by \cref{lemm:properties-order-well-behaved-under-operations} $f(V) \prec U$ for all open vector subspaces $U$ of $W$. By the previous observation and (a) we get (b). If $V$ is discrete, then $\{0\}$ is an open vector subspace of $E$, thus $V \prec U$, we get (c). 

	For (d), if $A \subseteq V$ is a closed vector subspace, and $V \prec U$ for all open vector subspaces $U$ by \cref{lemm:properties-order-well-behaved-under-operations} we get $A \prec U$. 

	Finally, for (d), it is enough proving for open vector subspaces $U = \prod_{\beta} U_{\beta} \times \prod_{\gamma} V_{\gamma}$ where $\beta$ ranges over a finite set, $\gamma$ ranges over $\alpha \neq \beta$ and $U_{\beta}$ is an open vector subspace of $V_{\beta}$. Then, the quotient
	\[
		\prod_{\alpha}V_{\alpha} / U \cong \prod_{\beta} V_{\beta}/U_{\beta}
	\]
	where $\cong$ is a topological and algebraic isomorphism. Since $V_{\alpha}$ is linearly compact for all $\alpha$ and $\beta$ ranges over a finite set we conclude that $\prod_{\alpha} V_{\alpha} / U$ is finite dimensional; therefore, $\prod_{\alpha} V_{\alpha}$ is linearly compact. The proof is analogous for $\bigoplus_{\alpha} V_{\alpha}$.
\end{proof} 
\subsection*{Completion}
\begin{definition}\label{def:completion}
	If $V$ be a linearly topologized vector space, recall that $V$ is said to be \textbf{complete} if 
	\[
		V \cong \varprojlim_{U \in \operatorname{Op}(V)} V/U
	\]
	where $\text{Op}(V)$ runs through all open vector subspaces of $V$. In particular, this implies that for every base $\mathscr{U}$ of zero made from open vector subspaces of $V$ we have
	\[
		V \cong \varprojlim_{U \in \mathscr{U}} V/U
	\]
\end{definition}

\section{Tate spaces}
\subsection*{Lattices}
\begin{definition}\label{def:c-lattice}
	If $V$ is a linearly topologized vector space we say that a \textbf{c-lattice} is an open linearly compact subspace of $V$, \textit{dually} a discrete linearly cocompact subspace is a \textbf{d-lattice}.
\end{definition}
First, we prove that existence of a c-lattice in a linearly topologized vector space is equivalent to existence of a d-lattice.
\begin{proposition}\label{prop:c-lattice-iff-d-lattice}
	A linearly topologized vector space $V$ has a c-lattice if and only if it has a d-lattice. 
\end{proposition}
\begin{proof}
	Suppose $L$ is a c-lattice in $V$, choose any direct complement $D$ of $L$, that is, $V = L \oplus D$. Since $L$ is open, then $D$ is discrete as $D\cap L = 0$, thus ${0}$ is open in $D$. Moreover, $D$ is closed as it is the fiber of $0$ under the projection $V \to L$. Finally, we check that $D$ is linearly cocompact: let $U$ be any open vector subspace of $V$, the composition $L \hookrightarrow V \twoheadrightarrow V/(D+U)$ induces a surjection 
	\[
		L/(L \cap U) \twoheadrightarrow V/(D+U)
	\]
	thus, since $\dim L / (L \cap U) < \infty$ we conclude $\dim V/(D+U) < \infty$. \\
	Now, suppose $D$ is a d-lattice, again, choose $L$ a direct complement for $D$. Analogous as the proof for $D$ being discrete and closed in the previous paragraph it follows the one for $L$ being open. We just check that $L$ is linearly compact. Let $U$ be any open vector subspace, the composition $V \twoheadrightarrow L \twoheadrightarrow L/(L \cap U)$ induces a surjection
	 \[
	 	E/(D + (L \cap U)) \twoheadrightarrow L/(L \cap U)
	 \]
	 since both $L$ and $U$ are open, also $L\cap U$, thus $\dim V/(D + (L \cap U)) < \infty$. It follows, $\dim L/(L \cap U) < \infty$ and $L$ linearly compact.  
\end{proof}
\begin{remark}\label{up-to-finite-dimension}
	Note that in the proof of \cref{prop:c-lattice-iff-d-lattice} it is not strictly necessary to choose a direct complement, one can choose a direct complement up to finite dimension; that is, $L + D \sim V$ and $L \cap D \sim 0$. 
\end{remark}
We now give a characterization of lattices in terms of $\prec$.
\begin{proposition}\label{prop:maximality-lattices}
	A linearly compact subspace is a c-lattice if and only if it is maximal among the set of linearly compact sets ordered by $\prec$. 
\end{proposition}
\begin{proof}
	\textcolor{red}{this one needs some thinking}
\end{proof}
\begin{proposition}\label{prop:lattices-are-basis}
	If $V$ admits a c-lattice, then the set of c-lattices constitutes a base of zero of mutually commensurable vector subspaces.
\end{proposition}
\begin{proof}
	By the previous proposition all c-lattices must be commensurable. Moreover, if $U$ is any open vector subspace and $L$ is a c-lattice we claim that $L \cap U$ is a c-lattice. Indeed, let $U'$ be any open vector subspace, then $L\cap U \prec L\prec U'$. In addition, since $L$ and $U$ are open, $L \cap U$ is open. Thus $L \cap U \subseteq U$ is a c-lattice, this proves the statement.
\end{proof}
We're now ready to introduce the definition of a Tate space.
\begin{definition}\label{def:tate-vector-space}
	A \textbf{Tate space} $V$ is a complete linearly topologized vector space that admits a c-lattice. By the previous proposition and the observation in \cref{def:completion} we get
	\[
		V \cong \varprojlim_{L \in \mathscr{L}(V)} V/L
	\]
	where $\mathscr{L}(V)$ runs through all c-lattices of $V$.
\end{definition}
\begin{example}\label{ex:tate-spaces}
We give some examples of Tate spaces.
\begin{enumerate}[label = (\alph*)]
	\item Any vector space endowed with the discrete topology is a Tate space.
	\item If $\{V_{\alpha}\}_{\alpha}$ is any pro-system of finite dimensional vector spaces (thus, each one endowed with the discrete topology by \cref{prop:finite_dimensional_linear_topologies}), let $V$ be their inverse limit endowed with the inverse limit topology. We claim that this is a linearly compact space. Indeed, if we realize $V$ as a subspace of the product $\prod_{\alpha} V_{\alpha}$, then basic open vector subspaces are just restriction of finite coordinates. Hence, the quotient of $V$ by any basic open vector subspace is a finite product of $V_{\alpha}$, since all $V_{\alpha}$ are finite dimensional we conclude that $V$ is linearly compact and therefore a Tate space.
	\item Let $V = k\left((t)\right)$ with the topology generated by letting $t^{n}k\left[[t]\right]$ for $n \in \Z$ be a system of neighborhoods of zero. Then, $V = k\left[[t]\right] \oplus t k[t^{-1}]$ where $k\left[[t]\right]$ is the completion of $k[x]$ in the $\left\langle x\right\rangle$-adic topology, hence by the previous item linearly compact and, since it is open a c-lattice. By \cref{prop:c-lattice-iff-d-lattice} $t k[t^{-1}]$ is a d-lattice. Therefore, $V$ is a Tate space that is not linearly compact nor discrete.
	
	\end{enumerate}

\end{example}
\subsection*{Duality}
If $V$ is a Tate space we consider the following topology on the dual space $V^{*}$ (where by dual space we mean topological dual). Open vector subspaces are given by
\[
	L^{\perp} = \{\phi\in E^{*} \colon \phi\lvert_{L} = 0\}
\]
where $L$ is a linearly compact subspace. Equivalently, one can define open vector subspaces in $E^{*}$ to be $D^{*}$ where $D$ a direct complement of a linearly compact vector subspace $L$ in $E$ (in this case $D^{*} \hookrightarrow E^{*}$ using the decomposition $L\oplus D$).  \\
First, we prove that the word \emph{dually} in \cref{def:linear_compactness} actually makes sense. 

\begin{lemma}\label{lemm:duality-d-lattice-c-lattice}
	Duality interchanges discrete and linearly compact spaces. 
\end{lemma}
\begin{proof}
	If $L$ is a linearly compact vector space, then $L^{\perp}$ is open in $L^{*}$, thus $L^{*}$ is discrete. If $D$ is discrete, then $D \cong k^{\oplus \Lambda}$ for some $\Lambda$ and endowing $k^{\oplus \Lambda}$ with the discrete topology the previous isomorphism is a homeomorphism too. Moreover, since $D$ is discrete every linear functional is continuous. Using \cref{rem:limits-and-colimits-in-lintop-category} and the well known identity (where maps are isomorphisms in $\mathsf{LinTop}_{k}$)
	\[
		(k^{\oplus \Lambda})^{*} = \Hom_{k}(k^{\oplus \Lambda}, k) \cong \prod_{\Lambda} \Hom_{k}(k, k) \cong \prod_{\Lambda} k
	\]
	we get the desired result by Tychonov's theorem in \cref{prop:linear_compactness_properties}.
\end{proof}
We're ready to prove the analog of Pontryagin's duality for locally compact groups in our context.
\begin{theorem}\label{thm:self-duality}
	For a Tate space $V$ the canonical map $V \to V^{**}$ is an isomorphism.
\end{theorem}
\begin{proof}
	It is enough to prove it for complete linearly compact spaces and discrete spaces, as every Tate space can be decomposed into a direct sum of a c-lattice and a d-lattice. First, we do it for discrete spaces. Suppose $D$ is a discrete vector space. Then, the canonical map
	\[
		\operatorname{ev}\colon D \to D^{**}
	\]
	is open and continuous because $D$ and $D^{**}$ are both discrete by \cref{lemm:duality-d-lattice-c-lattice}. Moreover, it is injective, because for every nonzero $v \in D$ there exists a linear continuous functional $\phi\in D^{*}$ such that $\phi(v)\neq 0$. Finally, we prove surjectivity. Let $\psi \in D^{**}$. Since $\ker \psi$ is open it contains a basic open vector subspace $A^{\perp}$ such that $A \subseteq D$ is a linearly compact subspace. Therefore, since $D^{*}$ is linearly compact then $D^{*} \sim A^{\perp}$, that is, the quotient $D^{*}/A^{\perp}$ is finite dimensional. Recall that the inclusion $\iota\colon A \to D$ induces an isomorphism $D^{*}/A^{\perp} \to A^{*}$ which is a homeomorphism since both spaces are discrete. We can factor $\psi$ as
	\[
	\begin{tikzcd}
		D^{*} \arrow[r, "\psi"] \arrow[d] & k \\
		D^{*}/A^{\perp} \arrow[ru, dashed, "\tilde{\psi}"] \arrow[d, "\cong"] & \\
		A^{*} \arrow[ruu, dashed]
	\end{tikzcd}
	\]
\end{proof}
\subsection*{Morphisms}
A \textbf{morphism} of Tate spaces is a continuous linear homomorphism between Tate spaces.
\begin{definition}\label{def:linearly-compact-and-discrete-morphisms}
	A morphism $f\colon A\to B$ of Tate spaces is said to be \textbf{linearly compact} if the closure of $fA$ is linearly compact in $B$. Dually, it is \textbf{discrete} if $\ker f$ is open in $A$. 
\end{definition}
\begin{proposition}\label{prop:duality-discrete-compact-maps}
	A morphism $f\colon A\to B$ of Tate spaces is linearly compact if and only if $f^{*}$ is discrete.
\end{proposition}
\begin{proof}
	If its linearly compact then $A$
\end{proof}



