\chapter{Tate's Linear Algebra}\label{ch:tate-linear-algebra}
\section{Linear topologies}
Fix a ground field $k$. From now on, a vector space will always mean a $k$-vector space.
\begin{definition}\label{linear_topology}
A \textbf{linear topology} on a vector space $E$ is a separated (Hausdorff) topology invariant under translations that admits an open local base around zero of vector subspaces. A vector space equipped with a linear topology will be referred as \textbf{linearly topologized}.
\end{definition}
If we endow $k$ with the discrete topology then $E$ will become a topological vector space. From now on, endow $k$ to have a discrete topology. \\
Linear topologies behave nicely under basic topological operations.
\begin{proposition}\label{linear_topologies_properties}
Let $E$ be a linearly topologized vector space. Then
	\begin{enumerate}[label = (\alph*)]
		\item Any vector subspace of $E$ is linearly topologized under its subspace topology.
		\item If $F \subseteq E$ is a closed vector subspace then $E/F$ is linearly topologized under its quotient topology.
		\item If $\{E_{\alpha}\}_{\alpha}$ is a collection of linearly topologized vector spaces its product $\prod_{\alpha} E_{\alpha}$ and its direct sum $\bigoplus_{\alpha} E_{\alpha}$ is linearly topologized under its product topology.
	\end{enumerate}
\end{proposition}
\begin{proof}
	Since intersection of vector subspaces is a vector subspace, (a) follows intersecting the fundamental system of neighborhoods in $E$ by the vector subspace. For (b), let $\pi\colon E \to E/F$ be the quotient map. Since $\pi$ is open and surjective the image of a local base is a local base; moreover, the image of a vector subspace under $\pi$ is a vector subspace, then (b) follows. Finally, for (c) let $\{U_{\alpha, \beta}\}_{\beta}$ be a local base of zero in $E_{\alpha}$ of vector subspaces, the products $U_{\alpha_{1}, \beta_{1}} \times \ldots \times U_{\alpha_{n}, \beta_{n}} \times \prod_{\gamma} E_{\gamma}$, where $\gamma$ ranges over $\alpha \neq \alpha_{1}, \ldots, \alpha_{n}$, for any set $\{(\alpha_{1}, \beta_{1}, \ldots, \alpha_{n}, \beta_{n})\}$ form a fundamental system of neighborhoods around zero in $\prod_{\alpha} E_{\alpha}$ of open vector subspaces. Note that since $\bigoplus_{\alpha} E_{\alpha} \subseteq \prod_{\alpha} E_{\alpha}$ is a vector subspace (c) follows from (a). 
\end{proof}
Finite dimensional vector spaces are meaningless for linear topologies. 
\begin{proposition}\label{finite_dimensional_linear_topologies}
	A finite dimensional linearly topologized vector space $E$ is discrete.
\end{proposition}
\begin{proof}
	Let $U$ be an open vector subspace and $0 \neq x \in U$, since $E$ is separated and linearly topologized there exists an open vector subspace $U_{x}$ such that $x \not\in U_{x}$ then $\dim U_{x} \cap U < \dim U$, since $E$ is finite dimensional this process can be repeted only a finite amount of times; that is $\{0\}$ is open. It follows that $E$ is discrete.
\end{proof}
\subsection*{Linear compactness}
\begin{definition}\label{linear_compactness}
	Let $E$ be a linearly topologized vector space. A closed subset $C \subseteq E$ is \textbf{linearly compact} (respectively \textbf{linearly cocompact}) if for every open vector subspace $U$ we have $\dim C/(C \cap U) < \infty$ (respectively $\dim E/(C+U) < \infty$). 
\end{definition}
Linear compactness behaves just as compactness if one uses the correct words.
\begin{proposition}\label{linear_compactness_properties}
	Let $E$ be a linearly compact vector space and $F$ a linearly topologized vector space. Then
	\begin{enumerate}[label = (\alph*)]
		\item If $\varphi\colon E \to F$ is a continuous linear homomorphism then $\varphi(E)$ is linearly compact.
		\item If $E$ is discrete then $E$ must be finite dimensional.
		\item Every closed vector subspace of $E$ is linearly compact.
		\item (Tychonov) If $\{E_{\alpha}\}_{\alpha}$ is a collection of linearly compact vector spaces then its product $\prod_{\alpha} E_{\alpha}$ and its direct sum $\bigoplus_{\alpha}E_{\alpha}$ are linearly compact.
	\end{enumerate}
\end{proposition}
\begin{proof}
	Let $U \subseteq F$ be an open vector subspace, then since $\varphi$ is linear a continuous $\varphi^{-1}(U)$ is an open vector subspace of $E$. Consider the surjective induced map 
	\[
		E/\varphi^{-1}(U) \twoheadrightarrow \varphi(E)/\varphi(E)\cap U
	\]
	as $E$ is linearly compact it follows that $\dim \varphi(E)/\varphi(E)\cap U < \infty$. We get (a). If $E$ is discrete, then $\{0\}$ is an open vector subspace of $E$, (b) follows. For (c), let $G \subseteq E$ be a closed vector subspace and take any open vector subspace $U$ of $E$, then the inclusion $G \hookrightarrow E$ induces
	\[
		G/G \cap U \hookrightarrow E/U
	\]
	where the latter is finite dimensional ($E$ is linearly compact). Finally, for (d), it is enough proving for open vector subspaces $U = \prod_{\beta} U_{\beta} \times \prod_{\gamma} E_{\gamma}$ where $\beta$ ranges over a finite set, $\gamma$ ranges over $\alpha \neq \beta$ and $U_{\beta}$ is an open vector subspace of $E_{\beta}$. Then, the quotient
	\[
		\prod_{\alpha}E_{\alpha} / U \cong \prod_{\beta} E_{\beta}/U_{\beta}
	\]
	where $\cong$ is a topological and algebraic isomorphism. Since $E_{\alpha}$ is linearly compact for all $\alpha$ and $\beta$ ranges over a finite set we conclude that $\prod_{\alpha} E_{\alpha} / U$ is finite dimensional; therefore, $\prod_{\alpha} E_{\alpha}$ is linearly compact. The proof is analogous for $\bigoplus_{\alpha} E_{\alpha}$.
\end{proof}
\subsection*{Completeness}
If $E$ is linearly topologized it admits a fundamental system of neighborhoods consisting of open vector subspaces 
\[
	E \supseteq U_{0} \supseteq U_{1} \supseteq \ldots \supseteq U_{\alpha} \supseteq \ldots
\]
\begin{definition}\label{compleness-inverse-limit}
	In the previous context, we say that $E$ is \textbf{complete} if 
	\[
		E \cong \hat{E} := \varprojlim_{\alpha} E/U_{\alpha}
	\]
	where $\cong$ is an isomorphism of topological vector spaces.
\end{definition}
%\textcolor{blue}{Include properties of completion.} 
\section{Tate spaces}
\begin{definition}\label{tate-vector-space}
	Let $E$ be a linearly topologized vector space. An open linearly compact subspace of $E$ is called a \textbf{c-lattice} if it is open; dually, a \textbf{d-lattice} is a discrete linearly cocompact subspace of $E$. We say that $E$ is a \textbf{Tate space} or \textbf{Tate vector space} if it contains a c-lattice.
\end{definition}
\begin{proposition}\label{c-lattice-iff-d-lattice}
	A linearly topologized vector space $E$ has a c-lattice if and only if it has a d-lattice. 
\end{proposition}
\begin{proof}
	Suppose $C$ is a c-lattice in $E$, choose any direct complement $D$ of $C$, that is, $E = C \oplus D$. Since $C$ is open, then $D$ is discrete as $D\cap C = 0$, thus ${0}$ is open in $D$. Moreover, $D$ is closed as it is the fiber of $0$ under the projection $E \to C$. Finally, we check that $D$ is linearly cocompact: let $U$ be any open vector subspace of $E$, the composition $C \hookrightarrow E \twoheadrightarrow E/(D+U)$ induces a surjection 
	\[
		C/(C \cap U) \twoheadrightarrow E/(D+U)
	\]
	thus, since $\dim C / (C \cap U) < \infty$ we conclude $\dim E/(D+U) < \infty$. \\
	Now, suppose $D$ is a d-lattice, again, choose $C$ a direct complement for $D$. Analogous as the proof for $D$ being discrete and closed in the previous paragraph it follows the one for $C$ being open and closed. We just check that $C$ is linearly compact. Let $U$ be any open vector subspace, the composition $E \twoheadrightarrow C \twoheadrightarrow C/(C \cap U)$ induces a surjection
	 \[
	 	E/(D + (C \cap U)) \twoheadrightarrow C/(C \cap U)
	 \]
	 since both $C$ and $U$ are open, also $C\cap U$, thus $\dim E/(D + (C \cap U)) < \infty$. It follows, $\dim C/(C \cap U) < \infty$ and $C$ linearly compact.  
\end{proof}
\begin{remark}\label{up-to-finite-dimension}
	Note that in the proof of \cref{c-lattice-iff-d-lattice} it is not strictly necessary to choose a direct complement, one can choose a direct complement up to finite dimension; that is, $C + D = E$ and $\dim C + D < \infty$. We used a direct complement to facilitate the proof.
\end{remark}
\subsection*{Duality}
If $E$ is a Tate space we consider the following topology on the dual space $E^{*}$ (where by dual space we mean topological dual). Open vector subspaces are given by
\[
	N(C) = \{\phi\in E^{*} \colon \phi\lvert_{C} = 0\}
\]
where $C$ is linearly compact subspace. Equivalently, one can define open vector subspaces in $E^{*}$ to be $D^{*}$ where $D$ a direct complement of a linearly compact vector subspace $C$ in $E$ (in this case $D^{*} \hookrightarrow E^{*}$ using the decomposition $C\oplus D$).  \\
First, we prove that the word \emph{dually} in \cref{linear_compactness} actually makes sense. 
\begin{proposition}\label{duality-d-lattice-c-lattice}
	Duality interchanges discrete and linearly compact spaces.
\end{proposition}
\begin{proof}
	
\end{proof}


