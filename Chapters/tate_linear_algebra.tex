%!TEX root = ../main.tex
\chapter{Tate's Linear Algebra}\label{ch:tate-linear-algebra}
\section{Linear topologies}
Fix a ground field $k$. From now on, a vector space will always mean a $k$-vector space.
\begin{definition}\label{linear_topology}
A \textbf{linear topology} on a vector space $V$ is a separated (Hausdorff) topology invariant under translations that admits an open local base around zero of vector subspaces. A vector space equipped with a linear topology will be referred as \textbf{linearly topologized}.
\end{definition}
If we endow $k$ with the discrete topology then $V$ will become a topological vector space. From now on, endow $k$ with the a discrete topology. \\
Linear topologies behave nicely under basic topological operations.
\begin{proposition}\label{linear_topologies_properties}
Let $V$ be a linearly topologized vector space. Then
	\begin{enumerate}[label = (\alph*)]
		\item Any vector subspace of $V$ is linearly topologized under its subspace topology.
		\item If $W \subseteq V$ is a closed vector subspace then $V/W$ is linearly topologized under its quotient topology.
		\item If $\{V_{\alpha}\}_{\alpha}$ is a collection of linearly topologized vector spaces its product $\prod_{\alpha} V_{\alpha}$ and its direct sum $\bigoplus_{\alpha} V_{\alpha}$ is linearly topologized under its product topology.
		\item If $W$ is a vector subspace of $V$, then its topological closure $\overline{W}$ also is a vector subspace of $V$.
	\end{enumerate}
\end{proposition}
\begin{proof}
	Since intersection of vector subspaces is a vector subspace, (a) follows intersecting the fundamental system of neighborhoods in $V$ by the vector subspace. For (b), let $\pi\colon V \to V/W$ be the quotient map. Since $\pi$ is open and surjective the image of a local base is a local base; moreover, the image of a vector subspace under $\pi$ is a vector subspace. In addition, since  Finally, for (c) let $\{U_{\alpha, \beta}\}_{\beta}$ be a local base of zero in $V_{\alpha}$ of vector subspaces, the products $U_{\alpha_{1}, \beta_{1}} \times \ldots \times U_{\alpha_{n}, \beta_{n}} \times \prod_{\gamma} V_{\gamma}$, where $\gamma$ ranges over $\alpha \neq \alpha_{1}, \ldots, \alpha_{n}$, for any set $\{(\alpha_{1}, \beta_{1}, \ldots, \alpha_{n}, \beta_{n})\}$ form a fundamental system of neighborhoods around zero in $\prod_{\alpha} V_{\alpha}$ of open vector subspaces. Note that since $\bigoplus_{\alpha} V_{\alpha} \subseteq \prod_{\alpha} V_{\alpha}$ is a vector subspace (c) follows from (a). Finally, for (d), suppose $x,y\in \overline{W}$, then, for every open vector subspace $U$, $(x + U)\cap W \neq \emptyset$ and $(y + U)\cap W \neq \emptyset$, therefore for every $\alpha, \beta \in k$ we have $(\alpha x + U)\cap W \neq \emptyset$ and $(\beta y + U)\cap W \neq \emptyset$. Hence, $(\alpha x + \beta y + U)\cap W\neq \emptyset$ for every open vector subspace $U$ and every pair $\alpha, \beta\in k$. It follows (d).
\end{proof}
Finite dimensional vector spaces are meaningless for linear topologies. 
\begin{proposition}\label{finite_dimensional_linear_topologies}
	A finite dimensional linearly topologized vector space $V$ is discrete.
\end{proposition}
\begin{proof}
	Let $U$ be an open vector subspace and $0 \neq x \in U$, since $V$ is separated and linearly topologized there exists an open vector subspace $U_{x}$ such that $x \not\in U_{x}$ then $\dim U_{x} \cap U < \dim U$, since $V$ is finite dimensional this process can be repeated only a finite amount of times; that is $\{0\}$ is open. It follows that $V$ is discrete.
\end{proof}
\subsection*{Commensurability}
We introduce a partial order in the set of vector subspaces of a vector space $V$.
\begin{definition}\label{def:commensurability}
	For vector subspaces $A$ and $B$ of a vector space $V$ we say that $A \prec B$ if the quotient $A/(A\cap B) \cong (A+B)/B$ is finite dimensional (or equivalently $A \subseteq B + W$ where $W$ is some finite dimensional subspace). In addition, we say that $A$ and $B$ are commensurable (denoted $A \sim B$) if $A \prec B$ and $B \prec A$.
\end{definition}
Observe that $A \sim B$ if and only if $(A+B)/(A\cap B) \cong A/(A\cap B) \oplus B/(A \cap B)$ is finite dimensional. We will constantly refer to a vector space $V$ being finite dimensional as $V \sim 0$.
\begin{proposition}\label{prop:equivalence-relation}
	Let $V$ be a vector spaces and $A,B$ and $C$ be vector subspaces, then:
	\begin{enumerate}[label = (\alph*)]
		\item If $A \sim B$ and $B \sim C$ then
		\[
			(A+B+C)/(A \cap B \cap C) \sim 0
		\]
		\item If $A \prec B$ and $B \prec C$ then $A \prec C$. Moreover, commensurability is an equivalence relation.
	\end{enumerate}
\end{proposition}
\begin{proof}
	Consider the following exact sequences
	\[
		0 \to (A\cap B)/(A \cap B \cap C) \to B/(B \cap C), 
	\]
	and,
	\[
		0 \to (A\cap B)/(A \cap B \cap C) \to (A+B)/(A \cap B \cap C) \to (A+B)/(A \cap B) \to 0
	\]
	induced by inclusions. The first inclusion plus the fact that $B \sim C$ imply that $(A\cap B)/(A \cap B \cap C)$ is finite dimensional. Now, since $A \sim B$ it follows that $(A+B)/(A \cap B)$ is finite dimensional. Hence, the second exact sequence concludes that $(A+B)/(A \cap B \cap C)$. A symmetrical argument shows that $(B+C)/(A \cap B \cap C) \sim 0$. These prove (a). For (b), the inclusion
	\[
		0 \to (A+C)/(A\cap C) \to (A+B+C)/(A \cap B \cap C)
	\]
	plus (a) implies transitivity. 
\end{proof}
Now, we state and prove some useful properties on the relation $\prec$.
\begin{lemma}\label{lemm:properties-order-well-behaved-under-operations}
\begin{enumerate}[label = (\alph*)]
	\item If $A \subseteq B$ then $A \prec B$.
	\item If $A \prec B$ then $f(A) \prec f(B)$ for any $k$-linear map $f$
	\item It holds that
	\[
		\sum_{i=1}^{m} A_{i} \prec \bigcap_{j=1}^{n} B_{j} \iff A_{i} \prec B_{j}\text{ for all } i \text{ and } j.
	\]
\end{enumerate}
\end{lemma}
\begin{proof}
	First, (a) is immediate from the definition of $\prec$. Second, for (b) the map $f$ factors
	\[
		A/(A\cap B) \to f(A)/(f(A)\cap f(B)) \to 0
	\]
	Finally, for (c), if $\sum_{i=1}^{m} A_{i} \prec \bigcap_{j=1}^{n} B_{j}$ holds then by (a) above, for all $i$ and $j$ we have
	\[
		A_{i} \prec \sum_{i=1}^{m} A_{i} \prec \bigcap_{j=1}^{n} B_{j} \prec B_{j}
	\]
	On the other hand, if $A_{i} \prec B_{j}$ for all $i$ and $j$ then there exists finite dimensional subspaces $W_{ij}$ such that $A_{i} \subseteq B_{j} + W_{ij}$ for all $i$ and $j$. Therefore,
	\[
		\sum_{i=1}^{m} A_{i} \subseteq \bigcap_{j=1}^{n} B_{j} + \sum_{i=1}^{m} \sum_{j=1}^{n} W_{ij}.
	\]

\end{proof}
Next, we consider another useful lemma.
\begin{lemma}\label{lemm:commensurability-addition-and-intersection}
	Let $A,B,A',B'$ be vector subspaces of a vector space $V$ and suppose that $A \sim A'$ and $B \sim B'$. Then $A + B \sim A' + B'$ and $A \cap B \sim A' \cap B'$.
\end{lemma}
\begin{proof}
	The following exact sequence 
	\begin{multline*}
		0 \to (A + A' + B + B')/(A\cap A')\cap(B\cap B') \to \\ (A + A')/(A \cap A') \oplus (B + B')/(B \cap B') \to \\ (A + A' + B + B')/(A\cap A') + (B\cap B') \to 0
	\end{multline*}
	plus $A \sim A'$ and $B \sim B'$ imply that both spaces
	\begin{multline*}
		(A + A' + B + B')/(A\cap A')\cap(B\cap B') \quad\text{and,}\\
		 (A + A' + B + B')/((A\cap A') + (B\cap B'))
	\end{multline*}
	are finite dimensional. Since, $(A + A' + B + B')/(A+A')\cap(B+B')$ is a quotient of the second space and $((A \cap A') + (B \cap B'))/((A\cap A')\cap(B\cap B'))$ is a subspace of the first space we can conclude $A + B \sim A' + B'$ and $A \cap B \sim A' \cap B'$.
\end{proof}
If we consider the set of equivalence classes $\mathscr{L}(V)$ of $\sim$ on a vector space $V$ then $\prec$ is a partial order on it and by \cref{lemm:commensurability-addition-and-intersection} above $\mathscr{L}(V)$ inherits operations $\cap$ and $+$.


\subsection*{Linear compactness}
\begin{definition}\label{linear_compactness}
	Let $V$ be a linearly topologized vector space. A closed subset $C \subseteq V$ is \textbf{linearly compact} (respectively \textbf{linearly cocompact}) if for every open vector subspace $U$ we have $\dim C/(C \cap U) < \infty$ (respectively $\dim V/(C+U) < \infty$). 
\end{definition}
Linear compactness behaves just as compactness if one uses the correct words.
\begin{proposition}\label{linear_compactness_properties}
	Let $V$ be a linearly compact vector space, then
	\begin{enumerate}[label = (\alph*)]
		\item If $A \subseteq V$ is a vector subspace such that for every open $U$ vector subspace of $V$ we have $\dim W/(W \cap U) < \infty$ then $\overline{A}$ is linearly compact.
		\item If $\varphi\colon V \to W$ is a continuous linear homomorphism then $\overline{\varphi(V)}$ is linearly compact.
		\item If $E$ is discrete then $E$ must be finite dimensional.
		\item Every closed vector subspace of $E$ is linearly compact.
		\item (Tychonov) If $\{E_{\alpha}\}_{\alpha}$ is a collection of linearly compact vector spaces then its product $\prod_{\alpha} E_{\alpha}$ and its direct sum $\bigoplus_{\alpha}E_{\alpha}$ are linearly compact.
	\end{enumerate}
\end{proposition}
\begin{proof}
	Let $U \subseteq F$ be an open vector subspace, then since $\varphi$ is linear a continuous $\varphi^{-1}(U)$ is an open vector subspace of $E$. Consider the surjective induced map 
	\[
		E/\varphi^{-1}(U) \twoheadrightarrow \varphi(E)/\varphi(E)\cap U
	\]
	as $E$ is linearly compact it follows that $\dim \varphi(E)/\varphi(E)\cap U < \infty$. We get (a). If $E$ is discrete, then $\{0\}$ is an open vector subspace of $E$, (b) follows. For (c), let $G \subseteq E$ be a closed vector subspace and take any open vector subspace $U$ of $E$, then the inclusion $G \hookrightarrow E$ induces
	\[
		G/G \cap U \hookrightarrow E/U
	\]
	where the latter is finite dimensional ($E$ is linearly compact). Finally, for (d), it is enough proving for open vector subspaces $U = \prod_{\beta} U_{\beta} \times \prod_{\gamma} E_{\gamma}$ where $\beta$ ranges over a finite set, $\gamma$ ranges over $\alpha \neq \beta$ and $U_{\beta}$ is an open vector subspace of $E_{\beta}$. Then, the quotient
	\[
		\prod_{\alpha}E_{\alpha} / U \cong \prod_{\beta} E_{\beta}/U_{\beta}
	\]
	where $\cong$ is a topological and algebraic isomorphism. Since $E_{\alpha}$ is linearly compact for all $\alpha$ and $\beta$ ranges over a finite set we conclude that $\prod_{\alpha} E_{\alpha} / U$ is finite dimensional; therefore, $\prod_{\alpha} E_{\alpha}$ is linearly compact. The proof is analogous for $\bigoplus_{\alpha} E_{\alpha}$.
\end{proof}
\subsection*{Completeness}
If $E$ is linearly topologized it admits a fundamental system of neighborhoods consisting of open vector subspaces 
\[
	E \supseteq U_{0} \supseteq U_{1} \supseteq \ldots \supseteq U_{\alpha} \supseteq \ldots
\]
\begin{definition}\label{compleness-inverse-limit}
	In the previous context, we say that $E$ is \textbf{complete} if 
	\[
		E \cong \hat{E} := \varprojlim_{\alpha} E/U_{\alpha}
	\]
	where $\cong$ is an isomorphism of topological vector spaces.
\end{definition}
%\textcolor{blue}{Include properties of completion.} 
\section{Tate spaces}
\begin{definition}\label{tate-vector-space}
	Let $E$ be a linearly topologized vector space. An open linearly compact subspace of $E$ is called a \textbf{c-lattice} if it is open; dually, a \textbf{d-lattice} is a discrete linearly cocompact subspace of $E$. We say that $E$ is a \textbf{Tate space} or \textbf{Tate vector space} if it contains a c-lattice.
\end{definition}
\begin{proposition}\label{c-lattice-iff-d-lattice}
	A linearly topologized vector space $E$ has a c-lattice if and only if it has a d-lattice. 
\end{proposition}
\begin{proof}
	Suppose $C$ is a c-lattice in $E$, choose any direct complement $D$ of $C$, that is, $E = C \oplus D$. Since $C$ is open, then $D$ is discrete as $D\cap C = 0$, thus ${0}$ is open in $D$. Moreover, $D$ is closed as it is the fiber of $0$ under the projection $E \to C$. Finally, we check that $D$ is linearly cocompact: let $U$ be any open vector subspace of $E$, the composition $C \hookrightarrow E \twoheadrightarrow E/(D+U)$ induces a surjection 
	\[
		C/(C \cap U) \twoheadrightarrow E/(D+U)
	\]
	thus, since $\dim C / (C \cap U) < \infty$ we conclude $\dim E/(D+U) < \infty$. \\
	Now, suppose $D$ is a d-lattice, again, choose $C$ a direct complement for $D$. Analogous as the proof for $D$ being discrete and closed in the previous paragraph it follows the one for $C$ being open and closed. We just check that $C$ is linearly compact. Let $U$ be any open vector subspace, the composition $E \twoheadrightarrow C \twoheadrightarrow C/(C \cap U)$ induces a surjection
	 \[
	 	E/(D + (C \cap U)) \twoheadrightarrow C/(C \cap U)
	 \]
	 since both $C$ and $U$ are open, also $C\cap U$, thus $\dim E/(D + (C \cap U)) < \infty$. It follows, $\dim C/(C \cap U) < \infty$ and $C$ linearly compact.  
\end{proof}
\begin{remark}\label{up-to-finite-dimension}
	Note that in the proof of \cref{c-lattice-iff-d-lattice} it is not strictly necessary to choose a direct complement, one can choose a direct complement up to finite dimension; that is, $C + D = E$ and $\dim C + D < \infty$. We used a direct complement to facilitate the proof.
\end{remark}
\subsection*{Duality}
If $E$ is a Tate space we consider the following topology on the dual space $E^{*}$ (where by dual space we mean topological dual). Open vector subspaces are given by
\[
	N(C) = \{\phi\in E^{*} \colon \phi\lvert_{C} = 0\}
\]
where $C$ is linearly compact subspace. Equivalently, one can define open vector subspaces in $E^{*}$ to be $D^{*}$ where $D$ a direct complement of a linearly compact vector subspace $C$ in $E$ (in this case $D^{*} \hookrightarrow E^{*}$ using the decomposition $C\oplus D$).  \\
First, we prove that the word \emph{dually} in \cref{linear_compactness} actually makes sense. 
\begin{proposition}\label{duality-d-lattice-c-lattice}
	Duality interchanges discrete and linearly compact spaces.
\end{proposition}
\begin{proof}
	
\end{proof}
\begin{theorem}\label{thm:poincare-duality}
	For a Tate space $A$ the canonical map $A \to A^{**}$ is an isomorphism.
\end{theorem}

\textcolor{red}{finish this}

\subsection*{Morphisms}
A \textbf{morphism} of Tate spaces is a continuous linear homomorphism between Tate spaces.
\begin{definition}\label{def:linearly-compact-and-discrete-morphisms}
	A morphism $f\colon A\to B$ of Tate spaces is said to be \textbf{linearly compact} if the closure of $fA$ is linearly compact in $B$. Dually, it is \textbf{discrete} if $\ker f$ is open in $A$.
\end{definition}
\begin{proposition}\label{prop:duality-discrete-compact-maps}
	A morphism $f\colon A\to B$ of Tate spaces is linearly compact if and only if $f^{*}$ is discrete.
\end{proposition}
\begin{proof}
	If its linearly compact then 
\end{proof}



