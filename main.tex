%!TEX root = main.tex
\documentclass[oneside, 12pt]{book} %oneside evens margins, twoside for printing
\usepackage{classicthesis}
% ****************************************************************************************************
% classicthesis-config.tex
% formerly known as loadpackages.sty, classicthesis-ldpkg.sty, and classicthesis-preamble.sty
% Use it at the beginning of your ClassicThesis.tex, or as a LaTeX Preamble
% in your ClassicThesis.{tex,lyx} with % ****************************************************************************************************
% classicthesis-config.tex
% formerly known as loadpackages.sty, classicthesis-ldpkg.sty, and classicthesis-preamble.sty
% Use it at the beginning of your ClassicThesis.tex, or as a LaTeX Preamble
% in your ClassicThesis.{tex,lyx} with % ****************************************************************************************************
% classicthesis-config.tex
% formerly known as loadpackages.sty, classicthesis-ldpkg.sty, and classicthesis-preamble.sty
% Use it at the beginning of your ClassicThesis.tex, or as a LaTeX Preamble
% in your ClassicThesis.{tex,lyx} with \input{classicthesis-config}
% ****************************************************************************************************
% If you like the classicthesis, then I would appreciate a postcard.
% My address can be found in the file ClassicThesis.pdf. A collection
% of the postcards I received so far is available online at
% http://postcards.miede.de
% ****************************************************************************************************


% ****************************************************************************************************
% 0. Set the encoding of your files. UTF-8 is the only sensible encoding nowadays. If you can't read
% äöüßáéçèê∂åëæƒÏ€ then change the encoding setting in your editor, not the line below. If your editor
% does not support utf8 use another editor!
% ****************************************************************************************************
\PassOptionsToPackage{utf8}{inputenc}
  \usepackage{inputenc}

\PassOptionsToPackage{T1}{fontenc} % T2A for cyrillics
  \usepackage{fontenc}


% ****************************************************************************************************
% 1. Configure classicthesis for your needs here, e.g., remove "drafting" below
% in order to deactivate the time-stamp on the pages
% (see ClassicThesis.pdf for more information):
% ****************************************************************************************************
\PassOptionsToPackage{
  drafting=true,    % print version information on the bottom of the pages
  tocaligned=false, % the left column of the toc will be aligned (no indentation)
  dottedtoc=false,  % page numbers in ToC flushed right
  eulerchapternumbers=true, % use AMS Euler for chapter font (otherwise Palatino)
  linedheaders=false,       % chaper headers will have line above and beneath
  floatperchapter=true,     % numbering per chapter for all floats (i.e., Figure 1.1)
  eulermath=false,  % use awesome Euler fonts for mathematical formulae (only with pdfLaTeX)
  beramono=true,    % toggle a nice monospaced font (w/ bold)
  palatino=true,    % deactivate standard font for loading another one, see the last section at the end of this file for suggestions
  style=classicthesis % classicthesis, arsclassica
}{classicthesis}


% ****************************************************************************************************
% 2. Personal data and user ad-hoc commands (insert your own data here)
% ****************************************************************************************************
\newcommand{\myTitle}{A Classic Thesis Style\xspace}
\newcommand{\mySubtitle}{An Homage to The Elements of Typographic Style\xspace}
\newcommand{\myDegree}{Doktor-Ingenieur (Dr.-Ing.)\xspace}
\newcommand{\myName}{André Miede \& Ivo Pletikosić\xspace}
\newcommand{\myProf}{Put name here\xspace}
\newcommand{\myOtherProf}{Put name here\xspace}
\newcommand{\mySupervisor}{Put name here\xspace}
\newcommand{\myFaculty}{Put data here\xspace}
\newcommand{\myDepartment}{Put data here\xspace}
\newcommand{\myUni}{Put data here\xspace}
\newcommand{\myLocation}{Saarbrücken\xspace}
\newcommand{\myTime}{June 2018\xspace}
%\newcommand{\myVersion}{\classicthesis}

% ********************************************************************
% Setup, finetuning, and useful commands
% ********************************************************************
\providecommand{\mLyX}{L\kern-.1667em\lower.25em\hbox{Y}\kern-.125emX\@}
\newcommand{\ie}{i.\,e.}
\newcommand{\Ie}{I.\,e.}
\newcommand{\eg}{e.\,g.}
\newcommand{\Eg}{E.\,g.}
% ****************************************************************************************************


% ****************************************************************************************************
% 3. Loading some handy packages
% ****************************************************************************************************
% ********************************************************************
% Packages with options that might require adjustments
% ********************************************************************
\PassOptionsToPackage{american}{babel} % change this to your language(s), main language last
% Spanish languages need extra options in order to work with this template
%\PassOptionsToPackage{spanish,es-lcroman}{babel}
    \usepackage{babel}

\usepackage{csquotes}
\PassOptionsToPackage{%
  %backend=biber,bibencoding=utf8, %instead of bibtex
  backend=bibtex8,bibencoding=ascii,%
  language=auto,%
  style=numeric-comp,%
  %style=authoryear-comp, % Author 1999, 2010
  %bibstyle=authoryear,dashed=false, % dashed: substitute rep. author with ---
  sorting=nyt, % name, year, title
  maxbibnames=10, % default: 3, et al.
  %backref=true,%
  natbib=true % natbib compatibility mode (\citep and \citet still work)
}{biblatex}
    \usepackage{biblatex}

\PassOptionsToPackage{fleqn}{amsmath}       % math environments and more by the AMS
  \usepackage{amsmath}

% ********************************************************************
% General useful packages
% ********************************************************************
\usepackage{graphicx} %
\usepackage{scrhack} % fix warnings when using KOMA with listings package
\usepackage{xspace} % to get the spacing after macros right
\PassOptionsToPackage{printonlyused,smaller}{acronym}
  \usepackage{acronym} % nice macros for handling all acronyms in the thesis
  %\renewcommand{\bflabel}[1]{{#1}\hfill} % fix the list of acronyms --> no longer working
  %\renewcommand*{\acsfont}[1]{\textsc{#1}}
  %\renewcommand*{\aclabelfont}[1]{\acsfont{#1}}
  %\def\bflabel#1{{#1\hfill}}
  \def\bflabel#1{{\acsfont{#1}\hfill}}
  \def\aclabelfont#1{\acsfont{#1}}
% ****************************************************************************************************
%\usepackage{pgfplots} % External TikZ/PGF support (thanks to Andreas Nautsch)
%\usetikzlibrary{external}
%\tikzexternalize[mode=list and make, prefix=ext-tikz/]
% ****************************************************************************************************


% ****************************************************************************************************
% 4. Setup floats: tables, (sub)figures, and captions
% ****************************************************************************************************
\usepackage{tabularx} % better tables
  \setlength{\extrarowheight}{3pt} % increase table row height
\newcommand{\tableheadline}[1]{\multicolumn{1}{l}{\spacedlowsmallcaps{#1}}}
\newcommand{\myfloatalign}{\centering} % to be used with each float for alignment
\usepackage{subfig}
% ****************************************************************************************************


% ****************************************************************************************************
% 5. Setup code listings
% ****************************************************************************************************
\usepackage{listings}
%\lstset{emph={trueIndex,root},emphstyle=\color{BlueViolet}}%\underbar} % for special keywords
\lstset{language=[LaTeX]Tex,%C++,
  morekeywords={PassOptionsToPackage,selectlanguage},
  keywordstyle=\color{RoyalBlue},%\bfseries,
  basicstyle=\small\ttfamily,
  %identifierstyle=\color{NavyBlue},
  commentstyle=\color{Green}\ttfamily,
  stringstyle=\rmfamily,
  numbers=none,%left,%
  numberstyle=\scriptsize,%\tiny
  stepnumber=5,
  numbersep=8pt,
  showstringspaces=false,
  breaklines=true,
  %frameround=ftff,
  %frame=single,
  belowcaptionskip=.75\baselineskip
  %frame=L
}
% ****************************************************************************************************




% ****************************************************************************************************
% 6. Last calls before the bar closes
% ****************************************************************************************************
% ********************************************************************
% Her Majesty herself
% ********************************************************************
\usepackage{classicthesis}


% ********************************************************************
% Fine-tune hyperreferences (hyperref should be called last)
% ********************************************************************
\hypersetup{%
  %draft, % hyperref's draft mode, for printing see below
  colorlinks=true, linktocpage=true, pdfstartpage=3, pdfstartview=FitV,%
  % uncomment the following line if you want to have black links (e.g., for printing)
  %colorlinks=false, linktocpage=false, pdfstartpage=3, pdfstartview=FitV, pdfborder={0 0 0},%
  breaklinks=true, pageanchor=true,%
  pdfpagemode=UseNone, %
  % pdfpagemode=UseOutlines,%
  plainpages=false, bookmarksnumbered, bookmarksopen=true, bookmarksopenlevel=1,%
  hypertexnames=true, pdfhighlight=/O,%nesting=true,%frenchlinks,%
  urlcolor=CTurl, linkcolor=CTlink, citecolor=CTcitation, %pagecolor=RoyalBlue,%
  %urlcolor=Black, linkcolor=Black, citecolor=Black, %pagecolor=Black,%
  pdftitle={\myTitle},%
  pdfauthor={\textcopyright\ \myName, \myUni, \myFaculty},%
  pdfsubject={},%
  pdfkeywords={},%
  pdfcreator={pdfLaTeX},%
  pdfproducer={LaTeX with hyperref and classicthesis}%
}


% ********************************************************************
% Setup autoreferences (hyperref and babel)
% ********************************************************************
% There are some issues regarding autorefnames
% http://www.tex.ac.uk/cgi-bin/texfaq2html?label=latexwords
% you have to redefine the macros for the
% language you use, e.g., american, ngerman
% (as chosen when loading babel/AtBeginDocument)
% ********************************************************************
\makeatletter
\@ifpackageloaded{babel}%
  {%
    \addto\extrasamerican{%
      \renewcommand*{\figureautorefname}{Figure}%
      \renewcommand*{\tableautorefname}{Table}%
      \renewcommand*{\partautorefname}{Part}%
      \renewcommand*{\chapterautorefname}{Chapter}%
      \renewcommand*{\sectionautorefname}{Section}%
      \renewcommand*{\subsectionautorefname}{Section}%
      \renewcommand*{\subsubsectionautorefname}{Section}%
    }%
    \addto\extrasngerman{%
      \renewcommand*{\paragraphautorefname}{Absatz}%
      \renewcommand*{\subparagraphautorefname}{Unterabsatz}%
      \renewcommand*{\footnoteautorefname}{Fu\"snote}%
      \renewcommand*{\FancyVerbLineautorefname}{Zeile}%
      \renewcommand*{\theoremautorefname}{Theorem}%
      \renewcommand*{\appendixautorefname}{Anhang}%
      \renewcommand*{\equationautorefname}{Gleichung}%
      \renewcommand*{\itemautorefname}{Punkt}%
    }%
      % Fix to getting autorefs for subfigures right (thanks to Belinda Vogt for changing the definition)
      \providecommand{\subfigureautorefname}{\figureautorefname}%
    }{\relax}
\makeatother


% ********************************************************************
% Development Stuff
% ********************************************************************
\listfiles
%\PassOptionsToPackage{l2tabu,orthodox,abort}{nag}
%  \usepackage{nag}
%\PassOptionsToPackage{warning, all}{onlyamsmath}
%  \usepackage{onlyamsmath}


% ****************************************************************************************************
% 7. Further adjustments (experimental)
% ****************************************************************************************************
% ********************************************************************
% Changing the text area
% ********************************************************************
%\areaset[current]{312pt}{761pt} % 686 (factor 2.2) + 33 head + 42 head \the\footskip
\setlength{\marginparwidth}{4em}%
%\setlength{\marginparsep}{2em}%

% ********************************************************************
% Using different fonts
% ********************************************************************
%\usepackage[oldstylenums]{kpfonts} % oldstyle notextcomp
% \usepackage[osf]{libertine}
%\usepackage[light,condensed,math]{iwona}
%\renewcommand{\sfdefault}{iwona}
%\usepackage{lmodern} % <-- no osf support :-(
%\usepackage{cfr-lm} %
%\usepackage[urw-garamond]{mathdesign} <-- no osf support :-(
%\usepackage[default,osfigures]{opensans} % scale=0.95
%\usepackage[sfdefault]{FiraSans}
% \usepackage[opticals,mathlf]{MinionPro} % onlytext
% ********************************************************************
%\usepackage[largesc,osf]{newpxtext}
%\linespread{1.05} % a bit more for Palatino
% Used to fix these:
% https://bitbucket.org/amiede/classicthesis/issues/139/italics-in-pallatino-capitals-chapter
% https://bitbucket.org/amiede/classicthesis/issues/45/problema-testatine-su-classicthesis-style
% ********************************************************************
% ****************************************************************************************************

% ****************************************************************************************************
% If you like the classicthesis, then I would appreciate a postcard.
% My address can be found in the file ClassicThesis.pdf. A collection
% of the postcards I received so far is available online at
% http://postcards.miede.de
% ****************************************************************************************************


% ****************************************************************************************************
% 0. Set the encoding of your files. UTF-8 is the only sensible encoding nowadays. If you can't read
% äöüßáéçèê∂åëæƒÏ€ then change the encoding setting in your editor, not the line below. If your editor
% does not support utf8 use another editor!
% ****************************************************************************************************
\PassOptionsToPackage{utf8}{inputenc}
  \usepackage{inputenc}

\PassOptionsToPackage{T1}{fontenc} % T2A for cyrillics
  \usepackage{fontenc}


% ****************************************************************************************************
% 1. Configure classicthesis for your needs here, e.g., remove "drafting" below
% in order to deactivate the time-stamp on the pages
% (see ClassicThesis.pdf for more information):
% ****************************************************************************************************
\PassOptionsToPackage{
  drafting=true,    % print version information on the bottom of the pages
  tocaligned=false, % the left column of the toc will be aligned (no indentation)
  dottedtoc=false,  % page numbers in ToC flushed right
  eulerchapternumbers=true, % use AMS Euler for chapter font (otherwise Palatino)
  linedheaders=false,       % chaper headers will have line above and beneath
  floatperchapter=true,     % numbering per chapter for all floats (i.e., Figure 1.1)
  eulermath=false,  % use awesome Euler fonts for mathematical formulae (only with pdfLaTeX)
  beramono=true,    % toggle a nice monospaced font (w/ bold)
  palatino=true,    % deactivate standard font for loading another one, see the last section at the end of this file for suggestions
  style=classicthesis % classicthesis, arsclassica
}{classicthesis}


% ****************************************************************************************************
% 2. Personal data and user ad-hoc commands (insert your own data here)
% ****************************************************************************************************
\newcommand{\myTitle}{A Classic Thesis Style\xspace}
\newcommand{\mySubtitle}{An Homage to The Elements of Typographic Style\xspace}
\newcommand{\myDegree}{Doktor-Ingenieur (Dr.-Ing.)\xspace}
\newcommand{\myName}{André Miede \& Ivo Pletikosić\xspace}
\newcommand{\myProf}{Put name here\xspace}
\newcommand{\myOtherProf}{Put name here\xspace}
\newcommand{\mySupervisor}{Put name here\xspace}
\newcommand{\myFaculty}{Put data here\xspace}
\newcommand{\myDepartment}{Put data here\xspace}
\newcommand{\myUni}{Put data here\xspace}
\newcommand{\myLocation}{Saarbrücken\xspace}
\newcommand{\myTime}{June 2018\xspace}
%\newcommand{\myVersion}{\classicthesis}

% ********************************************************************
% Setup, finetuning, and useful commands
% ********************************************************************
\providecommand{\mLyX}{L\kern-.1667em\lower.25em\hbox{Y}\kern-.125emX\@}
\newcommand{\ie}{i.\,e.}
\newcommand{\Ie}{I.\,e.}
\newcommand{\eg}{e.\,g.}
\newcommand{\Eg}{E.\,g.}
% ****************************************************************************************************


% ****************************************************************************************************
% 3. Loading some handy packages
% ****************************************************************************************************
% ********************************************************************
% Packages with options that might require adjustments
% ********************************************************************
\PassOptionsToPackage{american}{babel} % change this to your language(s), main language last
% Spanish languages need extra options in order to work with this template
%\PassOptionsToPackage{spanish,es-lcroman}{babel}
    \usepackage{babel}

\usepackage{csquotes}
\PassOptionsToPackage{%
  %backend=biber,bibencoding=utf8, %instead of bibtex
  backend=bibtex8,bibencoding=ascii,%
  language=auto,%
  style=numeric-comp,%
  %style=authoryear-comp, % Author 1999, 2010
  %bibstyle=authoryear,dashed=false, % dashed: substitute rep. author with ---
  sorting=nyt, % name, year, title
  maxbibnames=10, % default: 3, et al.
  %backref=true,%
  natbib=true % natbib compatibility mode (\citep and \citet still work)
}{biblatex}
    \usepackage{biblatex}

\PassOptionsToPackage{fleqn}{amsmath}       % math environments and more by the AMS
  \usepackage{amsmath}

% ********************************************************************
% General useful packages
% ********************************************************************
\usepackage{graphicx} %
\usepackage{scrhack} % fix warnings when using KOMA with listings package
\usepackage{xspace} % to get the spacing after macros right
\PassOptionsToPackage{printonlyused,smaller}{acronym}
  \usepackage{acronym} % nice macros for handling all acronyms in the thesis
  %\renewcommand{\bflabel}[1]{{#1}\hfill} % fix the list of acronyms --> no longer working
  %\renewcommand*{\acsfont}[1]{\textsc{#1}}
  %\renewcommand*{\aclabelfont}[1]{\acsfont{#1}}
  %\def\bflabel#1{{#1\hfill}}
  \def\bflabel#1{{\acsfont{#1}\hfill}}
  \def\aclabelfont#1{\acsfont{#1}}
% ****************************************************************************************************
%\usepackage{pgfplots} % External TikZ/PGF support (thanks to Andreas Nautsch)
%\usetikzlibrary{external}
%\tikzexternalize[mode=list and make, prefix=ext-tikz/]
% ****************************************************************************************************


% ****************************************************************************************************
% 4. Setup floats: tables, (sub)figures, and captions
% ****************************************************************************************************
\usepackage{tabularx} % better tables
  \setlength{\extrarowheight}{3pt} % increase table row height
\newcommand{\tableheadline}[1]{\multicolumn{1}{l}{\spacedlowsmallcaps{#1}}}
\newcommand{\myfloatalign}{\centering} % to be used with each float for alignment
\usepackage{subfig}
% ****************************************************************************************************


% ****************************************************************************************************
% 5. Setup code listings
% ****************************************************************************************************
\usepackage{listings}
%\lstset{emph={trueIndex,root},emphstyle=\color{BlueViolet}}%\underbar} % for special keywords
\lstset{language=[LaTeX]Tex,%C++,
  morekeywords={PassOptionsToPackage,selectlanguage},
  keywordstyle=\color{RoyalBlue},%\bfseries,
  basicstyle=\small\ttfamily,
  %identifierstyle=\color{NavyBlue},
  commentstyle=\color{Green}\ttfamily,
  stringstyle=\rmfamily,
  numbers=none,%left,%
  numberstyle=\scriptsize,%\tiny
  stepnumber=5,
  numbersep=8pt,
  showstringspaces=false,
  breaklines=true,
  %frameround=ftff,
  %frame=single,
  belowcaptionskip=.75\baselineskip
  %frame=L
}
% ****************************************************************************************************




% ****************************************************************************************************
% 6. Last calls before the bar closes
% ****************************************************************************************************
% ********************************************************************
% Her Majesty herself
% ********************************************************************
\usepackage{classicthesis}


% ********************************************************************
% Fine-tune hyperreferences (hyperref should be called last)
% ********************************************************************
\hypersetup{%
  %draft, % hyperref's draft mode, for printing see below
  colorlinks=true, linktocpage=true, pdfstartpage=3, pdfstartview=FitV,%
  % uncomment the following line if you want to have black links (e.g., for printing)
  %colorlinks=false, linktocpage=false, pdfstartpage=3, pdfstartview=FitV, pdfborder={0 0 0},%
  breaklinks=true, pageanchor=true,%
  pdfpagemode=UseNone, %
  % pdfpagemode=UseOutlines,%
  plainpages=false, bookmarksnumbered, bookmarksopen=true, bookmarksopenlevel=1,%
  hypertexnames=true, pdfhighlight=/O,%nesting=true,%frenchlinks,%
  urlcolor=CTurl, linkcolor=CTlink, citecolor=CTcitation, %pagecolor=RoyalBlue,%
  %urlcolor=Black, linkcolor=Black, citecolor=Black, %pagecolor=Black,%
  pdftitle={\myTitle},%
  pdfauthor={\textcopyright\ \myName, \myUni, \myFaculty},%
  pdfsubject={},%
  pdfkeywords={},%
  pdfcreator={pdfLaTeX},%
  pdfproducer={LaTeX with hyperref and classicthesis}%
}


% ********************************************************************
% Setup autoreferences (hyperref and babel)
% ********************************************************************
% There are some issues regarding autorefnames
% http://www.tex.ac.uk/cgi-bin/texfaq2html?label=latexwords
% you have to redefine the macros for the
% language you use, e.g., american, ngerman
% (as chosen when loading babel/AtBeginDocument)
% ********************************************************************
\makeatletter
\@ifpackageloaded{babel}%
  {%
    \addto\extrasamerican{%
      \renewcommand*{\figureautorefname}{Figure}%
      \renewcommand*{\tableautorefname}{Table}%
      \renewcommand*{\partautorefname}{Part}%
      \renewcommand*{\chapterautorefname}{Chapter}%
      \renewcommand*{\sectionautorefname}{Section}%
      \renewcommand*{\subsectionautorefname}{Section}%
      \renewcommand*{\subsubsectionautorefname}{Section}%
    }%
    \addto\extrasngerman{%
      \renewcommand*{\paragraphautorefname}{Absatz}%
      \renewcommand*{\subparagraphautorefname}{Unterabsatz}%
      \renewcommand*{\footnoteautorefname}{Fu\"snote}%
      \renewcommand*{\FancyVerbLineautorefname}{Zeile}%
      \renewcommand*{\theoremautorefname}{Theorem}%
      \renewcommand*{\appendixautorefname}{Anhang}%
      \renewcommand*{\equationautorefname}{Gleichung}%
      \renewcommand*{\itemautorefname}{Punkt}%
    }%
      % Fix to getting autorefs for subfigures right (thanks to Belinda Vogt for changing the definition)
      \providecommand{\subfigureautorefname}{\figureautorefname}%
    }{\relax}
\makeatother


% ********************************************************************
% Development Stuff
% ********************************************************************
\listfiles
%\PassOptionsToPackage{l2tabu,orthodox,abort}{nag}
%  \usepackage{nag}
%\PassOptionsToPackage{warning, all}{onlyamsmath}
%  \usepackage{onlyamsmath}


% ****************************************************************************************************
% 7. Further adjustments (experimental)
% ****************************************************************************************************
% ********************************************************************
% Changing the text area
% ********************************************************************
%\areaset[current]{312pt}{761pt} % 686 (factor 2.2) + 33 head + 42 head \the\footskip
\setlength{\marginparwidth}{4em}%
%\setlength{\marginparsep}{2em}%

% ********************************************************************
% Using different fonts
% ********************************************************************
%\usepackage[oldstylenums]{kpfonts} % oldstyle notextcomp
% \usepackage[osf]{libertine}
%\usepackage[light,condensed,math]{iwona}
%\renewcommand{\sfdefault}{iwona}
%\usepackage{lmodern} % <-- no osf support :-(
%\usepackage{cfr-lm} %
%\usepackage[urw-garamond]{mathdesign} <-- no osf support :-(
%\usepackage[default,osfigures]{opensans} % scale=0.95
%\usepackage[sfdefault]{FiraSans}
% \usepackage[opticals,mathlf]{MinionPro} % onlytext
% ********************************************************************
%\usepackage[largesc,osf]{newpxtext}
%\linespread{1.05} % a bit more for Palatino
% Used to fix these:
% https://bitbucket.org/amiede/classicthesis/issues/139/italics-in-pallatino-capitals-chapter
% https://bitbucket.org/amiede/classicthesis/issues/45/problema-testatine-su-classicthesis-style
% ********************************************************************
% ****************************************************************************************************

% ****************************************************************************************************
% If you like the classicthesis, then I would appreciate a postcard.
% My address can be found in the file ClassicThesis.pdf. A collection
% of the postcards I received so far is available online at
% http://postcards.miede.de
% ****************************************************************************************************


% ****************************************************************************************************
% 0. Set the encoding of your files. UTF-8 is the only sensible encoding nowadays. If you can't read
% äöüßáéçèê∂åëæƒÏ€ then change the encoding setting in your editor, not the line below. If your editor
% does not support utf8 use another editor!
% ****************************************************************************************************
\PassOptionsToPackage{utf8}{inputenc}
  \usepackage{inputenc}

\PassOptionsToPackage{T1}{fontenc} % T2A for cyrillics
  \usepackage{fontenc}


% ****************************************************************************************************
% 1. Configure classicthesis for your needs here, e.g., remove "drafting" below
% in order to deactivate the time-stamp on the pages
% (see ClassicThesis.pdf for more information):
% ****************************************************************************************************
\PassOptionsToPackage{
  drafting=true,    % print version information on the bottom of the pages
  tocaligned=false, % the left column of the toc will be aligned (no indentation)
  dottedtoc=false,  % page numbers in ToC flushed right
  eulerchapternumbers=true, % use AMS Euler for chapter font (otherwise Palatino)
  linedheaders=false,       % chaper headers will have line above and beneath
  floatperchapter=true,     % numbering per chapter for all floats (i.e., Figure 1.1)
  eulermath=false,  % use awesome Euler fonts for mathematical formulae (only with pdfLaTeX)
  beramono=true,    % toggle a nice monospaced font (w/ bold)
  palatino=true,    % deactivate standard font for loading another one, see the last section at the end of this file for suggestions
  style=classicthesis % classicthesis, arsclassica
}{classicthesis}


% ****************************************************************************************************
% 2. Personal data and user ad-hoc commands (insert your own data here)
% ****************************************************************************************************
\newcommand{\myTitle}{A Classic Thesis Style\xspace}
\newcommand{\mySubtitle}{An Homage to The Elements of Typographic Style\xspace}
\newcommand{\myDegree}{Doktor-Ingenieur (Dr.-Ing.)\xspace}
\newcommand{\myName}{André Miede \& Ivo Pletikosić\xspace}
\newcommand{\myProf}{Put name here\xspace}
\newcommand{\myOtherProf}{Put name here\xspace}
\newcommand{\mySupervisor}{Put name here\xspace}
\newcommand{\myFaculty}{Put data here\xspace}
\newcommand{\myDepartment}{Put data here\xspace}
\newcommand{\myUni}{Put data here\xspace}
\newcommand{\myLocation}{Saarbrücken\xspace}
\newcommand{\myTime}{June 2018\xspace}
%\newcommand{\myVersion}{\classicthesis}

% ********************************************************************
% Setup, finetuning, and useful commands
% ********************************************************************
\providecommand{\mLyX}{L\kern-.1667em\lower.25em\hbox{Y}\kern-.125emX\@}
\newcommand{\ie}{i.\,e.}
\newcommand{\Ie}{I.\,e.}
\newcommand{\eg}{e.\,g.}
\newcommand{\Eg}{E.\,g.}
% ****************************************************************************************************


% ****************************************************************************************************
% 3. Loading some handy packages
% ****************************************************************************************************
% ********************************************************************
% Packages with options that might require adjustments
% ********************************************************************
\PassOptionsToPackage{american}{babel} % change this to your language(s), main language last
% Spanish languages need extra options in order to work with this template
%\PassOptionsToPackage{spanish,es-lcroman}{babel}
    \usepackage{babel}

\usepackage{csquotes}
\PassOptionsToPackage{%
  %backend=biber,bibencoding=utf8, %instead of bibtex
  backend=bibtex8,bibencoding=ascii,%
  language=auto,%
  style=numeric-comp,%
  %style=authoryear-comp, % Author 1999, 2010
  %bibstyle=authoryear,dashed=false, % dashed: substitute rep. author with ---
  sorting=nyt, % name, year, title
  maxbibnames=10, % default: 3, et al.
  %backref=true,%
  natbib=true % natbib compatibility mode (\citep and \citet still work)
}{biblatex}
    \usepackage{biblatex}

\PassOptionsToPackage{fleqn}{amsmath}       % math environments and more by the AMS
  \usepackage{amsmath}

% ********************************************************************
% General useful packages
% ********************************************************************
\usepackage{graphicx} %
\usepackage{scrhack} % fix warnings when using KOMA with listings package
\usepackage{xspace} % to get the spacing after macros right
\PassOptionsToPackage{printonlyused,smaller}{acronym}
  \usepackage{acronym} % nice macros for handling all acronyms in the thesis
  %\renewcommand{\bflabel}[1]{{#1}\hfill} % fix the list of acronyms --> no longer working
  %\renewcommand*{\acsfont}[1]{\textsc{#1}}
  %\renewcommand*{\aclabelfont}[1]{\acsfont{#1}}
  %\def\bflabel#1{{#1\hfill}}
  \def\bflabel#1{{\acsfont{#1}\hfill}}
  \def\aclabelfont#1{\acsfont{#1}}
% ****************************************************************************************************
%\usepackage{pgfplots} % External TikZ/PGF support (thanks to Andreas Nautsch)
%\usetikzlibrary{external}
%\tikzexternalize[mode=list and make, prefix=ext-tikz/]
% ****************************************************************************************************


% ****************************************************************************************************
% 4. Setup floats: tables, (sub)figures, and captions
% ****************************************************************************************************
\usepackage{tabularx} % better tables
  \setlength{\extrarowheight}{3pt} % increase table row height
\newcommand{\tableheadline}[1]{\multicolumn{1}{l}{\spacedlowsmallcaps{#1}}}
\newcommand{\myfloatalign}{\centering} % to be used with each float for alignment
\usepackage{subfig}
% ****************************************************************************************************


% ****************************************************************************************************
% 5. Setup code listings
% ****************************************************************************************************
\usepackage{listings}
%\lstset{emph={trueIndex,root},emphstyle=\color{BlueViolet}}%\underbar} % for special keywords
\lstset{language=[LaTeX]Tex,%C++,
  morekeywords={PassOptionsToPackage,selectlanguage},
  keywordstyle=\color{RoyalBlue},%\bfseries,
  basicstyle=\small\ttfamily,
  %identifierstyle=\color{NavyBlue},
  commentstyle=\color{Green}\ttfamily,
  stringstyle=\rmfamily,
  numbers=none,%left,%
  numberstyle=\scriptsize,%\tiny
  stepnumber=5,
  numbersep=8pt,
  showstringspaces=false,
  breaklines=true,
  %frameround=ftff,
  %frame=single,
  belowcaptionskip=.75\baselineskip
  %frame=L
}
% ****************************************************************************************************




% ****************************************************************************************************
% 6. Last calls before the bar closes
% ****************************************************************************************************
% ********************************************************************
% Her Majesty herself
% ********************************************************************
\usepackage{classicthesis}


% ********************************************************************
% Fine-tune hyperreferences (hyperref should be called last)
% ********************************************************************
\hypersetup{%
  %draft, % hyperref's draft mode, for printing see below
  colorlinks=true, linktocpage=true, pdfstartpage=3, pdfstartview=FitV,%
  % uncomment the following line if you want to have black links (e.g., for printing)
  %colorlinks=false, linktocpage=false, pdfstartpage=3, pdfstartview=FitV, pdfborder={0 0 0},%
  breaklinks=true, pageanchor=true,%
  pdfpagemode=UseNone, %
  % pdfpagemode=UseOutlines,%
  plainpages=false, bookmarksnumbered, bookmarksopen=true, bookmarksopenlevel=1,%
  hypertexnames=true, pdfhighlight=/O,%nesting=true,%frenchlinks,%
  urlcolor=CTurl, linkcolor=CTlink, citecolor=CTcitation, %pagecolor=RoyalBlue,%
  %urlcolor=Black, linkcolor=Black, citecolor=Black, %pagecolor=Black,%
  pdftitle={\myTitle},%
  pdfauthor={\textcopyright\ \myName, \myUni, \myFaculty},%
  pdfsubject={},%
  pdfkeywords={},%
  pdfcreator={pdfLaTeX},%
  pdfproducer={LaTeX with hyperref and classicthesis}%
}


% ********************************************************************
% Setup autoreferences (hyperref and babel)
% ********************************************************************
% There are some issues regarding autorefnames
% http://www.tex.ac.uk/cgi-bin/texfaq2html?label=latexwords
% you have to redefine the macros for the
% language you use, e.g., american, ngerman
% (as chosen when loading babel/AtBeginDocument)
% ********************************************************************
\makeatletter
\@ifpackageloaded{babel}%
  {%
    \addto\extrasamerican{%
      \renewcommand*{\figureautorefname}{Figure}%
      \renewcommand*{\tableautorefname}{Table}%
      \renewcommand*{\partautorefname}{Part}%
      \renewcommand*{\chapterautorefname}{Chapter}%
      \renewcommand*{\sectionautorefname}{Section}%
      \renewcommand*{\subsectionautorefname}{Section}%
      \renewcommand*{\subsubsectionautorefname}{Section}%
    }%
    \addto\extrasngerman{%
      \renewcommand*{\paragraphautorefname}{Absatz}%
      \renewcommand*{\subparagraphautorefname}{Unterabsatz}%
      \renewcommand*{\footnoteautorefname}{Fu\"snote}%
      \renewcommand*{\FancyVerbLineautorefname}{Zeile}%
      \renewcommand*{\theoremautorefname}{Theorem}%
      \renewcommand*{\appendixautorefname}{Anhang}%
      \renewcommand*{\equationautorefname}{Gleichung}%
      \renewcommand*{\itemautorefname}{Punkt}%
    }%
      % Fix to getting autorefs for subfigures right (thanks to Belinda Vogt for changing the definition)
      \providecommand{\subfigureautorefname}{\figureautorefname}%
    }{\relax}
\makeatother


% ********************************************************************
% Development Stuff
% ********************************************************************
\listfiles
%\PassOptionsToPackage{l2tabu,orthodox,abort}{nag}
%  \usepackage{nag}
%\PassOptionsToPackage{warning, all}{onlyamsmath}
%  \usepackage{onlyamsmath}


% ****************************************************************************************************
% 7. Further adjustments (experimental)
% ****************************************************************************************************
% ********************************************************************
% Changing the text area
% ********************************************************************
%\areaset[current]{312pt}{761pt} % 686 (factor 2.2) + 33 head + 42 head \the\footskip
\setlength{\marginparwidth}{4em}%
%\setlength{\marginparsep}{2em}%

% ********************************************************************
% Using different fonts
% ********************************************************************
%\usepackage[oldstylenums]{kpfonts} % oldstyle notextcomp
% \usepackage[osf]{libertine}
%\usepackage[light,condensed,math]{iwona}
%\renewcommand{\sfdefault}{iwona}
%\usepackage{lmodern} % <-- no osf support :-(
%\usepackage{cfr-lm} %
%\usepackage[urw-garamond]{mathdesign} <-- no osf support :-(
%\usepackage[default,osfigures]{opensans} % scale=0.95
%\usepackage[sfdefault]{FiraSans}
% \usepackage[opticals,mathlf]{MinionPro} % onlytext
% ********************************************************************
%\usepackage[largesc,osf]{newpxtext}
%\linespread{1.05} % a bit more for Palatino
% Used to fix these:
% https://bitbucket.org/amiede/classicthesis/issues/139/italics-in-pallatino-capitals-chapter
% https://bitbucket.org/amiede/classicthesis/issues/45/problema-testatine-su-classicthesis-style
% ********************************************************************
% ****************************************************************************************************

\usepackage[left=1.5in,right=1.5in,top=1in,bottom=1in,includehead,includefoot]{geometry}
\usepackage{amsmath} % AMS packages
\usepackage[amsthm,thmmarks]{ntheorem}
\usepackage{amssymb}
\usepackage{amsfonts}
\usepackage{mathrsfs}
\usepackage{graphicx} 
\usepackage{tikz-cd} %commutative diagrams
\usepackage{enumitem} %different types of numerations
\usetikzlibrary{arrows, matrix} 
\usepackage[noabbrev,capitalise,nameinlink]{cleveref} %clever cross-reference in the text
\usepackage{hyperref}
\usepackage{mathtools}


\PassOptionsToPackage{dvipsnames}{xcolor}
    \RequirePackage{xcolor} % [dvipsnames]
\definecolor{CTcitation}{rgb}{0,0.5,0} % WebGreen
\definecolor{CTurl}{named}{Maroon} % Maroon
\definecolor{CTtitle}{named}{Maroon} % Maroon {cmyk}{0, 0.87, 0.68, 0.32}
\definecolor{CTlink}{named}{RoyalBlue} % RoyalBlue {cmyk}{1, 0.50, 0, 0}
\hypersetup{    
    pdfpagemode=UseOutlines,
    colorlinks={true},
    urlcolor=CTurl, linkcolor=CTlink, citecolor=CTcitation
}



%enviroments

\theoremstyle{definition}
\newtheorem{definition}{Definition}[chapter]
\newtheorem{example}[definition]{Example}
\newtheorem{notation}[definition]{Notation}

\theoremstyle{plain}
\newtheorem{theorem}[definition]{Theorem}
\newtheorem{corollary}[definition]{Corollary}
\newtheorem{lemma}[definition]{Lemma}
\newtheorem{proposition}[definition]{Proposition}

\theoremstyle{remark}
\newtheorem{remark}[definition]{Remark}



%sets

\def\N{{\mathbb N}}
\def\Z{{\mathbb Z}}
\def\F{{\mathbb F}}
\def\Q{{\mathbb Q}}
\def\R{{\mathbb R}}
\def\C{{\mathbb C}}
\def\P{{\mathbb P}}
\def\O{{\mathcal O}}
\def\A{{\mathbb A}}

%operators
\DeclareMathOperator{\tr}{tr} %trace
\DeclareMathOperator{\res}{res} %residue
\DeclareMathOperator{\img}{Im} %image of a map
\DeclareMathOperator{\End}{End} %Set of Endomorphisms
\DeclareMathOperator{\Hom}{Hom} %Set of Homomorphisms
\DeclareMathOperator{\Der}{Der} %Set of Derivations
\DeclareMathOperator{\Mat}{M} %matrices%
%commands

\addbibresource{Bibliography.bib} 



\begin{document}
%\cleardoublepage\include{FrontBackmatter/Contents}
%!TEX root = ../main.tex
\chapter{Tate's Linear Algebra}\label{ch:tate-linear-algebra}
\section{Linear topologies}
Fix a ground field $k$. From now on, a vector space will always mean a $k$-vector space.
\begin{definition}\label{def:linear_topology}
A \textbf{linear topology} on a vector space $V$ is a separated (Hausdorff) topology invariant under translations that admits an open local base around zero of vector subspaces. A vector space equipped with a linear topology will be referred as \textbf{linearly topologized}.
\end{definition}
If we endow $k$ with the discrete topology then $V$ will become a topological vector space. From now on, endow $k$ with the discrete topology. \\
Linear topologies behave nicely under basic topological operations.
\begin{proposition}\label{prop:linear_topologies_properties}
Let $V$ be a linearly topologized vector space. Then
	\begin{enumerate}[label = (\alph*)]
		\item Any vector subspace of $V$ is linearly topologized under its subspace topology.
		\item If $W \subseteq V$ is a closed vector subspace then $V/W$ is linearly topologized under its quotient topology.
		\item If $\{V_{\alpha}\}_{\alpha}$ is a collection of linearly topologized vector spaces its product $\prod_{\alpha} V_{\alpha}$ and its direct sum $\bigoplus_{\alpha} V_{\alpha}$ is linearly topologized under its product topology.
		\item If $W$ is a vector subspace of $V$, then its topological closure $\overline{W}$ also is a vector subspace of $V$.
	\end{enumerate}
\end{proposition}
\begin{proof}
	Since intersection of vector subspaces is a vector subspace, (a) follows intersecting the fundamental system of neighborhoods in $V$ by the vector subspace. For (b), let $\pi\colon V \to V/W$ be the quotient map. Since $\pi$ is open and surjective the image of a local base is a local base; moreover, the image of a vector subspace under $\pi$ is a vector subspace. In addition, since $W$ is closed then $V/W$ is Hausdorff.  Now, for (c) let $\{U_{\alpha, \beta}\}_{\beta}$ be a local base of zero in $V_{\alpha}$ of vector subspaces, the products $U_{\alpha_{1}, \beta_{1}} \times \ldots \times U_{\alpha_{n}, \beta_{n}} \times \prod_{\gamma} V_{\gamma}$, where $\gamma$ ranges over $\alpha \neq \alpha_{1}, \ldots, \alpha_{n}$, for any set $\{(\alpha_{1}, \beta_{1}, \ldots, \alpha_{n}, \beta_{n})\}$ form a fundamental system of neighborhoods around zero in $\prod_{\alpha} V_{\alpha}$ of open vector subspaces. Note that since $\bigoplus_{\alpha} V_{\alpha} \subseteq \prod_{\alpha} V_{\alpha}$ is a vector subspace (c) follows from (a). Finally, for (d), suppose $x,y\in \overline{W}$, then, for every open vector subspace $U$, $(x + U)\cap W \neq \emptyset$ and $(y + U)\cap W \neq \emptyset$, therefore for every $\alpha, \beta \in k$ we have $(\alpha x + U)\cap W \neq \emptyset$ and $(\beta y + U)\cap W \neq \emptyset$. Hence, $(\alpha x + \beta y + U)\cap W\neq \emptyset$ for every open vector subspace $U$ and every pair $\alpha, \beta\in k$. It follows (d).
\end{proof}
\begin{remark}\label{rem:limits-and-colimits-in-lintop-category}
	Using an argument similar to the previous proposition one can check that in the category $\mathsf{LinTop}_k$ of linearly topologized vector spaces limits and colimits indexed by small categories exist.
\end{remark}
Linear topologies are discrete over a finite dimensional vector space.
\begin{proposition}\label{prop:finite_dimensional_linear_topologies}
A finite dimensional linearly topologized vector space $V$ is discrete.
\end{proposition}
\begin{proof}
	Let $U$ be an open vector subspace and $0 \neq x \in U$, since $V$ is separated there exists an open vector subspace $U_{x}$ such that $x \not\in U_{x}$. Thus, $\dim U_{x} \cap U < \dim U$. Since $V$ is finite dimensional this process can be repeated only a finite amount of times; that is $\{0\}$ is open. It follows that $V$ is discrete.
\end{proof}
\subsection*{Commensurability}
We introduce a partial order in the set of vector subspaces of a vector space $V$.
\begin{definition}\label{def:commensurability}
	For vector subspaces $A$ and $B$ of a vector space $V$ we say that $A \prec B$ if the quotient $A/(A\cap B) \cong (A+B)/B$ is finite dimensional (or equivalently $A \subseteq B + W$ for some finite dimensional $W$). In addition, we say that $A$ and $B$ are \textbf{commensurable} (denoted $A \sim B$) if $A \prec B$ and $B \prec A$.
\end{definition}
Observe that $A \sim B$ if and only if $(A+B)/(A\cap B) \cong A/(A\cap B) \oplus B/(A \cap B)$ is finite dimensional. We will constantly refer to a vector space $V$ being finite dimensional as $V \sim 0$.
\begin{proposition}\label{prop:equivalence-relation}
	Let $V$ be a vector spaces and $A,B$ and $C$ be vector subspaces, then:
	\begin{enumerate}[label = (\alph*)]
		\item If $A \sim B$ and $B \sim C$ then
		\[
			\frac{A+B+C}{A \cap B \cap C} \sim 0
		\]
		\item If $A \prec B$ and $B \prec C$ then $A \prec C$. Moreover, commensurability is an equivalence relation.
	\end{enumerate}
\end{proposition}
\begin{proof}
	Consider the following exact sequences
	\[
		0 \to \frac{A\cap B}{A \cap B \cap C} \to \frac{B}{B \cap C}, 
	\]
	and,
	\[
		0 \to \frac{A\cap B}{A \cap B \cap C} \to \frac{A+B}{A \cap B \cap C}
		\to \frac{A+B}{A \cap B} \to 0
	\]
	induced by inclusions. The first inclusion plus the fact that $B \sim C$ imply that $(A\cap B)/(A \cap B \cap C)$ is finite dimensional. Now, since $A \sim B$ it follows that $(A+B)/(A \cap B)$ is finite dimensional. Hence, the second exact sequence concludes that $(A+B)/(A \cap B \cap C)$. A symmetrical argument shows that $(B+C)/(A \cap B \cap C) \sim 0$. These prove (a). For (b), the inclusion
	\[
		0 \to \frac{A+C}{A\cap C} \to \frac{A+B+C}{A \cap B \cap C}
	\]
	plus (a) implies transitivity. 
\end{proof}
Now, we state and prove some useful properties on the relation $\prec$.
\begin{lemma}\label{lemm:properties-order-well-behaved-under-operations}
\begin{enumerate}[label = (\alph*)]
	\item If $A \subseteq B$ then $A \prec B$.
	\item If $A \prec B$ then $f(A) \prec f(B)$ for any $k$-linear map $f$
	\item It holds that
	\[
		\sum_{i=1}^{m} A_{i} \prec \bigcap_{j=1}^{n} B_{j} \iff A_{i} \prec B_{j}\text{ for all } i \text{ and } j.
	\]
\end{enumerate}
\end{lemma}
\begin{proof}
	First, (a) is immediate from the definition of $\prec$. Second, for (b) the map $f$ factors as
	\[
		A/(A\cap B) \to f(A)/(f(A)\cap f(B)) \to 0
	\]
	Finally, for (c), if $\sum_{i=1}^{m} A_{i} \prec \bigcap_{j=1}^{n} B_{j}$ holds then by (a) above, for all $i$ and $j$ we have
	\[
		A_{i} \prec \sum_{i=1}^{m} A_{i} \prec \bigcap_{j=1}^{n} B_{j} \prec B_{j}
	\]
	On the other hand, if $A_{i} \prec B_{j}$ for all $i$ and $j$ then there exists finite dimensional subspaces $W_{ij}$ such that $A_{i} \subseteq B_{j} + W_{ij}$ for all $i$ and $j$. Therefore,
	\[
		\sum_{i=1}^{m} A_{i} \subseteq \bigcap_{j=1}^{n} B_{j} + \sum_{i=1}^{m} \sum_{j=1}^{n} W_{ij}.
	\]

\end{proof}
Next, we consider another useful lemma.
\begin{lemma}\label{lemm:commensurability-addition-and-intersection}
	Let $A,B,A',B'$ be vector subspaces of a vector space $V$ and suppose that $A \sim A'$ and $B \sim B'$. Then $A + B \sim A' + B'$ and $A \cap B \sim A' \cap B'$.
\end{lemma}
\begin{proof}
	The following exact sequence 
	\small
	\[
		0 \to \frac{A + A' + B + B'}{A\cap A'\cap B\cap B'} \to \frac{A + A'}{A \cap A'} \oplus \frac{B + B'}{B \cap B'} \to \frac{A + A' + B + B'}{(A\cap A') + (B\cap B')} \to 0
	\]
	\normalsize
	plus $A \sim A'$ and $B \sim B'$ imply that both spaces
	\[
		\frac{A + A' + B + B'}{A\cap A'\cap B\cap B'} \quad\text{and,}
		 \frac{A + A' + B + B'}{(A\cap A') + (B\cap B')}
	\]	
	are finite dimensional. Since, $(A + A' + B + B')/(A+A')\cap(B+B')$ is a quotient of the second space and $((A \cap A') + (B \cap B'))/((A\cap A')\cap(B\cap B'))$ is a subspace of the first space we can conclude $A + B \sim A' + B'$ and $A \cap B \sim A' \cap B'$.
\end{proof}
If we consider the set of equivalence classes of $\sim$ on a vector space $V$ then $\prec$ is a partial order on it and by \cref{lemm:commensurability-addition-and-intersection} above it inherits operations $\cap$ and $+$.
\subsection*{Linear compactness}
\begin{definition}\label{def:linear_compactness}
	Let $V$ be a linearly topologized vector space. A closed subset $L \subseteq V$ is \textbf{linearly compact} (respectively \textbf{linearly cocompact}) if for every open vector subspace $U$ we have $L \prec U$ (respectively $V/(L+U) \sim 0$). 
\end{definition}
Linear compactness behaves just as compactness if one uses the correct words.
\begin{proposition}\label{prop:linear_compactness_properties}
	Let $V$ be a linearly compact vector space, then
	\begin{enumerate}[label = (\alph*)]
		\item If $A \subseteq V$ is a vector subspace such that for every open vector subspace $U$ of $V$ it holds $A \prec U$ then $\overline{A}$ is linearly compact.
		\item If $f\colon V \to W$ is a continuous linear homomorphism then $\overline{f(V)}$ is linearly compact.
		\item If $V$ is discrete then $V \sim 0$.
		\item Every closed vector subspace of $V$ is linearly compact.
		\item (Tychonov) If $\{V_{\alpha}\}_{\alpha}$ is a collection of linearly compact vector spaces then its product $\prod_{\alpha} V_{\alpha}$ and its direct sum $\bigoplus_{\alpha}V_{\alpha}$ are linearly compact.
	\end{enumerate}
\end{proposition}
\begin{proof}
	Let $U$ be any open vector subspace of $V$, then $A + U$ is closed, that is $A + U = \overline{A + U} \supseteq \overline{A} + U \supseteq A + U$, thus, $\overline{A} + U = A + U$. Since, $(A + U)/U \sim 0$ then $(\overline{A} + U)/U \sim 0$. We get (a). 

	For (b), since $f$ is a continuous linear map $V \prec f^{-1}(U)$ for all $U$ open vector subspace of $W$, hence by \cref{lemm:properties-order-well-behaved-under-operations} $f(V) \prec U$ for all open vector subspaces $U$ of $W$. By the previous observation and (a) we get (b). If $V$ is discrete, then $\{0\}$ is an open vector subspace of $E$, thus $V \prec U$, we get (c). 

	For (d), if $A \subseteq V$ is a closed vector subspace, and $V \prec U$ for all open vector subspaces $U$ by \cref{lemm:properties-order-well-behaved-under-operations} we get $A \prec U$. 

	Finally, for (d), it is enough proving for open vector subspaces $U = \prod_{\beta} U_{\beta} \times \prod_{\gamma} V_{\gamma}$ where $\beta$ ranges over a finite set, $\gamma$ ranges over $\alpha \neq \beta$ and $U_{\beta}$ is an open vector subspace of $V_{\beta}$. Then, the quotient
	\[
		\prod_{\alpha}V_{\alpha} / U \cong \prod_{\beta} V_{\beta}/U_{\beta}
	\]
	where $\cong$ is a topological and algebraic isomorphism. Since $V_{\alpha}$ is linearly compact for all $\alpha$ and $\beta$ ranges over a finite set we conclude that $\prod_{\alpha} V_{\alpha} / U$ is finite dimensional; therefore, $\prod_{\alpha} V_{\alpha}$ is linearly compact. The proof is analogous for $\bigoplus_{\alpha} V_{\alpha}$.
\end{proof} 
\subsection*{Completion}
\begin{definition}\label{def:completion}
	If $V$ be a linearly topologized vector space, recall that $V$ is said to be \textbf{complete} if 
	\[
		V \cong \varprojlim_{U \in \operatorname{Op}(V)} V/U
	\]
	where $\text{Op}(V)$ runs through all open vector subspaces of $V$. In particular, this implies that for every base $\mathscr{U}$ of zero made from open vector subspaces of $V$ we have
	\[
		V \cong \varprojlim_{U \in \mathscr{U}} V/U
	\]
\end{definition}

\section{Tate spaces}
\subsection*{Lattices}
\begin{definition}\label{def:c-lattice}
	If $V$ is a linearly topologized vector space we say that a \textbf{c-lattice} is an open linearly compact subspace of $V$, \textit{dually} a discrete linearly cocompact subspace is a \textbf{d-lattice}.
\end{definition}
First, we prove that existence of a c-lattice in a linearly topologized vector space is equivalent to existence of a d-lattice.
\begin{proposition}\label{prop:c-lattice-iff-d-lattice}
	A linearly topologized vector space $V$ has a c-lattice if and only if it has a d-lattice. 
\end{proposition}
\begin{proof}
	Suppose $L$ is a c-lattice in $V$, choose any direct complement $D$ of $L$, that is, $V = L \oplus D$. Since $L$ is open, then $D$ is discrete as $D\cap L = 0$, thus ${0}$ is open in $D$. Moreover, $D$ is closed as it is the fiber of $0$ under the projection $V \to L$ (which is continuous because $L$ is open). Finally, we check that $D$ is linearly cocompact: let $U$ be any open vector subspace of $V$, the composition $L \hookrightarrow V \twoheadrightarrow V/(D+U)$ induces a surjection 
	\[
		L/(L \cap U) \twoheadrightarrow V/(D+U)
	\]
	thus, since $\dim L / (L \cap U) < \infty$ we conclude $\dim V/(D+U) < \infty$. \\
	Now, suppose $D$ is a d-lattice. Thus, there exists an open vector subspace $U$ such that $U \cap D = 0$. This time, choose $L$ a direct complement for $D$ containing $U$. Then, the projection $$. Let $U$ be any open vector subspace, the composition $V \twoheadrightarrow L \twoheadrightarrow L/(L \cap U)$ induces a surjection
	 \[
	 	V/(D + (L \cap U)) \twoheadrightarrow L/(L \cap U)
	 \]
	 since both $L$ and $U$ are open, also $L\cap U$, thus $\dim V/(D + (L \cap U)) < \infty$. It follows, $\dim L/(L \cap U) < \infty$ and $L$ linearly compact.  
\end{proof}
\begin{remark}\label{up-to-finite-dimension}
	Note that in the proof of \cref{prop:c-lattice-iff-d-lattice} it is not strictly necessary to choose a direct complement, one can choose a direct complement up to finite dimension; that is, $L + D \sim V$ and $L \cap D \sim 0$. 
\end{remark}
We now give a characterization of lattices in terms of $\prec$.
%\begin{proposition}\label{prop:maximality-lattices}
	%A linearly compact subspace is a c-lattice if and only if it is maximal among the set of linearly compact sets ordered by $\prec$. 
%\end{proposition}
%\begin{proof}
	%\textcolor{red}{this one needs some thinking}
%\end{proof}
%not necessary
\begin{proposition}\label{prop:lattices-are-basis}
	If $V$ admits a c-lattice, then the set of c-lattices constitutes a base of zero of mutually commensurable vector subspaces.
\end{proposition}
\begin{proof}
	If $L$ and $L'$ are two c-lattices in $V$ then $L \prec L'$ and $L' \prec L$ because both are open; therefore, all c-lattices are commensurable. Moreover, if $U$ is any open vector subspace and $L$ is a c-lattice we claim that $L \cap U$ is a c-lattice. Indeed, let $U'$ be any open vector subspace, then $L\cap U \prec L\prec U'$. In addition, since $L$ and $U$ are open, $L \cap U$ is open. Thus $L \cap U \subseteq U$ is a c-lattice, this proves the statement.
\end{proof}
We're now ready to introduce the definition of a Tate space.
\begin{definition}\label{def:tate-vector-space}
	A \textbf{Tate space} $V$ is a complete linearly topologized vector space that admits a c-lattice. By the previous proposition and the observation in \cref{def:completion} we get
	\[
		V \cong \varprojlim_{L \in \mathscr{L}(V)} V/L
	\]
	where $\mathscr{L}(V)$ runs through all c-lattices of $V$.
\end{definition}
\begin{example}\label{ex:tate-spaces}
We give some examples of Tate spaces.
\begin{enumerate}[label = (\alph*)]
	\item Any vector space endowed with the discrete topology is a Tate space.
	\item If $\{V_{\alpha}\}_{\alpha}$ is any pro-system of finite dimensional vector spaces (thus, each one endowed with the discrete topology by \cref{prop:finite_dimensional_linear_topologies}), let $V$ be their inverse limit endowed with the inverse limit topology. We claim that this is a linearly compact space. Indeed, if we realize $V$ as a subspace of the product $\prod_{\alpha} V_{\alpha}$, then basic open vector subspaces are just restriction of finite coordinates. Hence, the quotient of $V$ by any basic open vector subspace is a finite product of $V_{\alpha}$, since all $V_{\alpha}$ are finite dimensional we conclude that $V$ is linearly compact and therefore a Tate space.
	\item Let $V = k\left((t)\right)$ with the topology generated by letting $t^{n}k\left[[t]\right]$ for $n \in \Z$ be a system of neighborhoods of zero. Then, $V = k\left[[t]\right] \oplus t k[t^{-1}]$ where $k\left[[t]\right]$ is the completion of $k[x]$ in the $\left\langle x\right\rangle$-adic topology, hence by the previous item linearly compact and, since it is open is a c-lattice. By the argument given in \cref{prop:c-lattice-iff-d-lattice} $t k[t^{-1}]$ is a d-lattice. Therefore, $V$ is a Tate space that is not linearly compact nor discrete.
	
	\end{enumerate}

\end{example}
\subsection*{Duality}
If $V$ is a Tate space we consider the following topology on the dual space $V^{*}$ (where by dual space we mean topological dual). Open vector subspaces are given by
\[
	L^{\perp} = \{\phi\in E^{*} \colon \phi\lvert_{L} = 0\}
\]
where $L$ is a linearly compact subspace. Equivalently, one can define open vector subspaces in $E^{*}$ to be $D^{*}$ where $D$ a direct complement of a linearly compact vector subspace $L$ in $E$ (in this case $D^{*} \hookrightarrow E^{*}$ using the decomposition $L\oplus D$).  \\
First, we prove that the word \emph{dually} in \cref{def:linear_compactness} actually makes sense. 

\begin{lemma}\label{lemm:duality-d-lattice-c-lattice}
	Duality interchanges linearly compact with discrete spaces and vice-versa. 
\end{lemma}
\begin{proof}
	If $L$ is a linearly compact vector space, then $L^{\perp}$ is open in $L^{*}$, thus $L^{*}$ is discrete. If $D$ is discrete, then $D \cong k^{\oplus \Lambda}$ for some $\Lambda$ and endowing $k^{\oplus \Lambda}$ with the discrete topology the previous isomorphism is a homeomorphism too. Moreover, since $D$ is discrete every linear functional is continuous. Using \cref{rem:limits-and-colimits-in-lintop-category} and the well known identity (where maps are isomorphisms in $\mathsf{LinTop}_{k}$)
	\[
		(k^{\oplus \Lambda})^{*} = \Hom_{k}(k^{\oplus \Lambda}, k) \cong \prod_{\Lambda} \Hom_{k}(k, k) \cong \prod_{\Lambda} k
	\]
	we get the desired result by Tychonov's theorem in \cref{prop:linear_compactness_properties}. 
\end{proof}
\begin{remark}\label{rem:dual-of-discrete-is-complete}
	Note that in the proof of the previous lemma the dual space of a discrete space is complete, as it is an arbitrary product of $k$ endowed with the product topology.
\end{remark}
\begin{proposition}\label{prop:dual-space-is-tate}
	If $V$ is a Tate space then $V^{*}$ with the topology previously introduced is also a Tate space.
\end{proposition}
\begin{proof}
	If we decompose $V = L \oplus D$ where $L$ is a c-lattice and $D$ a d-lattice then $V^{*} \cong L^{*} \oplus D^{*}$ and by \cref{lemm:duality-d-lattice-c-lattice} $L^{*}$ is discrete and $D^{*}$ is linearly compact. Observe that $D^{*}$ is open in $V^{*}$ since it is the kernel of the projection $V^{*} \to V^{*}/L^{\perp}$ and $V^{*}/L^{\perp}$ is discrete by the description of our topology in the dual $V^{*}$. Since $L^{*}$ is discrete, then it is complete. Moreover, by the previous remark, $D^{*}$ is complete, hence $V^{*}$ is complete too.
\end{proof}

We're ready to prove the analog of Pontryagin's duality for locally compact groups in our context.
\begin{theorem}\label{thm:self-duality}
	For a Tate space $V$ the canonical map $V \to V^{**}$ is an isomorphism.
\end{theorem}
\begin{proof}
	It is enough to prove it for complete linearly compact spaces and discrete spaces, as every Tate space can be decomposed into a direct sum of a c-lattice and a d-lattice. First, we do it for discrete spaces. Suppose $D$ is a discrete vector space. Then, the canonical map
	\[
		\operatorname{ev}\colon D \to D^{**}
	\]
	is open and continuous because $D$ and $D^{**}$ are both discrete by \cref{lemm:duality-d-lattice-c-lattice}. Moreover, it is injective, because for every nonzero $v \in D$ there exists a linear continuous functional $\phi\in D^{*}$ such that $\phi(v)\neq 0$. Finally, we prove surjectivity. Let $\psi \in D^{**}$. Since $\ker \psi$ is open it contains a basic open vector subspace $A^{\perp}$ such that $A \subseteq D$ is a linearly compact subspace. Therefore, since $D^{*}$ is linearly compact it follows that $D^{*} \sim A^{\perp}$, that is, the quotient $D^{*}/A^{\perp}$ is finite dimensional. Recall that the inclusion $\iota\colon A \to D$ induces an isomorphism $D^{*}/A^{\perp} \to A^{*}$ which is a homeomorphism since both spaces are discrete. We can factor $\psi$ so that the following diagram commutes
	\[
	\begin{tikzcd}
		D^{*} \arrow[r, "\psi"] \arrow[d] & k \\
		D^{*}/A^{\perp} \arrow[ru, dashed, "\tilde{\psi}"] \arrow[d, swap, "\cong"] & \\
		A^{*} \arrow[ruu, dashed, swap, bend right, "\overline{\psi}"]
	\end{tikzcd}
	\]
	However, $A^{*}$ is finite dimensional, therefore, there exists some $a \in A$ such that $\overline{\psi} = \operatorname{ev}_{a}$ as maps from $A^{*}\to k$. Moreover, since $A^{\perp} \subseteq \ker\psi$ we conclude that $\psi = \operatorname{ev}_{a}$ as maps $D^{*} \to k$. This implies surjectivity. Thus $D \to D^{**}$ is an isomorphism of topological vector spaces. \\

	Now, suppose $L$ is a complete linearly compact space. We check first that the map 
	\[
		\operatorname{ev}\colon L \to L^{**}
	\]
	is continuous. Let $A^{\perp}$ be an open vector subspace in $L^{**}$ where $A \subseteq L^{*}$ is a linearly compact subspace. By \cref{lemm:duality-d-lattice-c-lattice} $L^{*}$ is discrete, hence $A$ is finite dimensional. Suppose that $A = \operatorname{span}(\phi_{1},\ldots,\phi_{n})$ for some $\phi_{1},\ldots,\phi_{n} \in A$. Then, $\operatorname{ev}^{-1}(A^{\perp}) = \ker \phi_{1} \cap \ldots \cap \ker \phi_{n}$ which is open in $L$. Now, we check that $\operatorname{ev}$ is injective. Let $v \in L$ be a nonzero vector. Choose a decomposition of $L = U \oplus F$ where $U$ is open and $F$ is finite dimensional containing $v$ (this can be done because $L$ is separated and linearly compact). Let $\phi$ be a linear functional such that restricted to $U$ is zero and $\phi(v) \neq 0$. Since $U$ is open and $F$ discrete such $\phi$ exists and it is continuous. This implies injectivity of $\operatorname{ev}$.  Now we check that $\operatorname{ev}$ is surjective. Since $L$ is complete 
	\[
		L \cong \varprojlim_{U \in \mathscr{U}} L/U
	\]
	where $\mathscr{U}$ runs over open vector subspaces of $U$. Let $\psi\colon L^{*} \to k$ be a continuous linear functional. By pulling back $\pi_{U}\colon L \to L/U$ we get an injection $\pi_{U}^{*}\colon(L/U)^{*} \hookrightarrow L^{*}$ for every $U \in \mathscr{U}$. Since $L$ is linearly compact, then $L/U$ is finite dimensional, thus, there exists some $v_{U}\in L$ such that $\psi \circ \pi_{U}^{*} = \operatorname{ev}_{v_{U}}$ where $\operatorname{ev}\colon L/U \to (L/U)^{**}$. In particular, observe that if $U,U' \in \mathscr{U}$ and $U' \subseteq U$ we get an induced injection $(L/U)^{*} \hookrightarrow (L/U')^{*}$ such that the following diagram
	\[
	\begin{tikzcd}
		L^{*} \arrow[rr, "\psi"] & & k \\
		(L/U')^{*} \arrow[u, hook] \arrow[urr, swap, "\operatorname{ev}_{v_{U'}}"] & & \\
		(L/U)^{*} \arrow[u, hook] \arrow[uurr, swap, bend right, "\operatorname{ev}_{v_{U}}"] \arrow[uu, bend left = 70, hook]& & 
	\end{tikzcd}
	\]
	commutes. Observe that this implies that $(v_{U})_{U\in \mathscr{U}}$ is a Cauchy net and by completeness of $V$ it follows that it is convergent. Therefore, there exists some $v \in L$ limit of $(v_{U})_{U\in \mathscr{U}}$. We claim that $\psi = \operatorname{ev}_{v}$. Let $\phi \in L^{*}$. Then, $\ker\phi$ is open and since $L$ is linearly compact then $\ker\phi \sim L$. Hence, if we factor $\phi$ as follows
	\[
	\begin{tikzcd}
		L \arrow[r, "\phi"] \arrow[d, swap, "\pi_{\ker\phi}"] & k \\
		L/\ker\phi \arrow[ur, swap, "\overline{\phi}"] &
	\end{tikzcd}
	\]
	 since $L/\ker\phi$ is discrete we conclude that $\overline{\phi}$ is continuous. In other words, the image of $\overline{\phi}$ under the inclusion $(L/\ker\phi)^{*} \hookrightarrow L^{*}$ is $\phi$. Thus, $\psi(\phi) = \operatorname{ev}_{v_{\ker\phi}}(\overline{\phi})$ and by convergence $\psi(\phi) = \operatorname{ev}_{v}(\phi)$. This implies surjectivity of $\operatorname{ev}\colon L \to L^{**}$. To conclude, we prove that $\operatorname{ev}$ is open. Let $U$ be any open vector subspace in $L$, thus $L = U \oplus F$ for some $F$ finite dimensional. We claim that $\operatorname{ev}(U) = (F^{*})^{\perp}$. First, the inclusion $\operatorname{ev}(U) \subseteq (F^{*})^{\perp}$ is immediate. Let $\psi \in (F^{*})^{\perp}$. Let $v \in L$ such that $\operatorname{ev}_{v} = \psi$. Write $v = u + f$ where $u \in U$ and $f \in F$. Hence, $\operatorname{ev}_{v} = \operatorname{ev}_{u} + \operatorname{ev}_{f}$. Since $\operatorname{ev}$ is injective, it follows that there exists some $\phi\in F^{*}$ such that $\phi(f) \neq 0$ if $f$ is nonzero. Therefore, $f = 0$ and $\psi \in \operatorname{ev}(U)$. This concludes the proof.
\end{proof}
\begin{remark}\label{rem:completeness-is-necessary-for-sel-duality}
	Observe that completeness cannot be dropped in the definition of a Tate space while preserving duality. Indeed, if $V$ is linearly compact but not complete its dual is discrete by \cref{lemm:duality-d-lattice-c-lattice} and by \cref{rem:dual-of-discrete-is-complete} its double dual is complete, hence $V \to V^{**}$ cannot be an isomorphism.
\end{remark}
\subsection*{Morphisms}
A \textbf{morphism} of Tate spaces is a continuous linear homomorphism between Tate spaces.
\begin{definition}\label{def:linearly-compact-and-discrete-morphisms}
	A morphism $f\colon V\to W$ of Tate spaces is said to be \textbf{linearly compact} if the closure of $f(V)$ is linearly compact in $W$. Dually, it is \textbf{discrete} if $\ker f$ is open in $V$. 
\end{definition}
First, we check the duality natural property for morphisms of Tate spaces.
\begin{proposition}\label{prop:duality-discrete-compact-maps}
	A morphism $f\colon V\to W$ of Tate spaces is linearly compact if and only if $f^{*}$ is discrete.
\end{proposition}
\begin{proof}
	Suppose $f^{*}$ is linearly compact, then $\ker f^{*} = f(V)^{\perp}$. However, if $\phi\in W^{*}$ and $\phi(f(V)) = 0$ then $\phi(\overline{f(V)}) = 0$ by continuity of $\phi$. Therefore, $\ker f^{*} = \overline{f(V)}^{\perp}$ which is open because $\overline{f(V)}$ is linearly compact. Now, suppose $f^{*}$ is discrete. Thus, $\ker f^{*}$ contains a basic open subspace $A^{\perp}$ such that $A$ is linearly compact in $W$. Therefore, $f(V)\subseteq A$ then $\overline{f(V)} \subseteq A$ and by item (c) in \cref{prop:linear_compactness_properties} $\overline{f(V)}$ is linearly compact.
\end{proof}
Discrete and linearly compact operators form a 2-sided ideal in $\Hom$; that is
\begin{proposition}\label{prop:compact-and-discrete-2-sided-ideal}
	If $f$ is a linearly compact operator (respectively discrete) then its composition (from any side) with an arbitrary morphism of Tate spaces is also linearly compact (respectively discrete). 
\end{proposition}
\begin{proof}
	Let $A \xrightarrow{f} B \xrightarrow{g} C \xrightarrow{h} D$ be morphisms of Tate spaces such that $g$ is linearly compact. Then, $\overline{g\circ f(A)} \subseteq \overline{g(B)}$ which is linearly compact, thus $\overline{g\circ f(A)}$ is linearly compact too. On the other hand, note that $h(\overline{g(B)}) \subseteq \overline{h\circ g(B)}$; therefore $\overline{h(\overline{g(B)})} = \overline{h\circ g(B)}$. However, $\overline{g(B)}$ is linearly compact and by item(b) in \cref{prop:linear_compactness_properties} $\overline{h(\overline{g(B)})}$ is linearly compact. In addition, the statement for discrete operators follows from the previous proposition.
\end{proof}
\begin{definition}\label{def:2-sided-ideals-in-hom}
	Let $V$ and $W$ be Tate spaces. \linebreak We denote $\Hom_{+}(V,W)$ to be the set of linearly compact morphisms and $\Hom_{-}(V,W)$ the set of discrete ones. Also, set $\Hom_{0}(V,W)$ to be $\Hom_{+}(V,W) \cap \Hom_{-}(V,W)$.
\end{definition}
\begin{proposition}\label{prop:discrete-compact-operators-present-the-whole-space}
	The sets $\Hom_{-}(V,W), \Hom_{+}(V,W)$ and \linebreak $\Hom_{0}(V,W)$ are vector subspaces of $\Hom(V,W)$. Moreover,
	\[
		\Hom_{-}(V,W) + \Hom_{+}(V,W) = \Hom(V,W).
	\]
\end{proposition}
\begin{proof}
	Let $L$ be a c-lattice in $V$ and consider $\pi\colon V \to L$ be a continuous linear projection. Then $\pi$ realized as an element of $\End(V)$ satisfies $\pi \in\End_{+}(V)$ and $1 -\pi\in\End_{-}(V)$. Hence, by \cref{prop:compact-and-discrete-2-sided-ideal} for every $f\in \Hom(V,W)$ $f\circ \pi$ and $f\circ (1 - \pi)$ are linearly compact and discrete respectively. It follows
	\[
		\Hom_{-}(V,W) + \Hom_{+}(V,W) = \Hom(V,W).
	\]
	The other statements are immediate.
\end{proof}
\textcolor{red}{I'll include further theory if necessary.}

%!TEX root = ../main.tex
\chapter{Trace and Residue}\label{ch:trace-and-residue}
We extend the definition of trace to a certain class of infinite rank endomorphisms in order to define an abstract residue. We follow the original structure of Tate's elegant article \cite{tate} while translating his statements in the language of Tate's Linear Algebra.
\section{Finite-potent maps and their trace}
Let $k$ be a fixed field and $V$ a vector space over $k$. In this section we will extend the notion of trace of a linear endomorphism to include certain operators even when $V$ is infinite dimensional.
\subsection{Finite-potent maps}
\begin{definition}\label{def:finite-potent}
	We will say a linear map $f\colon V \to V$ is \textbf{finite-potent} if
	\[
		\dim f^{n}(V) < \infty
	\]
	for sufficiently large $n$.
\end{definition}
The following is characterization of finite-potent endomorphisms.
\begin{lemma}\label{lemm:characterization-of-finite-potent-maps}
	A linear map $f\colon V \to V$ is finite-potent if and only if there exists a subspace $W \subseteq V$ such that
	\begin{enumerate}[label = (\roman*)]
		\item $\dim f(W) < \infty$,
		\item $f(W) \subseteq W$ and
		\item the induced map $\bar{f}\colon V/W \to V/W$ is nilpotent.
	\end{enumerate}
	A subspace $W$ is a \textbf{trace-subspace} for $f$ if satisfies the previous properties.   
\end{lemma}
\begin{proof}
	If $f$ is finite-potent, choose $W = f^{n}(V)$ for sufficiently large $n$. The first condition follows from definition. Also, $f(W) = f^{n+1}(V) \subseteq f^{n}(V) = W$. In addition, $\bar{f}^{n} = 0$. On the other hand, if such $W$ exists, note that condition (ii) assures that $\bar{f}$ is well defined. Moreover, as $\bar{f}$ is nilpotent, $f^{n}V \subseteq W$ for sufficiently large $n$ and by condition (i) above $\dim f^{n}(V) < \infty$.
\end{proof}
Observe that a trace-subspace for a finite-potent map $f$ is not unique. In particular, if $W$ is trace-subspace for $f$ then $f^{n}(W)$ is trace-subspace for $f$ for all $n$.
\begin{notation}\label{not:trace}
	If $f$ is a finite-rank endomorphism in a vector space $V$, we will denote its ordinary trace by $\tr_{V}(f)$. Moreover, if $W$ is a subspace of $V$ invariant under $f$, that is, $f(W) \subseteq W$, then $\tr_{W}(f) := \tr_{W}(f\lvert_{W})$. In addition, if $\overline{f}$ is the induced map such that the following diagram commutes
	\[
	\begin{tikzcd}
		V \arrow[r, "f"] \arrow[d, "\pi_{W}"] & V \arrow[d, "\pi_{W}"] \\
		V/W \arrow[r, "\overline{f}"] & V/W,
	\end{tikzcd} 
	\]
	then $\tr_{V/W}(f) := \tr_{V/W}(\overline{f})$. The use of this notation is consistent throughout the document. 
\end{notation}
\subsection{Trace}  If $f$ is a finite-potent map and $W$ is a trace-subspace for $f$, the \textbf{trace} $\tr_{V}(f)\in k$ of $f$ is defined by
\[
	\tr_{V}(f) := \tr_{W}(f).
\]
Observe that $\tr_{W}(f)$ is well-defined because $f\lvert_{W}$ is of finite-rank. 
\begin{proposition}\label{prop:trace-does-not-depend-on-W}
	The definition of $\tr_{V}$ does not depend on the choice of trace-subspace for $f$.
\end{proposition}
\begin{proof}
	Suppose $W_{1}, W_{2} \subseteq V$ are two trace-subspaces for $f$, then $W = W_{1} + W_{2}$ is trace-subspace for $f$ as well. Hence, the induced maps on $W/W_{1}$ and $W/W_{2}$ are nilpotent. Therefore, $\tr_{W/W_{1}}(f) = \tr_{W/W_{2}}(f) = 0$ and using the well-known identify of the ordinary trace
\begin{align*}
	\tr_{W}(f) &= \tr_{W_{1}}(f) + \tr_{W/W_{1}}(f) \\
	\tr_{W}(f) &= \tr_{W_{2}}(f) + \tr_{W/W_{2}}(f),
\end{align*}
we obtain $\tr_{W_{1}}(f) = \tr_{W_{2}}(f)$, our desired result. 
\end{proof}
This definition extends some of the properties of the ordinary trace.
\begin{lemma}\label{lemm:properties-trace}
	\begin{enumerate}[label = (\alph*)]
		\item If $\dim V < \infty$, then any endomorphism $f$ is finite-potent and $\tr_{V}(f)$ coincides with the ordinary trace.
		\item If $f$ is nilpotent, then it is finite-potent and $\tr_{V}(f) = 0$.
		\item If $f$ is finite-potent and $U$ is a subspace such that $f(U) \subseteq U$, then the induced maps on $U$ and $V/U$ are finite-potent and satisfy the identity
		\[
		 	\tr_{V}(f) = \tr_{U}(f) + \tr_{V/U}(f).
		 \] 
	\end{enumerate}
\end{lemma}
\begin{proof}
	Both (a) and (b) are immediate. For (c), if $W$ is a trace-subspace for $f$, then $W\cap U$ and $(W + U)/U$ are trace-subspaces for the induced maps respectively. Hence, by \cref{lemm:characterization-of-finite-potent-maps}, both induced maps are finite-potent. Since $W/(W\cap U) \cong (W+U)/U$, the diagram
	\[
		\begin{tikzcd}
			W/(W\cap U) \arrow[r, "\cong"] \arrow[d, "f"] & (W+U)/U \arrow[d, "f"] \\
			W/(W\cap U) \arrow[r, "\cong"] & (W+U)/U
		\end{tikzcd}
	\]
	commutes. Moreover, recall that the ordinary trace is invariant under conjugation, that is, $\tr_{W}(\varphi \circ f \circ \varphi^{-1}) = \tr_{W}(f)$ for every automorphism $\varphi$ of $W$. Therefore, it follows that $\tr_{W/(W\cap U)}(f) = \tr_{(W+U)/U}(f)$. We conclude that
	\[
	 	\tr_{V}(f) = \tr_{W}(f) = \tr_{W\cap U}(f) + \tr_{(W+U)/U}(f) = \tr_{U}(f) + \tr_{V/U}(f).
	\]
\end{proof}
\begin{definition}\label{def:finite-potent subspace}
	A subspace $F$ of $\End_{k}(V)$ is said to be a \textbf{finite-potent subspace} if there exists an $n$ such that for any family of $n$ elements $f_{1}, \ldots, f_{n}\in F$ the space $f_{1}f_{2}\cdots f_{n}V$ is finite dimensional.
\end{definition}
Observe that if $F$ is a finite-potent subspace of $\End_{k}(V)$, then every $f\in F$ is finite-potent. 
\begin{proposition}\label{prop:linearity-trace}
	If $F$ is a finite-potent subspace, then $\tr_{V}\colon F\to k$ is $k$-linear.
\end{proposition}
\begin{proof}
	It is enough to prove the statement in the case where $F$ is finite dimensional. Let $W = F^{n}V$ for $n$ as in the definition of finite-potent subspace. Then $\dim W < \infty$. Hence, $W$ is a trace-subspace for all $f\in F$. It follows that $\tr_{V}(f) = \tr_{W}(f)$ for all $f$. Since $\tr_{W}\colon \End_{k}(V) \to k$ is $k$-linear, so is $\tr_{V}\colon F \to k$.
\end{proof}
\begin{remark}\label{rem:general-linearity-trace}
	In his paper, Tate asked if general linearity for finite-potent maps followed. His question was answered negatively in \cite{TATE-TRACE-COUNTER-EXAMPLE} where general linearity is reduced to the following question: if the sum of two nilpotent endomorphisms is finite-potent, is the sum necessarily traceless?  
\end{remark}
\begin{proposition}\label{prop:trace-inavriant-under-commutator}
	If $f,g \in \End_{k}(V)$ and $fg$ is finite-potent, then so is $gf$ and 
	\[
		\tr_{V}(fg) = \tr_{V}(gf).
	\]
\end{proposition}
\begin{proof}
	Since $fg$ is finite-potent, $W =(fg)^{n}V$ is finite-dimensional for a sufficiently large $n$. On the other hand, $(gf)^{n+1}V= g(fg)^{n}f(V) \subseteq g(W)$; therefore, $gf$ is also finite-potent. Let $W' = (gf)^{n}V$. Then $g(W') \subseteq W$ and $f(W) \subseteq W'$. Thus, 
	\[
		\dim W' \leq \dim g(W) \leq \dim W,
	\]
	and, 
	\[
		\dim W \leq \dim f(W) \leq \dim W',
	\]
	which implies that $W\cong W'$ and that $g$ and $f$ induce mutually inverse isomorphisms between $W$ and $W'$. Moreover, the diagram
	\[
		\begin{tikzcd}
			W \arrow[r, "fg"]\arrow[d, "g"] & W \arrow[d, "g"] \\
			W' \arrow[r, "gf"] & W'
		\end{tikzcd}
	\]
	commutes. We conclude that $\tr_{W}(fg) = \tr_{W'}(gf)$. Hence, $\tr_{V}(fg) = \tr_{V}(gf)$.
\end{proof}
\subsection{Trace and Tate Spaces}\label{Tate-and-trace}
Suppose that $V$ is a Tate space and consider $\End_{k}(V)$ the space of continuous endomorphisms of $V$. By \cref{prop:compact-and-discrete-2-sided-ideal} and \cref{prop:discrete-compact-operators-present-the-whole-space}, there are 2-sided ideals $\End_{0}(V), \End_{+}(V)$ and $\End_{-}(V)$ of $\End_{k}(V)$ such that $\End_{+}(V) + \End_{-}(V) = \End_{k}(V)$ and $\End_{0}(V) = \End_{+}(V) \cap \End_{-}(V)$. Moreover, \cref{rem:discrete-composition-compact} implies that $\End_{0}(V)$ is a finite-potent subspace.
\begin{lemma}\label{lemm:traceless-commutator}
	Suppose that $f \in \End_{+}(V)$, $g \in \End_{-}(V)$ or $f \in \End_{-}(V)$ and $g\in \End_{+}(V)$. Then the commutator $[f,g] = fg - gf$ belongs to $\End_{0}(V)$ and it is traceless.
\end{lemma}
\begin{proof}
	This immediate from the previous discussion and \cref{prop:trace-inavriant-under-commutator}. \\
\end{proof}

\section{Differential Calculus}
In this section we introduce the theory of derivations and differentials over an arbitrary commutative $k$-algebra $A$. Let $M$ be an $A$-module. We follow \cite{EGA4} Section 20 and \cite{Matsumura} Section 25. 
\begin{definition}\label{def:derivations-and-differentials}
	A \textbf{derivation} from $A$ to $M$ is a map $D\colon A \to M$ satisfying properties
	\begin{enumerate}[label = (\roman*)]
			\item $D(a + b) = D(a) + D(b)$ and,
			\item (\textit{Leibniz Rule}) $D(ab) = aD(b) + bD(a)$ 
	\end{enumerate}
for all $a,b \in A$.

The set of derivations from $A$ to $M$ is an $A$-module in the natural way. We will denote it by $\Der(A,M)$. Moreover, if $A$ is a $k$-algebra through a map $\varphi\colon k \to A$ we say that $D$ is a $\mathbf{k}$-\textbf{derivation} if $D$ is a derivation and $D \circ \varphi = 0$. In this case, the set of all $k$-derivations is denoted $\Der_{k}(A,M)$. If $M = A$, we will denote $\Der_{k}(A,A)$ simply by $\Der_{k}(A)$.
\end{definition}
\begin{definition}\label{def:extension-of-algebras}
Let $B$ be a $k$-algebra and $C$ an ideal in $B$ with $C^{2} = 0$; set $A =B/C$. In this way, $C$ can be viewed as an $A$-module. In this situation, we say that $B$ is an \textbf{extension} of the $k$-algebra $A$ by the $A$-module $C$. We simply write the exact sequence
\[\label{eqn:extension}
	0 \to C \to B \xrightarrow{\pi} A \to 0.
\]
Moreover, we will say that such sequence \textbf{splits} if there exists a retraction; that is, a $k$-algebra homomorphism $\rho\colon A \to B$ such that $\pi \circ \rho = 1_{A}$. In this case, we can identify $B = C\oplus A$. Conversely, starting from any $k$-algebra $A$ and any $A$-module $C$, one can always define a structure on $A \oplus C$ such that $A\oplus C$ is an extension of $A$ by $C$. Namely, 
\[
			(a,c)(a',c') = (aa',ac' + a'c)
\]
for $a,a'\in A$ and $c,c' \in C$. Common notations for this algebra are $D_{A}(C)$ or $A * C$.		
\end{definition}
\begin{definition}\label{def:lifting}
Given a commutative diagram of $k$-algebras
\[
	\begin{tikzcd}
		 	B \arrow[r, "f"] & A \\
		 	& C\arrow[lu, "h"] \arrow[u, "g"]
	\end{tikzcd} 
\]
thinking of $f$ as a fixed map; we say that $h$ is a \textbf{lifting} of $g$ to $B$. 
\end{definition}
\begin{lemma}\label{lemm:two-liftings}
	Let $h$ and $h'\colon C \to B$ be two liftings of $g$ to $B$. Let $K = \ker f$ and suppose $K^{2} = 0$. Then $h - h'\colon C \to K$ is a $k$-derivation. Conversely, if $D \in \Der_{k}(C, K)$, then $h + D$ is another lifting of $g$ to $B$.
\end{lemma}
\begin{proof}
	First, observe that $(h - h')(C)$ lies in $K$ because both $h$ and $h'$ are liftings of $g$ to $B$. Since $K^{2} = 0$, $K$ can be considered as $f(B)$-module and by means of $g$ as a $C$-module. Then $h-h'\colon C \to K$ is a map of $C$-modules. Now, let $c,c' \in C$, then
	\begin{align*}
	(h -h')(cc') &= h(c)h(c') - h'(c)h'(c') \\
	&= h(c)h(c') - h'(c)h'(c') - h(c)h'(c') + h'(c')h(c).
	\end{align*}
	Since $c \cdot k = h(c)k = h'(c)k$ for all $k \in K$, it follows that
	\begin{align*}
		(h -h')(cc') &= c\cdot h(c') - c'\cdot h'(c') - c\cdot h'(c') + c'\cdot h(c) \\
		&= c\cdot (h - h')(c') + c'\cdot (h - h')(c).
	\end{align*}
	This implies that $h - h'$ is a $k$-derivation. Observe that $h + D$ is a lifting of $g$ to $B$ because $D(C)$ lies in $K$.
\end{proof}
\begin{theorem}\label{prop:differentials-representable-functor}
	If $A$ is a $k$-algebra, consider the covariant functor from the category $\mathsf{Mod}_{A}$ to itself given by $M \mapsto \Der_{k}(A,M)$. This functor is representable. 
\end{theorem}
\begin{proof}
	Let $\mu\colon A \otimes_{k} A \to A$ be the $k$-algebra homomorphism given by $f \otimes g \to fg$. Set
	\[
		I = \ker \mu, \quad \Omega_{A/k} = I/I^{2}, \quad \text{and,} \quad B = (A \otimes_{k} A)/I^{2}.
	\]
	Thus, $\mu$ induces $\mu'\colon B \to A$ such that
	\[
		0 \to \Omega_{A/k} \to B \to A \to 0
	\]
	is an extension of $A$ by $\Omega_{A/k}$. We claim that this extension splits. Moreover, it has two splittings. Indeed consider the retractions
	\[
		j_{1}\colon A \to B \quad\text{and,}\quad j_{2}\colon A \to B
	\]
	defined by $a \mapsto a\otimes 1$ mod $I^{2}$ and $a \mapsto 1 \otimes a$ mod $I^{2}$. By \cref{lemm:two-liftings}, $d := j_{2} - j_{1}$ is a $k$-derivation of $A$ to $\Omega_{A/k}$. Now, we prove that 
	\begin{align}\label{eqn:representable}
		\Der_{k}(A,M) &\cong \Hom_{A}(\Omega_{A/k}, M).
	\end{align}	
	Let $D \in \Der_{k}(A,M)$ and define $\varphi\colon A \otimes_{k} A \to A * M$ by $\varphi(x \otimes y) = (xy, xD(y))$. Then $\varphi$ is a $k$-algebra homomorphism because it is compatible with the operation in $A * M$ defined in \cref{def:extension-of-algebras}. In addition, if $\sum x_{i} \otimes y_{i}$ lies in $I$, then
	\[
		\mu\left(\sum x_{i} \otimes y_{i}\right) = \sum x_{i}y_{i} = 0 \implies \varphi\left(\sum x_{i} \otimes y_{i}\right) = (0, \sum x_{i}D(y_{i}))
	\]
	whence $\varphi(I)$ lies in $M$. Moreover, by Leibniz's Rule $\varphi$ factors through $I^{2}$ yielding a map $f\colon \Omega_{A/k}\to M$. For $a \in A$ it follows that 
	\begin{align*}
		f(da) &= f(1 \otimes a - a \otimes 1 \mod I^{2}) = \varphi(1 \otimes a) - \varphi(a \otimes 1) \\
		&= D(a) - aD(1) = D(a). 
	\end{align*} 	
	Therefore, $D = f\circ d$. Now, we prove that such $f$ is unique. First, observe that $\Omega_{A/k}$ has the $A$-module structure induced by multiplication by $a \otimes 1$ (or $1 \otimes a$ since $1 \otimes a - a \otimes 1 \in I$). Therefore, if $\xi = \sum x_{i} \otimes y_{i} \mod I^{2} \in \Omega_{A/k}$, then $a\xi = \sum ax_{i}\otimes y_{i} \mod I^{2}$, and $f(a\xi) = \sum ax_{i}D(y_{i}) = af(\xi)$, so that $f$ is $A$-linear. We have
	\[
		a \otimes a' = (a \otimes 1)(1 \otimes a' - a' \otimes 1) +  aa' \otimes 1,
	\]
	so that if $\omega = \sum x_{i}\otimes y_{i} \in I$, then $\omega \mod I^{2} = \sum x_{i} dy_{i}$ because $\sum x_{i}y_{i} = 0$. We conclude that $\{da\mid a\in A\}$ is a set of generators for the $A$-module $\Omega_{A/k}$. This implies uniqueness of $f$. Therefore, (\ref{eqn:representable}) holds.
\end{proof}
\begin{definition}\label{def:module-of-differentials}
	The module $\Omega_{A/k}$ introduced in the proof of the previous theorem is called the \textbf{module of differentials} of $A$ over $k$ or \textbf{module of Kähler differentials}, and for $a \in A$ the element $da\in \Omega_{A/k}$ is called the \textbf{differential} of $a$.
\end{definition}
\begin{example}\label{ex:differentials-of-polynomials}
	If $A$ is generated as $k$-algebra by a subset $S \subseteq A$, then $\Omega_{A/k}$ is generated by $\{ds\mid s \in S\}$. Indeed, if $a \in A$, then there exist $a_{i}\in S$ and a polynomial $f(X) \in k[X_{1}, \ldots, X_{n}]$ such that $a = f(a_{1}, \ldots, a_{n})$. Thus,
	\[
		da = \sum_{i=1}^{n} f_{i}(a_{1}, \ldots, a_{n})da_{i} \quad \text{where}\quad f_{i} = \frac{\partial f}{\partial x_{i}}.
	\]
	In particular, if $A = k[X_{1}, \ldots, X_{n}]$ then $\Omega_{A/k} = A dX_{1} + \ldots A dX_{n}$ because $X_{1}, \ldots, X_{n}$ are linearity independent; this follows from the fact that $\partial_{i} X_{j} = \delta_{ij}$.
\end{example}
\begin{lemma}\label{lemm:c-map-differentials}
	Let $K$ be a $k$-commutative algebra. The map $c\colon K \otimes_{k} K \to \Omega_{K/k}$ defined by $c(f \otimes g) = fdg$ satisfies:
	\begin{enumerate}[label = (\roman*)]
		\item $c$ is surjective.
		\item $\ker c$ is generated over $k$ by the elements of the form
		\[
			f \otimes gh - fg \otimes h - fh \otimes g
		\]
	\end{enumerate}
\end{lemma}
\begin{proof}
	The $k$-bilinear map $(f,g) \mapsto fdg$ induces $c$. Since $\{df \mid f\in K\}$ is a generating set for $\Omega_{K/k}$ as a $K$-module, it follows that $c$ is surjective. For (ii), observe that it is equivalent proving that $\ker(c)$ is generated over $K$ by the elements of the form $1 \otimes gh - g \otimes h - h \otimes g$. Let $A$ be the $K$-module generated by those elements. We wish to prove that
	\[
		A \to K \otimes_{k} K \to \Omega_{K/k} \to 0
	\]
	is exact. By left-exactness of $\Hom$ it is equivalent to prove that for all $K$-modules $M$ the induced sequence
	\[
		0 \to \Hom_{K}(\Omega_{K/k}, M) \to \Hom_{K}(K \otimes_{k} K, M) \to \Hom_{K}(A, M) 
	\]
	is exact. By \cref{prop:differentials-representable-functor}, there is a canonical isomorphism $\Hom_{K}(\Omega_{K/k}, M) \cong \Der_{k}(K, M)$. Under this identification, we wish to prove that
	\[
		0 \to \Der_{k}(K, M) \to \Hom_{K}(K \otimes_{k} K, M) \to \Hom_{K}(A, M) 
	\]
	is exact. Observe that the first map is given by $D \mapsto \varphi_{D}$ where $\varphi_{D}(f \otimes g) = f D(g)$. Note that the restriction $\varphi_{D}\colon A \to M$ is trivial. Indeed,
	\[
		\varphi_{D}(1 \otimes gh - g \otimes h - h \otimes g) = D(gh) - gD(h) - hD(g) = 0
	\]
	by the Leibniz rule. Now, let $\psi\in \Hom_{K}(K \otimes_{k} K, M)$ so that $\psi(A) = 0$. Let $D_{\psi}\colon K \to M$ be the $k$-derivation defined by $f \mapsto \psi(1 \otimes f)$. First, we prove that $\psi_{D}$ is a $k$-derivation. Observe that $k$-linearity is obvious. Now, we prove the Leibniz rule for $D_{\psi}$. Consider
	\begin{align*}
		D_{\psi}(fg) = \psi(1 \otimes fg) &= \psi(f \otimes g + g \otimes f) \\ &= f \psi (1 \otimes g) + g \psi (1 \otimes f) \\ &= f D_{\psi} (f) + g D_{\psi} (g),
	\end{align*}
	where the third equality follows from the fact that $\psi$ vanishes on $A$. Finally, it is clear that $\varphi_{D_{\psi}} = \psi$. 
\end{proof}
\section{Abstract Residue and its properties}
\subsection{Existence of residue map}
Throughout this section let $k$ be a field, $K$ a commutative $k$-algebra with $1$, and $V$ a $K$-module so that when viewed as a $k$-vector space it is a Tate space and $K$ acts continuously on $V$. Namely, for all $f \in K$ the map 
\begin{align*}
	f\colon &V \to V \\
	&x \mapsto fx 
\end{align*}
is continuous. In this way, $K$ operates on $V$ through $\End_{k}(V)$ (maintaining notation from \cref{Tate-and-trace}). We will not notationally distinguish $f\in K$ from its induced map in $\End_{k}(V)$.
\begin{lemma}\label{lemm:trace-only-depends-on-K}
	Let $f,g \in K$. Then there are $f_{+}, g_{+} \in \End_{+}(V)$ so that 
	\[
	f = f_{+} \mod \End_{-}(V) ,\quad g = g_{+} \mod \End_{-}(V)
	\]
	and, the equality
	\[
		\tr([f_{+},g_{+}]) = \tr([f, g_{+}]) = \tr([f_{+}, g])
	\]
	holds.
\end{lemma}
\begin{proof}
The existence of $f_{+}$ and $g_{+}$ is immediate from the fact that $\End_{k}(V) = \End_{+}(V) + \End_{-}(V)$.  Clearly $[f_{+},g_{+}]\in \End_{+}(V)$. Moreover, the fact that $K$ is commutative implies that $[f,g] = 0$. Therefore,
\[
	[f_{+},g_{+}] = [f,g] \mod \End_{-}(V).
\]
Hence, $[f_{+},g_{+}]\in E_0$. Similarly, $[f,g_{+}]$ and $[f_{+}, g]$ belong to $\End_{0}(V)$. Whence, one can consider their trace. Furthermore, if $f_{+} \in \End_{+}(V)$ and $g_{+} - g \in \End_{-}(V)$, then $\tr([f_{+}, g_{+} - g]) = 0$ by \cref{lemm:traceless-commutator}. We conclude that $\tr([f_{+}, g_{+}]) = \tr([f_{+}, g])$. The other equality follows similarly.
\end{proof}
\begin{notation}\label{not:plus-and-minus}
	\cref{lemm:trace-only-depends-on-K} implies that common values of traces $[f_{+},g_{+}]$, $[f_{+}, g]$ and $[f, g_{+}]$ depend only on $f$ and $g$ and not in the choice of $f_{+}$ and $g_{+}$. Therefore, we will always denote $f_{\pm}$ to be elements in $\End_{\pm}(V)$ such that
	\[
		f = f_{+} \mod \End_{-}(V) ,\text{ and } f = f_{-} \mod \End_{+}(V).
	\] 
\end{notation}
\cref{lemm:trace-only-depends-on-K} implies that the assignment $(f,g) \mapsto \tr([f_{+}, g_{+}])$  is well-defined. Observe that this assignment is $k$-bilinear by \cref{prop:linearity-trace}. Thus, there exists a map
\begin{align*}
	r\colon &K\otimes_{k} K \to k \\
	&f \otimes g \mapsto \tr([f_{+}, g_{+}]).
\end{align*}
With these tools at our hands we are ready to prove the existence of residue.
\begin{theorem}\label{thm:existence-of-residue}
	There exists a unique $k$-linear map
	\[
		\res_{V}\colon \Omega_{K/k}\to k
	\]
	such that for each pair of elements $f,g \in K$ we have
	\[
		\res_{V}(fdg) = \tr([f_{+}, g_{+}]).
	\]
\end{theorem}
\begin{proof}
	Let $c\colon K \otimes_{k} K \to \Omega_{K/k}$ be as in \cref{lemm:c-map-differentials}. Since $c$ is surjective, if $\res_{V}$ exists it is uniquely determined by the commutativity of the following diagram
	\[
	\begin{tikzcd}
		K \otimes_{k} K \arrow[r, "r"] \arrow[d, two heads, "c"] & k \\
		\Omega_{K/k} \arrow[ur, dotted, swap, "\res_{V}"] &
	\end{tikzcd}
	\]
	Therefore, such map exists if and only if it vanishes on $\ker c$. To see this, let $f,g$ and $h$ in $K$ and choose $f_{+}, g_{+}$ and $h_{+}$ in $\End_{+}(V)$ coinciding with $f, g$ and $h$ modulo $\End_{-}(V)$ respectively. Then,
	\[
		fg = f_{+}g_{+} + (f_{+}g_{-} + f_{-}g_{+} + f_{-}g_{-}),
	\]
	and $f_{+}g_{-} + f_{-}g_{+} + f_{-}g_{-} \in \End_{-}(V)$. Whence, $(fg)_{+} = f_{+}g_{+}$. Analogously $(gh)_{+} = g_{+}h_{+}$ and $(fh)_{+} = f_{+}h_{+}$. This fact and the identify
	\[
		[f_{+},g_{+}h_{+}] - [f_{+}g_{+}, h_{+}] - [f_{+}h_{+}, g_{+}] = 0
	\]
	imply the desired conclusion.
\end{proof}
\subsection{Properties of residue}
We prove some of the main properties of residue.
\begin{proposition}\label{prop:linearity-residue}
	For all $f,g \in K$ it follows that
	\begin{enumerate}[label = (\alph*)]
		\item $\res_{V}(fdg) + \res_{V}(gdf) = 0$, and
		\item $\res_V(df) = 0$.
	\end{enumerate}
\end{proposition}
\begin{proof}
	Since $[f_{+}, g_{+}] + [g_{+}, f_{+}] = 0$, we get (a). For (b) use (a) with $g = 1$. 
\end{proof}
\begin{proposition}\label{prop:linearity-residue-closed-submodule}
	Let $W$ be a closed $K$-submodule of $V$. Then, for $\omega \in \Omega_{K/k}$, the identity
	\[
		\res_{V}(\omega) = \res_{W}(\omega) + \res_{V/W}(\omega)
	\]
	holds.
\end{proposition}
\begin{proof}
	It is enough to prove the claim for $\omega = f dg$. By \cref{lemm:properties-trace} item (c), we only need to check that for all $f \in K$ the induced map $\overline{f}\colon V/W \to V/W$ and $f \circ \iota$, where $\iota$ denotes the inclusion $W \to V$, satisfy
	\begin{align*}
	\overline{f} &= \overline{f_{+}} \mod \End_{-}(V/W), \\
	f \circ \iota &= f_{+} \circ \iota \mod \End_{-}(W), \\ 
	\overline{f_{+}}&\in \End_{+}(V/W), \quad\text{and} \\
	f_{+} \circ \iota &\in \End_{+}(W).
	\end{align*}
	These statements are straightforward from definitions.
\end{proof}
\begin{proposition}\label{prop:direct-sum-residue}
	If $V$ is the direct sum of two closed submodules $W_{1}$ and $W_{2}$, then 
	\[
		\res_{V}(\omega) = \res_{W_{1}}(\omega) + \res_{W_{2}}(\omega)
	\]
	holds for all $\omega \in \Omega_{K/k}$.
\end{proposition}
\begin{proof}
	Immediate from \cref{prop:linearity-residue-closed-submodule}.
\end{proof}
If our Tate space is trivial, so its residue.
\begin{proposition}\label{prop:residue-trivial-tate-space}
	If $V$ is either linearly compact or discrete, then $\res_{V}(\Omega_{K/k}) = 0$.
\end{proposition}
\begin{proof}
	If $V$ is linearity compact, then $\End_{+}(V) = \End_{k}(V)$ and $f_{+} = f$ for all $f \in K$. Since $[f,g] = 0$, it follows that 
	\begin{align}\label{eqn:trivial-residue}
		\res_{V}(fdg) = 0.
	\end{align}
	On the other hand, if $V$ is discrete, then $\End_{k}(V) = \End_{-}(V)$. Hence, $f = 0 \mod \End_{-}(V)$ for all $f \in K$. Thus, (\ref{eqn:trivial-residue}) holds.
\end{proof}
\begin{proposition}\label{prop:residue-and-continuity}
	Let $f$ and $g$ belong to $K$. Then, if there exists a c-lattice $L$ in $V$ so that $fL + fgL + fg^{2}L \subseteq L$, it holds $\res_{V}(fdg) = 0$. In particular, if there exists $L$ a c-lattice so that $fL \subseteq L$ and $gL \subseteq L$, then $\res_{V}(fdg) = 0$.
\end{proposition}
\begin{proof}
	Let $\pi$ be a continuous projection from $V$ to $L$. Then $\pi f \in \End_{+}(V)$ and $\pi f = f \mod \End_{-}(V)$. Thus, it follows that
	\[
		\res_{V}(fdg) = \tr([\pi f, g]) 
	\] 
	by \cref{lemm:trace-only-depends-on-K}. Let $h = [\pi f,g]$ and $W = L + gL$. Let $h_{V/W}$ and $h_{W}$ be the induced maps on $V/W$ and $W$ respectively. Then the relation $fL + fgL + fg^{2}L \subseteq L$ implies that $h_{V/W} = 0$ and $h_{W} = 0$. By \cref{lemm:properties-trace} item (c), we conclude that
	\[
		\res_{V}(fdg) = \tr_{V}(h) = \tr_{W}(h) + \tr_{V/W}(h) = 0.
	\]
\end{proof}
In the following two propositions we examine the residue of a power.
\begin{proposition}\label{prop:residue-of-a-power}
	Let $f \in K$, then $\res_{V}(f^{n}df) = 0$ for all $n \geq 0$. Moreover, if $f$ is invertible the same identity holds for $n \leq -2$.
\end{proposition}
\begin{proof}
	First, if $f_{+} = f \mod \End_{-}(V)$, then $f_{+}^{n} = f^{n} \mod \End_{-}(V)$. Therefore,
	\[
		\res_{V}(f^{n} df) = \tr([f_{+},f_{+}^{n}]) = 0.
	\]
	Second, if $f$ is invertible, then
	\[
		fd(f^{-1}) + f^{-1}df = d(ff^{-1}) = d(1) = 0.
	\]
	This implies that
	\[
	f^{-2}df = -d(f^{-1}).
	\]
	Hence, multiplying by $f^{-n}$ on both sides, where $n\geq 0$, it follows that
	\[
	f^{-2-n}df = -(f^{-1})^{n}d(f^{-1}).
	\]
	By the preceding statement, $(f^{-1})^{n}d(f^{-1})$ has zero residue. 
\end{proof}
\begin{proposition}\label{prop:residue-of-invertible-element}
	If $f$ is invertible, so that $fL \subseteq L$ for some c-lattice $L$, then
	\[
		\res_{V}(f^{-1}df) = \dim_{k}(L/fL).
	\]
\end{proposition}
\begin{proof}
	If $\pi$ is a continuous projection of $V$ into $L$, then 
	\[
	\res_{V}(f^{-1}df) = \tr([\pi f^{-1}, f]). 
	\]
	Let $g = [\pi f^{-1}, f]$. Since $fL\subseteq L$, we obtain
	\[
		g_{V/L} = 0, \quad g_{L/fL}= 1\quad\text{and,}\quad g_{fL} = 0, 
	\]
	where $g_{V/L}, g_{L/fL}$ and $g_{fL}$ denote the induced maps in $V/L$, $L/fL$ and, $fL$ respectively. Then by \cref{lemm:properties-trace} item (c), it follows that
	\[
		\tr_{V}(g) = \tr_{L}(g) + \tr_{V/L}(g) = \tr_{fL}(g) + \tr_{L/fL}(g) + \tr_{V/L}(g).
	\]
	Observe that $\dim L/fL < \infty$ since $fL$ is open and $L$ is linearly compact.
\end{proof}
%\subsection{Relationship of residues under extensions}
%Finally, we explore the case where $K'$ is a commutative $k$-algebra containing $K$. We will examine $\Omega_{K'/k}$ and $\Omega_{K/k}$ and the relationship between their residues. In this case the injection $K \to K'$ induces a map between $\Omega_{K/k} \to \Omega_{K'/k}$ which may not be injective. 
%\begin{proposition}\label{prop:residue-commutes-extension}
%	Let $V$ be a Tate space such that multiplication by any $f\in K'$ induces a continuous endomorphism in $\End_{k}(V)$. Therefore, for all $g\in K$ multiplication by $g$ is continuous as well. Hence, we can define
%\[
%	\res_{V}\colon\Omega_{K/k} \to k, \quad\text{and}\quad \res_{V}'\colon \Omega_{K'/k}\to k.
%\]
%In this situation, the diagram
%\[
%\begin{tikzcd}
%	\Omega_{K/k} \arrow[r] \arrow[dr, swap, "\res_{V}"] & \Omega_{K'/k} \arrow[d, "\res_{V}'"] \\
%	& k  
%\end{tikzcd}
%\]
%commutes.
%\end{proposition}
%\begin{proof}
%	For $f,g \in K$ their residue symbol is independent whether $f\,dg$ is thought as an element in $\Omega_{K'/k}$ or $\Omega_{K/k}$. This observation implies the commutativity of the diagram.
%\end{proof}
%Now, assume that $K'$ is free $K$-module of finite rank $n$ and consider the tensor product $V' = K' \otimes_{K} V$. Since the tensor product and direct sum commute, it follows that $V' \cong K^{n} \otimes_{K} V \cong (K \otimes_{K} V)^{n}\cong V^{n}$. In coordinates, if $(x_{i})$ is a $K$-base for $K'$ then the map $(v_{1}, \ldots, v_{n}) \mapsto x_{1}\otimes v_{1} + \ldots + x_{n}\otimes v_{n}$ is an isomorphism. With the topology induced by this isomorphism $V'$ is a Tate space.
%\begin{proposition}\label{prop:identification-matrices-entries-in-endomorphisms}
%	The space $\End(V')$ is isomorphic to the space of $n \times n$ matrices with entries in $\End(V)$ denoted $\Mat_{n}(\End_{0}(V))$. Moreover, if $K$ acts continuously on $V$ so does $K'$ on $V'$.
%\end{proposition}
%\begin{proof}
%	Let $\varphi$ be a continuous $k$-endomorphism of $V'$, then there exists a unique set $\{\varphi_{ij}\}_{i,j=1}^{n}$ contained in $\End(V)$ such that
%	\[
%	\varphi\left(\sum_{i} x_{i} \otimes v_{i}\right) = \sum_{i,j} x_{i} \otimes \varphi_{ij}(v_{j})
%	\]
%	for all $v_{1}, \ldots, v_{n} \in V$. Now, let $f' \in K'$, then
%	\[
%		f'x_{i} = \sum f_{ij}x_{j}
%	\]
%	where $f_{ij}\in K$. Since $f_{ij}\in\End(V)$ it follows that $f' \in \End(V')$ by the description of our topology in $V'$.
%\end{proof}
%Let $\End'_{0}(V')$ be the inverse image of $\Mat_{n}(\End_{0}(V))$ under the isomorphism in \cref{prop:identification-matrices-entries-in-endomorphisms}. Note that $\End'_{0}(V')\subseteq \End_{0}(V')$. Therefore, the map
%\[
%	\tr_{V'}\colon \End'_{0}(V') \to k
%\]
%is well-defined. 
%\begin{proposition}\label{prop:trace-is-trace}
%	For $\varphi \in \End'_{0}(V')$ the identity
%	\[
%		\tr_{V'}(\varphi) = \sum_{i}\tr_{V}(\varphi_{ii})
%	\]
%	holds.
%\end{proposition}
%\begin{proof}
%	Write $(\varphi_{ij})$ as a sum of a strictly lower triangular,  strictly upper triangular and diagonal matrix. Namely,
%	\[
%		\varphi = \varphi_{LT} + \varphi_{UT} + \varphi_{D}, 
%	\]
%	where $\varphi_{LT}$, $\varphi_{UT}$ and $\varphi_{D}$  have a matrix representation of a strictly lower, strictly upper and diagonal matrix respectively. Observe that $\varphi_{LT}$, $\varphi_{UT}$, $\varphi_{D}$ belong to $\End'_{0}(V')$ and $\varphi_{LT}$ and $\varphi_{UT}$ are nilpotent. By \cref{lemm:properties-trace} it follows that
%	\[
%		\tr_{V'}(\varphi) = \tr_{V'}(\varphi_{D}).
%	\]
%	On the other hand, by definition
%	\[
%		\tr_{V'}(\varphi_{D}) = \sum \tr_{V}(\varphi_{ii}).
%	\]
%\end{proof}


%\begin{theorem}\label{thm:residue-of-finite-rank}
%	For all $f' \in K'$ and $g \in K$ the equality
%	\[
%		\res_{V}'(f' dg) = \res_{V}((\tr_{K'/K}(f')dg))
%	\]
%	holds.
%\end{theorem}
%\begin{proof}
%	Let $L$ be a c-lattice in $V$ then $L' = x_{1} \otimes L + \ldots + x_{n} \otimes L$ is a c-lattice in $V'$. Let $\pi\colon V\to L$ be a linear continuous projection and $\pi'$ be the corresponding element to $(\delta_{ij}\pi)$ under the isomorphism $\End(V') \cong \Mat_{n}(\End(V))$. Therefore, $\pi'\colon V' \to L'$ is a linear continuous projection. On the other hand, let $f'\in K'$ and $g\in K$. Then, $f'$ corresponds to $(f_{ij})\in \Mat_{n}(K)$ and let $g'$ be the corresponding element to $(\delta_{ij} g)$ in $\End(V')$. Hence, the commutator $[\pi'f', g']$ is mapped to $[\pi f_{ij}, g]$ by the map $\End(V') \to \Mat_{n}(\End(V))$. By \cref{prop:trace-is-trace}, it follows that
%	\begin{align*}
%		\res_{V'}(f' dg) &= \tr_{V'}([\pi'f', g'])  \\
%		&= \sum \tr_{V}([\pi f_{ii}, g]) \\
%		&= \sum \res_{V}(f_{ii} dg) \\
%		&= \res_{V}\left(\left(\sum f_{ii}\right)dg\right) \\
%		&= \res_{V}\left(\tr_{K'/K}(f')dg\right).
%	\end{align*}
%\end{proof}


%!TEX root = ../main.tex
\chapter{Algebraic curves}\label{ch:algebraic-curves}
In the preceding chapter we presented the ``residue map'' in an abstract context. In this chapter, we explore residues on algebraic curves using Tate's construction. 
%\section{Sheaf Theory} 
%We just assume (as in previous chapters) basic homological algebra and algebraic geometry (e.g. the first chapter in \cite{hartshorne}).

%Let $X$ be a topological space.
%\subsection{Presheaves and sheaves}
%\begin{definition}\label{def:presheaf}
%	A \textbf{presheaf} $\mathscr F$ over $X$ is a functor from the opposite category of open sets in $X$ ($U \to V \iff V\subseteq U$ ) to an arbitrary category $\mathcal{C}$ 
%	\[
%		\mathscr F\colon \mathsf{Op}(X)^{\mathtt{op}} \to \mathcal{C}.
%	\]
%	If $U \subseteq V$ then $\mathscr{F}(V) \to \mathscr{F}(U)$. We will call such map $\res^{V}_{U}$ (not to be confused with \textit{residue}) the \textit{restriction} of $V$ to $U$. If $\mathcal{C}$ is $\mathsf{Ab}$ or $\mathsf{Mod}_{A}$ we will simply say a presheaf of abelian groups or $A$-modules respectively.
%\end{definition}
%\begin{definition}\label{def:sheaf}
%	A \textbf{sheaf} $\mathscr F$ on $X$ is a presheaf such that for every open set $U \subseteq X$ and every open cover $\mathscr U = \{U_{\alpha}\}_{\alpha}$ of $U$ the diagram
%	\[
%		\begin{tikzcd}
%			\mathscr{F}(U)\arrow[r] & \prod_{\alpha}\mathscr{F}(U_{\alpha}) \arrow[r, shift right] \arrow[r, shift left] & \prod_{\alpha, \beta} \mathscr{F}(U_{\alpha} \cap U_{\beta})
%		\end{tikzcd}
%	\]
%	is an equalizer. This equalizer can be understood by the following two axioms usually called \textit{sheaf axioms}
%	\begin{enumerate}[label = (\roman*)]
%		\item If $s,t \in \mathscr{F}(U)$ and $\res^{U}_{U_{\alpha}}(s) = \res^{U}_{U_{\alpha}}(t)$ for all $\alpha$ then $s = t$.
%		\item For a collection $\{s_{\alpha}\}_{\alpha}$ such that $\res^{U_{\alpha}}_{U_{\alpha}\cap U_{\beta}}(s_{\alpha}) = \res^{U_{\beta}}_{U_{\alpha}\cap U_{\beta}}(s_{\beta})$ for all $\alpha$ and $\beta$ there exists some $s \in \mathscr{F}(U)$ such that $\res^{U}_{U_{\alpha}}(s) = s_{\alpha}$.
%	\end{enumerate}
%\end{definition}
%\begin{example}\label{ex:examples-of-sheaves}
%	\begin{enumerate}[label = (\alph*)]
%		\item For all $U$ open in $X$ let $C(U)$ denote the ring of continuous functions $U \to \R$. This is a sheaf of rings over $X$. 
%		\item The example we are more interested in is the \textbf{structure sheaf} on a projective variety. That is, for a projective variety $X$ we denote by $\O_{X}(U)$ the ring of regular functions defined on $U$. It is clear that it is a presheaf of rings. The other conditions follow from the fact that a regular function is zero if and only if it is locally zero and it is locally regular if and only if it is locally regular.
%		\item As observed in the previous two examples, sheaves encode local behavior. For instance, let $\mathscr{G}(U)$ denote the ring of continuous constants functions $U \to \R$. In this case $\mathscr{G}$ in a presheaf but not a sheaf (if $X$ has at least two non trivial open sets), this follows from the fact that a locally constant function is not necessarily constant.
%	\end{enumerate}
%\end{example}
%The collection of presheaves of $\mathcal{C}$ over $X$ is a category. Indeed a morphism between two presheaves $\mathscr{F}$ and $\mathscr{G}$ is a natural transformation $\varphi\colon \mathscr{F} \to \mathscr{G}$. That is, for all $U \subseteq V$ open in $X$ the diagram
%\[
%	\begin{tikzcd}
%		\mathscr{F}(V) \arrow[r, "\varphi(V)"] \arrow[d, "\res^{V}_{U}"] & \mathscr{G}(U) \arrow[d, "\res^{V}_{U}"] \\
%		\mathscr{F}(U) \arrow[r, "\varphi(U)"] & \mathscr{G}(U)
%	\end{tikzcd}
%\]
%commutes. We denote $\mathsf{PSh}(X)$. Analogously, the collection of sheaves of $\mathcal{C}$ over $X$ is also a category. We denote it by $\mathsf{Sh}(X)$. 

%\subsection{Sheaf cohomology}

\section{Basic theory of algebraic curves}
In this section we recall briefly and with little detail the basic theory of algebraic projective curves. For a complete exposition we reference the reader to \cite{curves} and \cite{hartshorne}. We will borrow many results from commutative algebra, most of them can be found in \cite{Matsumura}, \cite{comm-alg} and \cite{atiyah}.

Let $k$ be an algebraically closed field. Let $(X, \O_{X})$ be an algebraic projective variety. Then 
\[
	k(X) := \varinjlim_{U\subseteq X} \O_{X}(U)
\]
 is the \textbf{function field} or \textbf{field of rational functions} of $X$. In addition, consider the stalk 
\[
	\O_{X,p} := \varinjlim_{p\in U\subseteq X} \O_{X}(U)
\]
of regular functions near $p$. We obtain natural injections 
\[
 	\O_{X}(X) \to \O_{X,p} \to k(X).
\] 
\begin{proposition}\label{prop:fraction-field-local-function-field}
	The fraction field of $\O_{X,p}$ is $k(X)$.
\end{proposition}
\begin{proof}
	Let $U \subseteq X$ be an affine neighborhood of $p$. Suppose that $A$ is the coordinate ring of $X$ on $U$ and let $\mathfrak{p}$ be the maximal ideal of $A$ corresponding to $p$. Therefore, $A_{\mathfrak{p}} = \O_{X,p}$. Since $U$ is affine $\O_{X}(U) = A$ and $k(U) = \Frac{A}$. Moreover, irreducibility of $X$ implies $k(X) = k(U)$. Hence, 
	\[
		k(X) = k(U) = \Frac{A} = \Frac{A_{\mathfrak{p}}} = \Frac{\O_{X,p}}.
	\]
\end{proof}

In addition, $\O_{X,p}$ is a noetherian local ring of Krull dimension $\dim X$.
Its maximal ideal of regular functions near $p$ that vanish in $p$ is denoted $\mathfrak{m}_{p}$. Observe that evaluation at $p$ yields the isomorphism $\O_{X,p}/\mathfrak{m}_{p} \cong k$.

\subsection{Smoothness and completeness}
\begin{definition}\label{def:regular-local-ring}
	A local ring $(A, \mathfrak{m})$, where $\mathfrak{m}$ denotes its maximal ideal, is called \textbf{regular} if $\dim_{A/\mathfrak{m}} \mathfrak{m}/\mathfrak{m}^{2} = \dim A$. 
\end{definition}
Let $(A,\mathfrak{m})$ be a noetherian regular local ring. Let $k = A/\mathfrak{m}$ be its residue field. In this situation, $(A,\mathfrak{m})$ carries a natural topology, called the $\mathfrak{m}$-adic topology. Namely, $\{\mathfrak{m}^{n}\}_{n\geq 1}$ is a system of neighborhoods around zero and we let the topology to be translation invariant. We already mentioned this topology briefly in \cref{ex:tate-spaces} for the polynomial ring. The $\mathfrak{m}$-adic topology is separated. Indeed, 
\[
	\bigcap_{n\geq 1} \mathfrak{m}^{n} = \{0\}
\]
by Krull intersection theorem. See Theorem (18.29) in \cite{comm-alg}. Just as in \cref{def:completion} we define the \textbf{completion} $\widehat{A}$ of $A$ to be 
\[
	\widehat{A} := \varprojlim_{n\geq 1} A/\mathfrak{m}^{n}.
\]
There is a natural map $A \to \widehat{A}$. In particular, since $A$ is separated this map is injective. When this map is an isomorphism we say that $A$ is \textbf{complete}. We summarize several properties of completion in the following theorem:
\begin{theorem}\label{thm:properties-adic-completion}
	Let $(A,\mathfrak{m})$ be a noetherian regular local ring. Then
	\begin{enumerate}[label = (\alph*)]
		\item $\widehat{A}$ is a noetherian regular local ring and $\widehat{\mathfrak{m}}$ is its maximal ideal.
		\item Krull dimension is preserved under completion, that is, $\dim A = \dim \widehat{A}$.
		\item (Cohen structure theorem) If $\dim A = n$, then
		\[
			\widehat{A} \cong k\left[[t_{1}, \ldots, t_{n}]\right].
		\]
		Where $t_{1}, \ldots t_{n}$ are mapped to generators of $\mathfrak{m}$.
	\end{enumerate}
	\begin{proof}
	See Chapter 22 in \cite{comm-alg}.  
	\end{proof}
	
\end{theorem}
Now, we explore these results in the geometrical setting. 
\begin{definition}\label{def:}
	If $\O_{X,p}$ is a regular local ring, that is, $\dim_{k} \mathfrak{m}_{p}/\mathfrak{m}_{p}^{2} = \dim \O_{X,p} = \dim X$, we say that $X$ is \textbf{smooth at} $p$. Naturally, $X$ is called \textbf{smooth} if it is smooth at every point $p\in X$.
\end{definition}
We get the following result immediately from \cref{thm:properties-adic-completion}.
\begin{corollary}\label{cor:smooth-iff-isomorphic-to-power-series}
	If $X$ is smooth then $\widehat{\O_{X,p}} \cong k[[t_{1}, t_{2}, \ldots, t_{n}]]$ where $n = \dim X$.
\end{corollary}
Now, we focus in one-dimensional varieties.
\begin{definition}\label{def:algebraic-curve}
	An \textbf{algebraic curve} is a one-dimensional smooth variety.
\end{definition}
In dimension $1$ smoothness can be interpreted in the language of valuations.
\subsection{Valuation theory}
Let $k$ be a field. 
\begin{definition}\label{def:discrete-valuation}
	A \textbf{discrete valuation} is a surjective group homomorphism $\nu\colon k^{\times} \to \Z$ such that, for every $x\in k^{\times}$ and $y \neq -x$ in $k^{\times}$
	\[
		\nu(x + y) \geq \min\{\nu(x), \nu(y)\}.
	\]
	As a convention, we let $\nu(0) = \infty$. We denote by
	\[
	A_{\nu} = \{x \in k\colon \nu(x)\geq 0\}
	\]
	the \textbf{discrete valuation ring} or \textbf{DVR} of $\nu$. Clearly, $A$ is a subring, thus a domain. Consider
	\[
		\mathfrak{m}_{\nu} = \{x\in k\colon \nu(x) > 0\}. 
	\]
	Notice that, if $x\in k$, but $x\notin A_{\nu}$, then $x^{-1}\in \mathfrak{m}_{\nu}$. Hence, $\operatorname{Frac}(A_{\nu}) = K$. Further, observe that
	\[
	A_{\nu}^{\times} = A_{\nu} - \mathfrak{m}_{\nu}.
	\]
	Therefore, $A_{\nu}$ is a local domain with maximal ideal $\mathfrak{m}_{\nu}$. An element $t \in \mathfrak{m}_{\nu}$ with $\nu(t) = 1$ is called a \textbf{uniformizing parameter}. Such $t$ is irreducible, because if $t = ab$ with $\nu(a)\geq 0$ and $\nu(b)\geq 0$ implies $\nu(a) = 0$ or $\nu(b) = 0$ since $1 = \nu(a) + \nu(b)$. Further, any $x \in k^{\times}$ has the unique factorization $x = u t^{n}$ where $u \in A_{\nu}^{\times}$ and $n := \nu(x)$. Moreover, $A_{\nu}$ is a principal ideal domain. In fact, any nonzero ideal $\mathfrak{a} \subseteq A_{\nu}$ has the form
	\[
		\mathfrak{a} = \left\langle t^{m}\right\rangle \quad\text{where}\quad m:=\min\{\nu(x)\colon x\in \mathfrak{a}\}.
	\]
	Indeed, given a nonzero $x \in \mathfrak{a}$, say $x = ut^{n}$ where $u \in A_{\nu}^{\times}$. Then $t^{n}\in \mathfrak{a}$. Son $n \geq m$. Set $y := ut^{n-m}$. Then $y\in A_\nu$ and $x = yt^{m}$, as desired. Finally, $\mathfrak{m} = \langle t\rangle$ and $\dim A_{\nu} = 1$. Therefore, $A$ is regular local of dimension one.
\end{definition}
We have the following characterization theorem for DVRs.
\begin{theorem}\label{thm:characterization-of-DVRs}
	Let $A$ be a noetherian one-dimensional local ring, $\mathfrak{m}$ its maximal ideal and $k = A/\mathfrak{m}$ its residue field. Then these conditions are equivalent:
	\begin{enumerate}[label = (\roman*)]
		\item $A$ is a DVR.
		\item $A$ is integrally closed.
		\item $\mathfrak{m}$ is principal.
		\item $\dim_{k}(\mathfrak{m}/\mathfrak{m}^{2}) = 1$.
		\item Every non-zero ideal is a power of $\mathfrak{m}$.
	\end{enumerate}
\end{theorem}
\begin{proof}
	See Proposition 9.2 in \cite{atiyah}.
\end{proof}
\begin{corollary}\label{cor:smoothness-DVR-curves}
	Let $X$ be a one-dimensional variety. Then, $X$ is smooth if and only if $\O_{X,p}$ is a DVR for all $p$.
\end{corollary}
\begin{example}\label{ex:stalk-of-regular-functions-as-a-DVR}
	Let $(X,\O_{X})$ be an algebraic curve. Let $p \in X$ and consider $\O_{X,p}$. In this case $\O_{X,p}$ is a DVR. Let $t_{p}\in \O_{X,p}$ be an uniformizing parameter. Then, if $f \in k(X)$ it follows that $f = ut_{p}^{n}$ for some  unit $u\in \O_{X,p}$ and $n \in \Z$. Then, $\nu_{p}(f) = n$.
\end{example}
\subsection{Morphisms}
Let $(X,\O_{X})$ and $(Y,\O_{Y})$ be two algebraic curves. 
\begin{proposition}\label{prop:non-constant-morphism-is-dominant}
	A non-constant morphism $\varphi\colon X\to Y$ is \textbf{dominant}, i.e, the image of $X$ is dense in $Y$. Therefore, every non-constant morphism from $X$ to $Y$ yields a field extension $k(Y)\subseteq k(X)$.
\end{proposition}
\begin{proof}
	We have that $Z = \overline{\varphi(X)}$ is closed and irreducible in $Y$. Suppose that $Z \neq Y$. We may intersect with affine space such that $Z \cap \A^{n}\neq \emptyset$. It follows that $Z\cap \A^{n}\neq Y\cap \A^{n}$, otherwise their projective closures will coincide. Hence, the ideal defined by $Z$ is properly contained in the ideal defined by $Y$. Since $\dim Y = 1$ it follows that $\dim Z = 0$, a contradiction.
\end{proof}

\begin{proposition}\label{prop:non-constant-morphism-is-surjective}
	A non-constant morphism $\varphi\colon X\to Y$ is surjective. 
\end{proposition}

\begin{proof}
	The map yields a field extension $k(Y) \subseteq k(X)$. Given a point $p\in Y$ consider the corresponding DVR $A = \O_{Y,p}\subseteq k(Y)$. Let $B_{0}$ be the integral closure of $A$ in $k(X)$. If we localize at a maximal ideal of $B_{0}$ we obtain a DVR $B \subseteq k(X)$. Then $B = \O_{X,q}$ for some $q \in X$. Then $q \mapsto p$. 
\end{proof}


\subsection{Divisors}
Let $X$ be an algebraic curve. 
\begin{definition}\label{def:divisors}
	A \textbf{divisor} $D$ on $X$ is a finite formal sum of points on $X$, namely
	\[
		D = \sum_{p\in X} n_{p}p, \quad n_{p}\in \Z,\text{ and }n_{p} = 0 \quad\text{for almost all } p.
	\]
	The \textbf{degree} of $D$ is
	\[
		\deg D := \sum_{p\in X}n_{p}.
	\]
	We say that a divisor $D$ is \textbf{effective} and write $D\geq 0$ if $n_{p}\geq 0$ for all $p\in X$. We write $D\geq D'$ if the difference $D - D'$ is an effective divisor. \\
	If $f\in k(X)$ is a non-zero rational function on $X$, the \textbf{principal divisor} $(f)$ is defined by
	\[
		(f):=\sum_{p\in X}\nu_{p}(f)p.
	\]
	In other words, $(f)$ is the sum of the zeroes of $f$ minus the sum of poles of $f$, counting multiplicities. Observe that $(f) + (g) = (fg)$. 

	We say that two divisors $D$ and $D'$ are \textbf{linearly equivalent} if the differ by a principal divisor, that is, there exists some rational function $f$ such that $D + (f) = D'$.

	Given a divisor $D = \sum_{p}n_{p}p$ we define the $k$-vector space
	\[
		L(D) := \{f\in k(X)\colon f = 0\text{ or } \nu_{p}(f) \geq -n_{p}\}.
	\]
	For instance, if $D$ is effective $L(D)$ consists of rational functions having a pole at $p\in X$ of order at worst $n_{p}$ for each $p\in X$. We write $\ell(D) := \dim_{k}(L(D))$.
\end{definition}
%\begin{proposition}\label{prop:degree-of-principal-divisor}
	%For all $f\in k(X)^{\times}$ the principal divisor of $f$ has zero degree.
%\end{proposition}
%\begin{proof}
	
%\end{proof}

\section{Tate spaces over algebraic curves}
Let $(X,\O_{X})$ be an algebraic curve over an algebraically closed field $k$.
\subsection{Residue in coordinates}
Let $K := k(X)$. For all $p \in X$ we use the following notation $L_{p} := \widehat{\O_{X,p}}$ and $K_{p}:=\Frac L_{p}$. From \cref{thm:properties-adic-completion} and $\cref{thm:characterization-of-DVRs}$ it follows that $L_{p}$ is a DVR. Let $t_{p}$ be an uniformizing parameter in $L_{p}$. The topology in $K_{p}$ defined by letting $\{t_{p}^{n}L_{p}\}_{n\in \Z}$ be a system of neighborhoods of zero in $K_{p}$. This system is compatible with the valuation induced by $L_{p}$.
\begin{proposition}\label{prop:complete-fraction-field-is-a-Tate-space}
	$K_{p}$ is a Tate space and $L_{p}$ is a c-lattice in $K_{p}$.
\end{proposition}
\begin{proof}
	Observe that the map $\mathfrak{m}_{p}^{n}\O_{X,p}/\mathfrak{m}_{p}^{n+1}\O_{X,p} \to \O_{X,p}/\mathfrak{m}_{p}\O_{X,p}$ induced by inclusions is an isomorphism. Then, the exactness of
	\[
		0 \to \mathfrak{m}_{p}^{n}\O_{X,p}/\mathfrak{m}_{p}^{n+1}\O_{X,p} \to \O_{X,p}/\mathfrak{m}_{p}^{n+1}\O_{X,p} \to \O_{X,p}/\mathfrak{m}_{p}^{n}\O_{X,p} \to 0
	\]
	implies that every quotient $\O_{X,p}/\mathfrak{m}_{p}^{n}\O_{X,p}$ is finite-dimensional over $k$. Therefore, 
	\[
		L_{p} = \varprojlim_{n\geq 1} \O_{X,p}/\mathfrak{m}_{p}^{n}\O_{X,p}
	\]
	is an inverse limit of finite-dimensional $k$-vector spaces. Hence, $L_{p}$ is complete and linearly compact by \cref{ex:tate-spaces}. Since $L_{p}$ is open in $K_{p}$ then it is a c-lattice and $K_{p}$ is a Tate space.
\end{proof}
\begin{remark}\label{rem:mutually-commensurable-system}
\begin{enumerate}[label = (\alph*)]
	\item Observe that $\{t_{p}^{n}L_{p}\}_{n\in \Z}$ is a mutually commensurable system of neighborhoods around zero of consisting of $k$-vector subspaces of $K_{p}$. 
	\item Let $f\in K_{p}$, observe that $fL_{p} = t_{p}^{n}L_{p}$ for some $n \in \Z$ and uniformizing parameter $t_{p} \in L_{p}$. It follows that multiplication by $f$ in $K_{p}$ is continuous in $K_{p}$, that is, $K_{p}$ (and particularly $K$) acts continuously over itself.
\end{enumerate}
\end{remark}

\begin{definition}\label{def:residue-at-p}
	Let $f,g \in K_{p}$. We define the residue of the differential $fdg$ at $p\in X$ to be
	\[
		\res_{p}(fdg) = \res_{K_{p}}(fdg).
	\]
	where $\res_{K_{p}}$ denotes the abstract residue defined in \cref{thm:existence-of-residue}.
\end{definition} 
\begin{proposition}\label{prop:no-poles-zero-residue}
	Let $p\in X$. If $\omega \in \Omega_{K_{p}/k}$ has no poles at $p$ then $\res_{p}(\omega) = 0$.
\end{proposition}
\begin{proof}
	Clear from \cref{prop:residue-and-continuity}.
\end{proof}

\begin{theorem}\label{thm:resiude-coincides-with-coefficient}
	Let $f,g \in K_{p}$. By the structure theorem in \cref{thm:properties-adic-completion} it follows that $f = \sum_{\nu \gg -\infty}^{\infty} a_{\nu}t_{p}^{\nu}$ and $g = \sum_{\mu \gg -\infty}^{\infty} b_{\mu}t_{p}^{\mu}$ for some $a_{\nu},b_{\mu} \in k$ and a uniformizing parameter $t_{p}\in L_{p} $. Recall that the formal derivative of $g$ is
	\[
		g' = \sum_{\mu \gg -\infty}^{\infty} \mu b_{\mu}t_{p}^{\mu -1}.
	\]
	Then,
	\[
		\res_{p}(fdg) = \text{coefficient of }t_{p}^{-1}\text{ in } fg'
	\]
	which is given by the Cauchy product
	\[
		\res_{p}(fdg) = \sum_{\mu + \nu = 0}\mu a_\nu{}b_{\mu}.
	\]
\end{theorem}
\begin{proof}
	Let
	\[
		\tilde{f} = \sum_{\nu\gg-\infty}^{N} a_{\nu}t_{p}^{\nu},\quad\text{and}\quad\tilde{g} = \sum_{\mu\gg-\infty}^{N} b_{\mu}t_{p}^{\mu}
	\]
	then 
	\begin{align*}
		\res_{p}(fdg) &= \res_{p}( (\tilde{f} + (f - \tilde{f}))d(\tilde{g} + (g - \tilde{g}))) \\
		&= \res_{p}(\tilde{f}d\tilde{g}) + \res_{p}(\tilde{f}d(g - \tilde{g})) + \res_{p}((f-\tilde{f}) d\tilde{g}) \\ &+ \res_{p}( (f - \tilde{f})d(g-\tilde{g})).
	\end{align*}
	If $N$ is sufficiently large, then by \cref{prop:residue-and-continuity} it follows that 
	\[
		\res_{p}( (f - \tilde{f})d(g-\tilde{g})) = \res_{p}(\tilde{f}d(g - \tilde{g})) = \res_{p}((f-\tilde{f}) d\tilde{g}) = 0.
	\]
	Therefore, we can assume that only finitely many of the $a_{\nu}$ and $b_{\mu}$ are non-zero. Now, $fdg = f g' dt$ and by \cref{prop:residue-of-a-power} only the term of $t_{p}^{-1}$ can have non-zero residue. Then, by \cref{prop:residue-of-invertible-element}, it follows that
	\[
		\res_{p}(t_{p}^{-1}dt_{p}) = \dim_{k}(L_{p}/t_{p}L_{p}) = \dim_{k}k = 1. 	
	\]
	Hence, by $k$-linearity of residue implies the desired conclusion.
\end{proof}
\begin{corollary}\label{cor:invriance-of-residue-coefficient}
	Let $f\in K_{p}$. Then, the coefficient of $t_{p}^{-1}$ in the Laurent series expansion of $f$ is independent of the choice of uniformizing parameter $t_{p}$.	
\end{corollary}
\begin{remark}\label{rem:previouses-approaches-residues}
	Before Tate introduced this approach to residues of differentials on algebraic curves, residues were defined by the formula in \cref{thm:resiude-coincides-with-coefficient}. However, to prove well-definition of such formula, it is necessary to argue that the coefficient of $t_{p}^{-1}$ is independent of the choice of uniformizing parameter $t_{p}$. In $\operatorname{char}k = 0$ one can realize $X$ as an analytical variety and reduce independence to the invariance of the formula
	\[
		\res_{p}(\omega) = \frac{1}{2\pi i}\oint_{p} \omega.
	\]
	Nevertheless, in the general setting it is not obvious why invariance follows. In \cref{cor:invriance-of-residue-coefficient} we gave a clean but theory-demanding proof of such result. We reference the reader to \cite{serre} Chapter 2 Section 10 for a direct proof.
\end{remark}
\subsection{Adèles and the residue theorem}
Our next goal is to prove the residue theorem. In order to prove it, we will take an \textit{adèlic} approach, borrowing many techniques from number theory.
\begin{definition}\label{def:adele-ring-over-algebraic-curve}
	Let $X$ be an algebraic curve. Let $Y \subseteq X$ be any subset. Let $\mathscr{F}$ denote the set of all finite subsets of $Y$. Let $S \in \mathscr{F}$, the $\mathbf{S}$-\textbf{adèle} of $K$ indexed by $Y$ is defined as the product
	\[
		\widetilde{K}_{Y,S} := \prod_{p\in Y\setminus S}L_{p} \times \prod_{p\in S}K_{p}
	\]
	in its product topology. The \textbf{adéle} $\widetilde{K}_{Y}$ of $K$ indexed by $Y$ is the direct limit of the system indexed by $\mathscr{F}$, namely, if $S \subseteq T$ there exists a injection $\iota_{ST}\colon \widetilde{K}_{Y,S} \hookrightarrow \widetilde{K}_{Y,T}$ given by the inclusion. Endow
	\[
		\widetilde{K}_{Y} = \varinjlim_{S \in \mathscr{F}} \widetilde{K}_{Y,S}
	\]
	with its direct limit topology. Usually, for $X = Y$ we will simply write $\widetilde{K}$ for $\widetilde{K}_{X}$.
\end{definition}
The adèle of $K$ is a particular case of a \textbf{restricted product} of a collection of topological spaces. In the literature, this construction is usually defined as a \textit{set} in terms of the product. We give this characterization in the following proposition.
\begin{proposition}\label{prop:adèle-as-subset-of-product}
	\[
		\widetilde{K}_{Y} = \{(f_{p})\colon f_{p} \in K_{p}\text{ for all }p \in Y\text{ and }f_{p}\in L_{p}\text{ for almost all }p\in Y\}
	\]
	where \textit{almost all} means for all but finitely many $p\in Y$, equipped with the following collection as a basis for its topology
	\[
		\left\{\prod_{p\in Y} U_{p} \colon U_{p}\text{ is open for all }p\in Y\text{ and }U_{p} = L_{p}\text{ for almost all }p\in Y\right\}.
	\]
\end{proposition}
\begin{proof}
	Let $K^{\sharp}$ be the topological space defined in the statement of the proposition. Observe that $K^{\sharp}$ is linearly topologized as a $k$-vector space. We will prove that $K^{\sharp}$ satisfies the universal property of $\widetilde{K}_{Y}$ in the category $\mathsf{LinTop}_{k}$. First, observe that the inclusion
	\[
		\widetilde{K}_{Y,S} \hookrightarrow K^{\sharp}
	\]
	is a continuous homomorphism and the diagram
	\[
		\begin{tikzcd}
			& K^{\sharp} & \\
			\widetilde{K}_{Y,S} \arrow[rr, "\iota_{ST}", hook]\arrow[ur, hook] & & \widetilde{K}_{Y,T} \arrow[lu, hook]
		\end{tikzcd}
	\]
	commutes. Moreover, for every $P$ equipped with continuous homomorphisms $\varphi_{S}\colon \widetilde{K}_{Y,S}\to P$ such that the diagram 
	\[
		\begin{tikzcd}
			& P & \\
			\widetilde{K}_{Y,S} \arrow[rr, "\iota_{ST}", hook]\arrow[ur, hook, "\varphi_{S}"] & & \widetilde{K}_{Y,T} \arrow[lu, hook, "\varphi_{T}", swap]
		\end{tikzcd}
	\]
	commutes. Then, define $\varphi\colon K^{\sharp} \to P$ as follows: for $(f_{p}) \in K^{\sharp}$ there exists a $S\in \mathscr{F}$ such that $f_{p} \in L_{p}$ if and only if $p \in S$. Define $\varphi( (f_{p})_{p\in Y} ) = \varphi_{S}( (f_{p})_{p\in Y} )$. This is a continuous homomorphism and it is the only one such that the diagram
	\[
		\begin{tikzcd}
			& P & \\
			& K^{\sharp} \arrow[u, "\varphi", dotted] & \\
			\widetilde{K}_{X,S} \arrow[rr, "\iota_{ST}"]\arrow[uur, "\varphi_{S}"]\arrow[ur, hook] & & \widetilde{K}_{X,T} \arrow[luu, "\varphi_{T}", swap] \arrow[lu, hook]
		\end{tikzcd}
	\]
	commutes. This implies $K^{\sharp} = \widetilde{K}_{Y}$ (or canonically isomorphic).
\end{proof}
\begin{remark}\label{rem:adèle-not-subspace-topology}
	Observe that the topology in $\widetilde{K}_{Y}$ is finer than the one it inherits as a subspace of the product $\prod_{p\in Y}K_{p}$. For instance, observe that 
	\[
		\widetilde{L}_{Y} := \prod_{p\in Y}L_{p} 
	\] 
	is open in $\widetilde{K}$, but it is open in $\prod_{p\in Y} K_{p}$ if and only if $Y$ is finite. 
\end{remark}
If $D = \sum_{p}n_{p}p$ is a divisor, let
\[
	\widetilde{K}(D) = \{(f_{p})_{p}\in \widetilde{K}\colon \nu_{p}(f_{p}) \geq -n_{p}\}
\]
a $k$-vector subspace of $\widetilde{K}$. By the description we gave in \cref{prop:adèle-as-subset-of-product} $\widetilde{K}(D)$ is open for all divisors $D$ and they form a system of neighborhoods around zero of mutually commensurable vector subspaces in $\widetilde{K}$. In this notation, $\widetilde{K}(0) = \widetilde{L}$.
\begin{proposition}\label{prop:adèle-is-a-tate-space}
	$\widetilde{K}_{Y}$ is a Tate space and $\widetilde{L}_{Y}$ is a c-lattice in $\widetilde{K}_{Y}$. 
\end{proposition}
\begin{proof}
	First, observe that
	\[
		\prod_{p\in Y} L_{p} = \varprojlim_{p\in Y} L_{p} = \varprojlim_{p\in Y}\varprojlim_{n\geq 1} \O_{Y,p}/\mathfrak{m}_{p}^{n}\O_{Y,p} = \varprojlim_{(p,n)\in Y\times \N^{\geq 0}} \O_{Y,p}/\mathfrak{m}_{p}^{n}\O_{Y,p}
	\] 
	for $X$ realized as trivial category. Hence, $\widetilde{L}_{Y}$ is the inverse limit of a projective system of finite dimensional $k$-vector spaces. Therefore, $\widetilde{L}_{Y}$ is a complete linearly compact vector space over $k$. Since $\widetilde{L}_{Y}$ is open in $\widetilde{K}_{Y}$, it is a c-lattice in $\widetilde{K}_{Y}$ and $\widetilde{K}_{Y}$ is a Tate space.
\end{proof}

\begin{proposition}\label{prop:K-discrete-in-adèle}
	$K$ is realized as a discrete vector subspace of $\widetilde{K}$ by means of the diagonal embedding $f \mapsto (f)_{p\in X}$. 
\end{proposition}
\begin{proof}
	Observe that 
	\[
		K\cap \widetilde{L} = \bigcap_{p\in X}\O_{X,p} = \O_{X}(X).
	\]
	Indeed, the first equality is obvious and the second follows from the fact that a regular function is globally defined on $X$ if and only if it is regular at every point $p$. Since $X$ is projective and geometrically irreducible, $\O_{X}(X) \cong k$ (see, e.g \cite{hartshorne} Chapter 1 Theorem 3.4). Therefore, $K\cap \widetilde{L}$ is a finite-dimensional $k$-vector space, thus discrete. Since $\widetilde{L}$ is open, it follows that $K$ is discrete. 
\end{proof}
Our next objective is to show that $\widetilde{K}/K$ is linearly compact. To prove this, for a divisor $D = \sum_{p\in X}n_{p}p$ we study the quotient
\[
	I(D) := \widetilde{K}(D)/(\widetilde{K}(D) + K).
\]
By construction we have the following exact sequence
\[
	0 \to L(D) \to K \to \widetilde{K}/\widetilde{K}(D) \to I(D) \to 0.
\]
An element $(\overline{f}_{p})_{p}\in \widetilde{K}/\widetilde{K}(D)$ can be described as follows: at each point $p \in X$, $\overline{f}_{p}$ is given by a \textbf{Laurent tail}
\[
	\overline{f}_{p} = a_{\nu}t_{p}^{\nu} + a_{\nu + 1}t_{p}^{\nu+1} + \ldots + a_{-n_{p}-1}t_{p}^{-n_{p}-1}
\]
where $t_{p}$ is a uniformizing parameter at $p$, $\nu\in \Z$, $\nu\leq n_{p}$, and $a_{i}\in k$. Moreover, only finitely many $\overline{f}_{p}$ are non-zero.
\begin{example}\label{ex:I(D)-for-P^1}
	Let $X = \P^{1}$. We claim $I(0) = 0$. Indeed, an element $\overline { f } \in \widetilde{K} / \widetilde{K} ( 0 )$ is a finite collection of Laurent tails as above, where $n _ { p } = 0$ for all $p .$ Choose an affine coordinate $x = X _ { 1 } / X _ { 0 }$ on $\mathbb { P } ^ { 1 }$ such that the points $p _ { 1 } , \ldots , p _ { k }$ such that $\overline { f } _ { p } \neq 0$ lie in the affine piece $\A _ { x } ^ { 1 } = \left( X _ { 0 } \neq 0 \right) \subset \mathbb { P } ^ { 1 } ,$ with $x$ coordinates $\alpha _ { 1 } , \ldots , \alpha _ { k } .$ Then $x - \alpha _ { i }$ is a local parameter at $p _ { i } ,$ and we can write 
	\[
		\overline { f } _ { p_ { i } } = g _ { i } = a _ { \nu _ { i } , i } \left( x - \alpha _ { i } \right) ^ { \nu _ { i } } + \cdots + a _ { - 1 , i } ( x - \alpha_{1} ) ^ { - 1 }
	\]
	Define $g = \sum g _ { i } \in k ( X ) ,$ then $g$ has Laurent tail $\overline { f } _ { p _ { i } }$ at $p _ { i }$ and is regular elsewhere, that is, $g \mapsto \overline { f } \in \widetilde{K}/ \widetilde{K} ( 0 ) .$ Hence $I ( 0 ) = 0$ as claimed.
\end{example}
\begin{lemma}\label{lemm:inequality-divisors-yields-surjection}
	Suppose $D\leq D'$. Then, there is a natural surjection $I(D) \to I(D')$, and the kernel has dimension
	\[
		(\deg D' - \ell(D')) - (\deg D - \ell(D)).
	\]
\end{lemma}
\begin{proof}
	By definition $\widetilde{K}(D) \subseteq \widetilde{K}(D')$, so $I(D)$ surjects onto $I(D')$. Consider the commutative diagram 
	\[
	\begin{tikzcd}
		0  \ar[r] &  K/L(D) \ar[d] \ar[r] & \widetilde{K}/\widetilde{K}(D) \ar[d] \ar[r] & I(D) \ar[d] \ar[r] & 0 \\
		0 \ar[r] &  K/L(D')  \ar[r] & \widetilde{K}/\widetilde{K}(D')  \ar[r] & I(D')  \ar[r] & 0
	\end{tikzcd}
	\]
	The vertical arrows are surjective, so the kernels form an exact sequence
	\[
	0 \to L(D')/L(D) \to \widetilde{K}(D')/\widetilde{K}(D) \to M \to 0
	\]
	where $M = \ker(I(D) \to I(D'))$. Finally, observe that $\widetilde{K}(D')/\widetilde{K}(D)$ has dimension $\deg D' - \deg D$. Indeed, writing $D = \sum_{p}n_{p}p$ and $D' = \sum_{p}n_{p}'p$, an element $(\overline{f}_{p})_{p} \in \widetilde{K}(D')/\widetilde{K}(D)$ is given by Laurent tails 
	\[
	 \overline{f}_{p} =  a _ { - n _ { p } ^ { \prime } } t_{p} ^ { - n _ { p } ^ { \prime } } + a _ { - n _ { p } ^ { \prime } + 1 } t_{p} ^ { - n _ { p } ^ { \prime } + 1 } + \cdots + a _ { - n _ { p } - 1 } t_{p} ^ { - n _ { p } - 1 }
	\]
	where $t_{p}$ denotes a uniformizing parameter at $p$. So, 
	\[
		\dim_{k} \widetilde{K}(D')/\widetilde{K}(D) = \sum_{p}(n_{p}' - n_{p}) = \deg D' - \deg D
	\]
	as claimed. This yields the formula for $\dim_{k}M$.
\end{proof}
\begin{lemma}\label{lemm:bound-on-L(D)}
	There exists $N \in \N$ such that $\deg D - \ell(D)\leq N$ for all divisors $D$ on $X$.
\end{lemma}
\begin{proof}
	Let $f \in K$ be a non-constant rational function and 
	\[
		F = (1 : f)\colon X \to \P^{1}
	\]
	the associated morphism. Let $A = F^{*}(0 : 1)$, so $A$ is the divisor of degree $d$ given by the sum of the poles of $f$ with multiplicities. Let $D$ be a divisor on $X$. We claim that there exists a linearly equivalent divisor $D'$ such that $D' \leq nA$ for some $n \in \N$. Indeed, write $D = \sum_{p}n_{p}p$, and define 
	\[
		h = \prod_{p\in S}(f - f(p))^{n_{p}}
	\]
	where
	\[
		S = \{p \in X\colon n_{p} > 0, f(p)\neq \infty\}.
	\]
	Then $(h) \geq D - nA$ for some $n \in \N$, that is, $D' := D - (h) \leq nA$, as desired. 

	We now establish the result for divisors $D = nA$ for $n\in \N$. The morphism $F$ corresponds to the field extension $k(f) \subseteq k(X)$ of degree $d$. Pick a basis $g_{1}, \ldots, g_{d}$ for $k(X)$ over $k(f)$. Then, by the construction in the previous paragraph, there exist polynomials $q_{i}(t) \in k[t]$ such that $q_{i}(f)g_{i} \in L(n_{0}A)$ for all $1 \leq i \leq d$ and some $n_{0}\in \N$. Indeed, let $D = -(g_{i})$ and define $q_{i}(f) = h$ as above. Then $(hg_{i}) = -D' \leq n_{i}A$ for some $n_{i}\in \N$. Let $n_{0} = \max{n_{i}}$. Then $f^{j}p_{i}(f)g_{i} \in L(nA)$ for each $1 \leq i \leq d$ and $0 \leq j \leq n - n_{0}$. Moreover, these functions are linearly independent over $k(f)$ because $g_{1},\ldots,g_{d}$ are linearly independent over $k(f)$. So,
	\[
		\ell(nA) \geq (n - n_{0} + 1)d = \deg nA - (n_{0} - 1)d,
	\]
	that is, $\deg nA - \ell(nA) \leq N$ where $N:= (n_{0} -1)d$. 

	Combining our results, if $D$ and $D'$ are linearly equivalent and $D' \leq nA$ then 
	\[
		\deg D - \ell(D) = \deg D' - \ell(D') \leq \deg nA - \ell(nA) \leq N. 		
	\]
\end{proof}
\begin{theorem}\label{thm:I(D)-is-finite-dimensional}
	$\dim_{k}I(D) < \infty$.
\end{theorem}
\begin{proof}
	By \cref{lemm:bound-on-L(D)} there exists a divisor $D_{0}$ on $X$ such that $\deg D_{0} - \ell(D_{0})$ is maximal. We claim that $I(D_{0}) = 0$. Indeed, otherwise let $(\overline{f}_{p})_{p}\in I(D)$ nonzero. Pick $D' \geq D_{0}$  $D '= \sum_{p}n_{p}' p$ such that $\nu_{p}(f_{p}) \geq -n_{p}'$ for all $p\in X$, then $(f_{p})_{p}$ lies in the kernel of the surjection $I(D_{0})\to I(D')$. So $\deg D' - \ell(D') > \deg(D_{0}) - \ell(D_{0})$ by \cref{lemm:inequality-divisors-yields-surjection}, a contradiction. 

	If $D\leq D'$ we have a surjection $I(D) \to I(D')$ with finite-dimensional kernel by \cref{lemm:inequality-divisors-yields-surjection}. Thus $I(D) \sim I(D')$. Since $I(D_{0})$ is finite dimensional, we deduce that $I(D)$ is finite dimensional for every divisor $D$.
\end{proof}
\begin{corollary}\label{cor:adele-mod-K-is-linearly-compact}
	$\widetilde{K}/K$ is linearly compact.
\end{corollary}
\begin{proof}
	For every divisor $D$ on $X$ the exact sequence
	\[
	0 \to (\widetilde{K}(D) + K)/ K \to \widetilde{K}/K \to I(D) \to 0,
	\]
	the fact that the collection $\widetilde{K}(D)$ is a system of neighborhoods around zero of vector subspaces in $\widetilde{K}$ and \cref{thm:I(D)-is-finite-dimensional} yield the result.
\end{proof}
\begin{corollary}\label{cor:residue-adéle-trivial}
	For all $p\in X$
	\[
		\res_{\widetilde{K}}(\Omega_{K/k}) = 0.
	\]
\end{corollary}
\begin{proof}
	The statement follows from \cref{prop:linearity-residue-closed-submodule}, \cref{prop:residue-trivial-tate-space}, \cref{prop:K-discrete-in-adèle} and \cref{cor:adele-mod-K-is-linearly-compact}.
\end{proof}

Now, we are ready to prove the residue theorem. 
\begin{theorem}\label{thm:residue-theorem}
	For $X$ an algebraic curve and $\omega \in \Omega_{K/k}$, the identity 
	\[
		\sum_{p\in X}\res_{p}(\omega) = 0
	\]
	holds. 
\end{theorem}
\begin{proof}
	First, observe that the expression $\sum_{p\in X}\res_{p}(\omega) = 0$ makes sense by \cref{prop:no-poles-zero-residue} and the fact that $\omega$ has a finite amount of poles. Now, it is enough to prove the statement for $\omega = fdg$ for $f,g \in K$. Let $p_{1}, \ldots, p_{n}$ be the collection of poles of $f$ and $g$ combined. Then, the product $K_{p_{1}} \oplus K_{p_{2}} \oplus \cdots K_{p_{n}}$ is a closed $K$-submodule of $\widetilde{K}$ (it is the kernel of the projection on the coordinates different than $p_{i}$). Choose $M$ a complementary subspace. Then
	\[
		\widetilde{K} = K_{p_{1}} \oplus K_{p_{2}} \oplus \cdots K_{p_{n}} \oplus M. 
	\] 
	Let $U = \widetilde{L} \cap M$. Then $U$ is a c-lattice in $M$ and $fU \subseteq U$ and $gU \subseteq U$. Hence, \cref{prop:residue-and-continuity} yields
	\[
		\res_{M}(\omega) = 0.
	\]
	Therefore, by \cref{prop:direct-sum-residue} and \cref{cor:residue-adéle-trivial} it follows that
	\[
		0 = \res_{\widetilde{K}}(\omega) = \sum_{p\in X}\res_{p}(\omega).
	\]
\end{proof}
\section{Riemann-Roch formula}
% sheaves
%We claim that $\widetilde{K}/(\widetilde{L} + K)$ is finite dimensional. Consider the exact sequence of abelian sheaves on $X$
%\[
%	0 \rightarrow \O_{X} \rightarrow  { K } ^ { 0 } \stackrel { \delta } { \rightarrow }  { K } ^ { 1 } \rightarrow 0
%\]
%where for all $U$ open in $X$ we let $K^{0}(U) = \operatorname{Im}(K \to \widetilde{K}_{U})$ by the diagonal embedding and $K^{1}(U) = \widetilde{K}_{U}/\widetilde{L}_{U} = \bigoplus_{p\in U}K_{p}/L_{p}$. The map $\delta(U)\colon \widetilde{K}_{U} \to \widetilde{K}_{U}/\widetilde{L}_{U}$ is the natural projection.
\include{FrontBackmatter/Bibliography}
\end{document}